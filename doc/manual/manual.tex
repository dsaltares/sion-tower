% -*-memoria.tex-*-
% Este fichero es parte de la plantilla LaTeX para
% la realización de Proyectos Final de Carrera, protejido
% bajo los términos de la licencia GFDL.
% Para más información, la licencia completa viene incluida en el
% fichero fdl-1.3.tex

% Copyright (C) 2009 Pablo Recio Quijano 

%-------------------------------------------------------
% ---- Plantilla para libros / memorias PFC -----

% Realizada por Pablo Recio Quijano y Noelia Sales Montes 
% Formato de portada y primera página tomado del PFC de
% Francisco Javier Vázquez Púa, en su proyecto 'libgann'
% -------------------------------------------------------

\documentclass[a4paper,11pt]{article}

\usepackage{./estilos/estiloBase} % Basicamente son todas las
                                  % librerias usadas. En caso de que
                                  % falten librerias se van añadiendo
                                  % al fichero.
\usepackage{./estilos/colores}  % Algunos colores ya generados, para
                                % los algunos estilos más avanzados.
\usepackage{./estilos/comandos} % Algunos comandos personalizados

\graphicspath{{./imagenes/}} % Indicamos la ruta donde se encuentran
                             % las imagenes, para ahorrarnos la ruta
                             % completa, y solo modificar aquí si en
                             % un momento dado lo movemos

\begin{document}


\thispagestyle{empty}
\begin{picture}(0,0)
	\put(65,-90){\includegraphics[scale=0.5]{imagenes/logo-siontower.png}}
\end{picture}\\[5cm]
	
	\begin{center}
		\makeatletter
		{\bf {\Huge Manual de usuario}}
		\\[4cm]
		\@date\\[2cm]
		{\footnotesize Revisión 1}
		\\[5cm]
		\begin{tabular}[t]{c} David Saltares Márquez \end{tabular}\\[3cm]
		\makeatother
		
		\begin{center}
			\includegraphics[scale=0.9]{imagenes/by-nc-sa.png}
		\end{center}
	\end{center}

\cleardoublepage

\tableofcontents

\cleardoublepage

\section{Bienvenido a Sion Tower}

Bienvenido a \juego, un videojuego de estrategia y acción en 3D ambientado
en un mundo fantástico compatible con sistemas GNU/Linux y Windows. A lo
largo de estas páginas podrás conocer los detalles sobre la historia y la
ambientación del juego. Más adelante encontrarás los requisitos mínimos,
de forma que puedas saber si tu equipo podrá ejecutarlo sin problemas.
A continuación se ofrece una completa guía de instalación para los dos
sistemas operativos soportados. Por supuesto, este manual contiene todos
los detalles sobre el control y las posibilidades de \juego. Finalmente
encontrarás información detallada sobre cómo crear tus propios escenarios
y añadirlos al juego.\\

\figura{logo-siontower.png}{scale=0.4}{Logo de Sion Tower}{fig:logo-siontower}{H}

\subsection{Historia}

El gremio de magos está localizado en la Torre Sagrada, \juego. Allí viven,
estudian los libros de hechizos, celebran sus ritos secretos y protegen
innumerables riquezas. Un día, el gremio al completo abandona la Torre para
celebrar un ritual en el bosque cercano. \prota, protagonista
de \juego\ es el único que permanece en el edificio sagrado para protegerlo.
A pesar de ser un joven iniciado e inexperto, no iba a suceder nada porque
el resto se ausentase durante unas horas.\\

La imprudencia del gremio resulta ser completamente desastrosa y una horda
de monstruos no tarda en rodear \juego. La misión del joven aprendiz consiste
en detener la invasión piso por piso de la Torre. ¡Debe evitar que roben
la \textbf{Reliquia Sagrada}!

\subsection{Protagonista}

\prota\ es pertenece a una de las categorías inferiores del gremio, los
iniciados. Se le reconoce fácilmente por su pequeña estatura y sus ropas
sencillas. Es un simple estudiante de magia en la Torre Sagrada y lleva
tiempo tratando de hacerse un hueco entre sus superiores a base de esfuerzo
y entrega. La invasión de la Torre es, en cierta medida, una gran oportunidad
para \prota\ de mostrar su valía. No obstante, está muerto de miedo, nunca
se había visto obligado a emplear sus poderes en una situación tan extrema.

\figura{personaje.jpg}{scale=0.3}{Protagonista de Sion Tower}{fig:protagonista}{H}

Características físicas:

\begin{itemize}
    \itemsep0em
    \item Vida: 100
    \item Maná: 100 (regeneración automática)
    \item Velocidad: $5 m/s$
    \item Hechizos: Bola de fuego, Furia de Gea y Ventisca.
\end{itemize}

\subsection{Enemigos}

Los enemigos de \juego\ son monstruos abominables de procedencia desconocida.
No obstante, su objetivo es claro: eliminar al gremio de magos y robar la
Reliquia Sagrada, una leyenda de grandioso poder. Cada enemigo tiene unas
características visuales y físicas determinadas (mostradas en la tabla inferior)
aunque su comportamiento es similar. Todos tratan de encontrar a los supervivientes
para acabar con ellos.\\

El \textbf{Goblin} es una criatura de mente muy básica, verde, enana y 
muy desagradable. Sólo va equipado con una tosca espada corta y un burdo
taparrabos. No son extremadamente duchos en combate pero sí cuentan con una
gran velocidad. Su gran ventaja es el número, son capaces de utilizar su
superioridad numérica para atosigar al enemigo y acabar con él. 

\figura{goblin.jpg}{scale=0.3}{Goblin}{fig:goblin}{H}

El \textbf{Diablillo} es una criatura venida del propio averno y bastante
superior al Goblin en lo que a propiedades físicas se refiere. Sus garras, alas
y cola de demonio les basta para atacar desgarrando la carne que se encuentra
a su paso, no necesitan ningún tipo de arma adicional.

\figura{demon.jpg}{scale=0.3}{Diablillo}{fig:demon}{H}

El \textbf{Golem de hielo} es una criatura de gran tamaño venida de las
montañas heladas del norte. Su estructura corporal está formada por piedra
maciza y bloques de hielo. Es una de las criaturas más lentas que existen
pero sus golpes son demoledores.

\figura{golem.jpg}{scale=0.3}{Golem de hielo}{fig:golem}{H}

\begin{table}[H]
  \label{caracteristicas-enemigos}
  \begin{center}
  \begin{tabular}{| c ||m{2cm}|m{2cm}|m{2cm}|}
    \hline
    Nombre del enemigo & Vida & Daño & Velocidad \\
    \hline
    Goblin & 5 & 10 & 5 \\
    \hline
    Diablillo & 7 & 15 & 4.5 \\
    \hline 
    Golem de hielo & 10 & 20 & 2 \\
    \hline
  \end{tabular}
\end{center}
\caption{Comparativa de características de enemigos}
\end{table}

\subsection{Hechizos}

Los hechizos son el único arma de \prota\ para detener el avance de los
enemigos. Sus conocimientos mágicos no le permiten hacer un enorme despliegue
de proyectiles devastadores. Debe administrar cuidadosamente su limitada
energía mágica (maná) para convocar débiles cúmulos de energía.\\

Con \textbf{Bola de fuego} puedes invocar un proyectil mágico que abrase
a tus enemigos a su paso. \textbf{Furia de Gea} convoca la propia fuerza
de la naturaleza para volverla en contra de cualquier criatura. Finalmente,
\textbf{Ventisca} lanza varios proyectiles mágicos helados y afilados como
cuchillas.\\

\begin{table}[H]
  \label{caracteristicas-hechizos}
  \begin{center}
  \begin{tabular}{| c ||m{2cm}|m{2cm}|m{2cm}|}
    \hline
    Nombre del hechizo & Maná & Daño & Velocidad \\
    \hline
    Bola de fuego & 2 & 3 & 8 \\
    \hline
    Furia de Gea & 3 & 6 & 8 \\
    \hline 
    Ventisca & 4 & 8 & 8 \\
    \hline
  \end{tabular}
\end{center}
\caption{Comparativa de atributos de hechizos}
\end{table}


\section{Requisitos mínimos}

Para poder disfrutar de \juego\ sin problemas necesitas disponer de un equipo
con unas características iguales o superiores a las que se listan a
continuación:

\begin{itemize}
    \itemsep0em
    \item \textbf{Sistema operativo}: GNU/Linux, Windows XP Service Pack
    2, Windows Vista o Windows 7.
    \item \textbf{Procesador}: Pentium 2GHz o AMD.
    \item \textbf{Memoria}: 100 MB de memoria RAM disponibles.
    \item \textbf{Tarjeta de vídeo}: 128 MB de memoria y aceleración 3D.
    \item \textbf{Espacio en disco}: 100 MB.
    \item \textbf{Control}: ratón y teclado.
\end{itemize}

En caso de utilizar un sistema operativo GNU/Linux y experimentar problemas,
revisa que tengas instalada la última versión de los drivers para tu
tarjeta gráfica y que éstos soportan aceleración por hardware.

\section{Descarga e instalación}
En esta sección se proporcionan las indicaciones pertinentes para que puedas
descargar e instalar \juego\ en tu sistema. En primer lugar, debes acudir
a la sección de ficheros del repositorio del juego en la forja de RedIRIS:\\

\url{http://forja.rediris.es/frs/?group_id=820}\\

Simplemente descarga la versión de \juego\ que coincida con tu sistema. 
No se han publicado binarios para GNU/Linux, por lo que utilizas dicho sistema,
tendrás que descargar el paquete con el código fuente y compilarlo tú mismo.
En las página \pageref{sec:instalacion-linux} encontrarás más información
al respecto.

\subsection{Instalación en GNU/Linux}
\label{sec:instalacion-linux}

Para poder jugar a \juego\ en un sistema operativo GNU/Linux es necesario
descargar e instalar el entorno de desarrollo completo ya que, por el
momento, no se distribuyen binarios del software. El videojuego cuenta
con las siguientes dependencias:

\begin{itemize}
    \itemsep0em
    \item libogre-dev
    \item libsdl1.2-dev
    \item libsdl-mixer1.2-dev
    \item mygui
\end{itemize}

A lo largo del proceso de instalación, supondremos que se posee una distribución
basada en Debian que dispone de la utilidad \texttt{apt-get}. De no ser así,
el proceso será similar utilizando el programa equivalente para instalar
paquetes en la distribución correspondiente. En primer lugar, Debemos
instalar todas las dependencias de \textsc{Ogre3D}, entre las que
se encuentran el compilador \textit{GCC}, la herramienta \textit{make}
o el generador de makefiles \textit{CMake}. Para ello utilizamos
la siguiente orden:

\begin{lstlisting}[style=consola]
sudo apt-get install g++ make libfreetype6-dev libboost-date-time-dev \
  libboost-thread-dev nvidia-cg-toolkit libfreeimage-dev zlib1g-dev \
  libzzip-dev libois-dev libcppunit-dev doxygen libxt-dev libxaw7-dev \
  libxxf86vm-dev libxrandr-dev libglu-dev cmake
\end{lstlisting}

Una vez instaladas las dependencias podemos dirigirnos a la web oficial
de \textsc{Ogre3D} y descargar el código fuente para sistemas GNU/Linux.\\

\url{http://www.ogre3d.org/download/source}\\

El siguiente paso consiste en descomprimir el paquete y acceder al directorio
resultante utilizando la terminal. La herramienta \textit{CMake} nos ayuda
a generar un fichero \texttt{makefile} dependiente de nuestra plataforma
de manera muy sencilla. Para generarlo el makefile, compilar e instalar
\textsc{Ogre3D} introducimos la siguiente secuencia de órdenes:

\begin{lstlisting}[style=consola]
cmake.
make
sudo make install
\end{lstlisting}

Instalar \textsc{libSDL} y \textsc{libSDL mixer} es muy sencillo puesto
que usualmente se encuentran en los repositorios de la distribución. Simplemente
introducimos:

\begin{lstlisting}[style=consola]
sudo apt-get install libsdl1.2-dev libsdl-mixer1.2-dev
\end{lstlisting}

Finalmente procedemos a la instalación de la biblioteca \textsc{MyGUI}.
Hemos de acudir a su sección de ficheros en la forja SourceForge y descargar
la versión \textit{3.2.0 Release Candidate 1}.\\

\url{http://sourceforge.net/projects/my-gui/files/MyGUI/}\\

Cuando finalice la descarga, descomprimimos el paquete en formato zip y
accedemos al directorio resultante utilizando la terminal. En este caso
también generaremos un makefile de forma automática con \textit{CMake}
para después compilar e instalar a través de \textit{make}.

\begin{lstlisting}[style=consola]
cmake .
make
sudo make install
\end{lstlisting}

Llegados a este punto habremos finalizado instalando las dependencias de
\juego. Ahora simplemente es necesario acceder a su directorio mediante
la terminal y comenzar con la compilación. Una vez finalizada con éxito,
podremos ejecutarlo.

\begin{lstlisting}[style=consola]
make
./siontower
\end{lstlisting}



\subsection{Instalación en Windows}

Tras descargar el paquete, utiliza cualquier programa de compresión/descompresión
que soporte el formato zip. Si no tienes ninguno instalado, puedes emplear
\textit{7Zip} sin problemas ya que es Software Libre. Para descargarlo, dirígete
a la siguiente dirección:\\

\url{http://www.7-zip.org}\\

\juego\ es completamente portable y autocontenido, así que si lo prefieres
puedes guardarlo en un lápiz de memoria o en cualquier disco duro externo.
Para iniciar el juego, haz doble click sobre el fichero ejecutable
\texttt{siontower.exe}.

\section{Guía de juego}

En este bloque el jugador novel podrá aprender cómo se juega a \juego.
Recorreremos las pantallas de la aplicación explicando la funcionalidad de
cada una de forma que aprenderemos a crear y gestionar perfiles de jugadores,
seleccionar nivel, controlar al personaje guiándolo hacia su objetivo y
progresar sumando puntos o desbloqueando nuevos escenarios.\\

\subsection{Menú principal}

Justo al iniciar \juego\ se nos presenta el menú principal del juego con
un paisaje montañoso en el que se ve a la Torre Sagrada siendo atacada
por enemigos en medio de una intensa lluvia.

\figura{siontower-menu.png}{scale=0.3}{Menú principal de Sion Tower}{fig:siontower-menu}{H}

Desde esta pantalla podemos llevar a cabo las siguientes acciones:

\begin{itemize}
    \itemsep0em
    \item \textbf{Jugar}: nos lleva a la pantalla de selección de perfil
    y al camino para comenzar una partida.
    \item \textbf{Créditos}: nos muestra una escena con las personas
    que han participado en el desarrollo.
    \item \textbf{Salir}: cierra el juego por completo, ¿seguro que quieres
    dejar de jugar?
\end{itemize}

\subsection{Selección de perfil}

\juego\ cuenta con un sistema de perfiles para que varias personas puedan
jugarlo y controlar su progreso. Al comienzo de la partida sólo está disponible
el primer nivel. Cuando finalices con éxito un escenario, el
siguiente nivel se convertirá en accesible. Además, cada perfil cuenta con
una puntuación que mide nuestras habilidades en el juego. Esta puntuación
puede subir o bajar en función de nuestra actuación en las partidas. Esto
permite que varios usuarios compitan en una máquina de forma asíncrona
para alcanzar la máxima puntuación.\\

\figura{siontower-perfil.png}{scale=0.3}{Pantalla de selección de perfil en Sion Tower}{fig:siontower-perfil}{H}

La pantalla de selección de perfil consta de dos paneles bien diferenciados.
Selección de un perfil ya existente y creación de un nuevo perfil. En el primer
bloque puedes visualizar la experiencia y los niveles desbloqueados de
cada jugador pulsando con el ratón sobre sus nombres. Si tu
perfil se encuentra en la lista y quieres utilizarlo tienes que seleccionar
el nombre y pulsar el botón \textbf{Aceptar}. Es posible eliminar
un perfil de la lista con el botón \textbf{Eliminar}. No obstante, 
hay que ser cautelosos, puesto que un perfil eliminado no es recuperable.\\

Cuando quieras crear un nuevo perfil, escribe su nombre en el panel
inferior y pulsa sobre \textbf{Crear}. Al menos debes introducir algún
carácter y el nombre elegido no debe figurar entre los perfiles existentes,
de lo contrario se mostrará un error.\\

Para volver al menú principal de \juego\ puedes pulsar sobre el botón
\textbf{Atrás}.


\subsection{Selección de nivel}
\label{sec:levelselection}

Tras seleccionar nuestro perfil se nos presentará la pantalla de selección
de nivel. En la lista de niveles aparecen todos los existentes, ya estén
bloqueados o disponibles. Los bloqueados para nuestro perfil lo indican
en rojo y su icono es translúcido.\\

\figura{siontower-nivel.png}{scale=0.3}{Pantalla de selección de nivel en Sion Tower}{fig:siontower-nivel}{H}

Para seleccionar un nivel, pulsa con el ratón sobre su icono. Si deseas
volver a la pantalla de selección de perfil, pulsa sobre \textbf{Atrás}.

\subsection{Partida}

Tras seleccionar el nivel comienza la partida y los enemigos aparecen
en pequeñas oleadas. El objetivo consiste en eliminarlos a todos utilizando
nuestra energía mágica (maná) para invocar hechizos y mantener nuestra
energía vital por encima de cero.\\

\figura{siontower-juego.png}{scale=0.3}{Partida de Sion Tower}{fig:siontower-juego}{H}

Los controles son los siguientes:

\begin{itemize}
    \itemsep0em
    \item \textbf{Movimiento}: para mover al personaje, utiliza las teclas W, A, S y D.
    \item \textbf{Cámara}: la cámara siempre hace un seguimiento al personaje
    para que nunca salga del encuadre. Puedes acercar o alejar la cámara
    utilizando la rueda del ratón. También puedes moverla alrededor del
    personaje moviendo el ratón mientras pulsas el botón derecho.
    \item \textbf{Hechizos}: para lanzar un hechizo apunta con el ratón
    en la dirección deseada y pulsa el botón izquierdo. Si tienes maná suficiente
    lanzarás el hechizo seleccionado. Para cambiar de hechizo, pulsa
    sobre el que quieras seleccionar en el menú de juego. También puedes
    hacer uso de las teclas de acceso rápido: 1, 2 y 3.
    \item \textbf{Pausa}: para activar la pausa puedes pulsar la tecla
    escape o el botón del menú de juego. En el menú de pausa puedes \textbf{Reanudar}
    la partida, volver a la pantalla de \textbf{Selección de nivel} o dirigirte
    al \textbf{Menú principal}.
\end{itemize}

Si pasas el ratón por encima de un hechizo puedes acceder a más información
sobre el mismo: nombre, descripción, maná necesario y daño que causa. Cuando
no tengas maná suficiente para lanzar un hechizo determinado, éste se mostrará
en gris. La energía mágica se recupera progresivamente de forma automática
cuando no se está haciendo uso de ella. No es posible recuperar vida a lo largo
de la partida.

\figura{siontower-pausa.png}{scale=0.3}{Menú de pausa en Sion Tower}{fig:siontower-pausa}{H}

Si nuestra energía se termina por completo habremos perdido la partida y se
nos avisará mediante un cartel. Mientras decides si volver a intentarlo
puedes seguir moviendo la cámara y observando cómo los enemigos han alcanzado
la victoria. Si pulsas la \textbf{barra espaciadora} volverás a la pantalla
de selección de nivel.

\figura{siontower-derrota.png}{scale=0.3}{Mensaje de derrota en Sion Tower}{fig:siontower-derrota}{H}

\subsection{Pantalla de victoria}

Al eliminar a todos los enemigos de un nivel habrás completado con éxito
el escenario y llegarás a la pantalla de victoria en la que el personaje
celebra su logro. Si no estabas jugando al último nivel disponible, 
habrás conseguido desbloquear el siguiente. Tanto la puntuación obtenida
como los niveles desbloqueados se guardan automáticamente en este punto.

\figura{siontower-victoria.png}{scale=0.3}{Pantalla de victoria en Sion Tower}{fig:siontower-victoria}{H}

La puntuación de \juego\ se obtiene mediante la suma de los siguientes
factores:

\begin{itemize}
    \itemsep0em
    \item \textbf{Enemigos}: por cada enemigo eliminado se obtienen puntos.
    Cuanto más poderoso sea el enemigo, mayor será la recompensa en forma de puntos.
    \item \textbf{Vida}: recibimos un complemento por la energía que nos quede
    al finalizar la partida con éxito.
    \item \textbf{Maná}: en el juego se valora mucho la puntería y la administración
    del maná o energía mágica. Cuanto menos maná utilicemos, mayor será la puntuación.
    Es posible que si se malgasta demasiado, la puntuación sea negativa.
    \item \textbf{Tiempo}: recibimos recompensa en forma de punto por
    eliminar a los enemigos lo antes posible, este apartado también puede
    recibir una valoración negativa.
\end{itemize}

Tras terminar con éxito un nivel puedes elegir la opción \textbf{Volver a jugar},
volver a la pantalla de \textbf{Selección de nivel} o regresar al \textbf{Menú principal}.

\subsection{Créditos}

En la pantalla de créditos aparece una escena de la Torre Sagrada con
un panel listando los participantes en el desarrollo de \juego. Cuando
desees volver puedes pulsar el botón \textbf{Atrás}.

\figura{siontower-creditos.png}{scale=0.3}{Autores de Sion Tower}{fig:siontower-creditos}{H}


\section{Creación de niveles}

¿Tienes una gran idea para un nuevo nivel de \juego? ¿Sabes como puedes
mejorar sustancialmente la historia? ¿Te gustaría que todos viesen tus ideas?
El motor del videojuego permite a los usuarios crear y añadir nuevos niveles
de manera muy sencilla utilizando la herramienta de modelado y animación
3D libre por excelencia: \textit{Blender}. El proceso es largo pero sencillo,
sobre todo si sigues detenidamente todos los pasos de este manual.

\subsection{Instalación de Blender y complementos}

\textit{Blender} es la herramienta de modelado y animación 3D de carácter
completamente libre más potente que existe. Utilizando \textit{Blender}
se han producido en su totalidad cortos de animación como
Big Buck Bunny\footnote{Big Buck Bunny: \url{http://bigbuckbunny.org}},
Sintel\footnote{Sintel: \url{http://sintel.org}} o videojuegos como
Yo Frankie!\footnote{Yo Frankie!: \url{http://yofrankie.org}}. \textit{Blender} es
multiplataforma, no importa si posees Windows, Mac o GNU/Linux ya que no
tendrás problema alguno para utilizarlo. Recientemente se ha publicado la
serie 2.5 de la herramienta, que incluye mejoras como un cambio radical de
la interfaz. No obstante, los scripts de exportación necesarios funcionan
en la 2.49, por tanto, esa sera la versión que utilizaremos.\\

\figura{sintel.jpg}{scale=0.3}{Fotograma del cortometraje Sintel}{fig:sintel}{H}

Si utilizas un sistema operativo GNU/Linux basado en Debian, es probable
que \textit{Blender} se encuentre en los repositorios de tu distribución.
En tal caso, bastaría con introducir en una terminal:

\begin{lstlisting}[style=consola]
sudo apt-get install blender
\end{lstlisting}

En caso de contar con un sistema operativo de tipo Windows, debes acudir
a la sección de versiones anteriores de \textit{Blender} y descargar el
instalador:\\

\url{http://download.blender.org/release/Blender2.49a/blender-2.49a-windows.exe}\\

Simplemente abre el instalador haciendo doble click sobre el ejecutable,
sigue los pasos indicados y habrás instalado Blender con éxito.\\

En ambos sistemas operativos es necesario instalar un plugin de exportación
que convierte una escena creada con Blender a un fichero de texto en formato
xml procesable por el motor de \juego. El formato se conoce como \textit{Dotscene}
y es ampliamente utilizado. Así mismo, es necesario otro complemento que
convierte un modelo creado con \textit{Blender} a un xml. Es posible
descargar ambos plugins desde:\\

\url{http://ogreaddons.svn.sourceforge.net/viewvc/ogreaddons/trunk/blender\
sceneexporter/ogredotscene.py}\\

\url{http://ogre.svn.sourceforge.net/viewvc/ogre/branches/v1-6/Tools/BlenderExport}\\

Para instalar los exportadores en sistemas GNU/Linux, hemos de copiar los ficheros
descargados en:\\

\texttt{~/.blender/scripts}\\

En cambio, si utilizamos Windows el directorio será:\\

\texttt{[Instalación de Blender]/.blender/scripts}\\


\subsection{Composición del nivel}

Para seguir esta guía con soltura es recomendable que poseas unos conocimientos
mínimos sobre \textit{Blender}. Deberías saber cómo manejar los elementos
básicos de la interfaz y la forma de manipular los objetos dentro de la
escena: movimiento, rotación, escala, duplicado, etc. Si nunca has utilizado
\textit{Blender} con anterioridad es recomendable que acudas a algún
manual para principiantes como el libro gratuito bajo Creative Commons
\textit{Aprende Blender en 24 horas}\footnote{Aprende Blender en 24 horas: 
\url{http://www.inf-cr.uclm.es/www/cglez/downloads/blender/24hBlenderYafray.pdf}}
del profesor certificado por la Blender Foundation Carlos González Morcillo.

% Nuevo documento
\subsubsection{Crear un nuevo documento}

El primer paso es crear un nuevo documento de \textit{Blender}, para ello
simplemente abrimos la herramienta por primera vez o hacemos click sobre
File $\rightarrow$ New y aceptamos eliminar la escena actual. Para comenzar
no podemos tener objetos presentes por lo que seleccionamos toda la escena
con A y eliminamos con X.

\figura{blender-1.jpg}{scale=0.35}{Creación de un nuevo documento en Blender}{fig:blender-1}{H}

Es recomendable contar con un espacio de trabajo cómodo, la disposición
de las vistas tridimensionales de \textit{Blender} es fundamental. Para
crear un nuevo nivel lo ideal es contar con las vistas frontal, lateral,
cenital y libre. Puedes ver un ejemplo de esta disposición en la figura
\ref{fig:blender-1}.

% Convenciones de nombrado
\subsubsection{Convenciones de nombrado}

En un escenario de \juego\ se dan cita varios tipos de elementos: objetos
del escenario, geometría arbitraría, efectos de partículas, oleadas de enemigos,
etc. \textit{Blender} no distingue entre ellos, para la herramienta de modelado
todos son objetos tridimensionales comunes. Hemos de seguir unas reglas de
nombrado especiales para que el motor sepa distinguir entre una mesa que
forma parte del mobiliario y un icono simbólico que representa la llegada
de un enemigo.\\

Durante todo el proceso de creación del escenario es importante prestar
atención para que los objetos tengan un nombre acorde con las reglas
establecidas. En la pestaña de edición (tecla F9) aparece el panel
\textit{Link and Materials} mostrando el nombre de la malla del objeto
y el del propio objeto. En nuestro caso, objetos semejantes comparten
la misma malla mientras que el nombre de cada objeto debe ser distinto
y ajustarse a las reglas. Puedes verlo en la figura \ref{fig:blender-3}.

\figura{blender-3.jpg}{scale=0.7}{Nombre de una puerta en un escenario}{fig:blender-3}{H}

Las reglas de nombrado son las siguientes:

\begin{itemize}
    \itemsep0em
    \item \textbf{Escenario}: todos los elementos del escenario (paredes,
    suelo, mesas, sillas y otros muebles) siguen la regla \texttt{scene-nombre.numero}.
    El nombre del objeto es lo que lo identifica dentro del catálogo para poder
    recuperar su modelo de colisión y el número ayuda a hacerlo único dentro de la
    escena (dos objetos no pueden tener el mismo nombre). Ejemplo: \texttt{scene-door.005}.
    \item \textbf{Efectos de partículas}: puedes incluir efectos de partículas
    en cualquier punto de la escena. La regla es \texttt{particle-nombre.numero}.
    Ejemplo: \texttt{particle-flame.001}.
    \item \textbf{Luces}: el motor del juego reconoce las luces de manera automática,
    por lo que no debes preocuparte de sus nombres.
    \item \textbf{Malla de navegación}: la malla de navegación indica a los
    enemigos cuales son las zonas transitables del escenario y siempre
    debe llamarse \texttt{navMesh}. Veremos más sobre las mallas de navegación
    en secciones posteriores.
    \item \textbf{Enemigos}: los enemigos siguen la regla \texttt{enemy-tipo-t.numero}.
    La letra t representa el segundo en el que aparecerá el enemigo desde
    que se inicia la partida. Ejemplo: \texttt{enemy-goblin-25.001}.
    \item \textbf{Protagonista}: el elemento cuyo nombre sea \texttt{player}
    definirá la posición inicial del jugador dentro del nivel.
    \item \textbf{Geometría arbitraria}: toda la geometría que no siga
    la convención de nombrado será tratada como complementos del escenario.
    No se calcularán colisiones contra ellos (se podrán atravesar).
\end{itemize}

% Enlazar objetos
\subsubsection{Enlazar objetos}

Los escenarios del juego están repletos de objetos de distintas clases
y puedes reutilizarlos sin ningún tipo de problemas para tus propias creaciones.
Incluir los objetos ya creados para \juego\ en tu nuevo nivel es muy sencillo.
En el menú de \textit{Blender} seleccionamos File $\rightarrow$ Append or
Link (Image Browser). Se abrirá un pequeño navegador de ficheros para
que seleccionemos el \texttt{objeto.blend} a importar. Al abrirlo, tendremos
que seleccionar Object y el nombre del objeto creado en \textit{Blender}.

\figura{blender-2.jpg}{scale=0.35}{Enlazar con un objeto de Blender existente}{fig:blender-2}{H}

Al enlazarlo se añadirá al origen de coordenadas el objeto deseado. No obstante,
los datos de su malla pertenecen a un fichero externo (contorno azul) y lo
deseable es convertirlo en local. De esta manera nuestro proyecto de nivel
no contará con indeseables dependencias externas. Para hacerlo
seleccionamos el objeto con el botón derecho del ratón, pulsamos L y
seleccionamos la opción All. Una vez localizado, podremos mover el objeto,
escalarlo y rotarlo por la escena con total libertad.

\figura{blender-4.jpg}{scale=0.4}{Convertir en local un objeto enlazado}{fig:blender-4}{H}

% Colocando objetos
\subsubsection{Elementos del escenario}

Los elementos del escenario son aquellos objetos que cuentan con un modelo
de colisión. Esto quiere decir que ni el protagonista ni los hechizos podrán
atravesarlos. Los objetos que venían incluidos en \juego\ ya tienen un modelo
de colisión asignado y para utilizarlos sólo hemos de enlazarlos siguiendo
los pasos de la sección anterior. Estos objetos se encuentran en
\texttt{[siontower]/media/blender} y la lista completa junto a sus códigos
es la siguiente:

\begin{itemize}
    \itemsep0em
    \item Cama (bed)
    \item Candelabro (candle)
    \item Columna (column)
    \item Muro con arco (door)
    \item Muro con puerta (door2)
    \item Muro (wall)
    \item Suelo $8x8 m$ (floor8x8)
    \item Suelo $4x4 m$ (floor4x4)
    \item Reliquia (reliq)
    \item Silla (chair)
    \item Taburete (roundchair)
    \item Estanterías (shelves)
    \item Escaleras (staircase)
    \item Mesa (table)
    \item Caja de madera (woodenbox)
\end{itemize}

Cuando enlacemos un objeto y lo convirtamos en local podemos moverlo utilizando
la tecla G, rotarlo con R o escalarlo mediante S. Las tres operaciones anteriores
pueden restringirse a uno de los tres ejes. Por ejemplo, para desplazar
una mesa a lo largo del eje X será necesario seleccionar la mesa con el botón
derecho del ratón, pulsar G, pulsar X y desplazarla. Es necesario que
no haya desniveles y que el nivel del suelo esté situado en el plano $Z = 0$
de \textit{Blender}\\

Basta con enlazar al principio cada objeto y duplicarlo cada vez que deseemos
uno semejante. En \textit{Blender} es posible duplicar objetos seleccionando
el elemento a duplicar y pulsando SHIFT $+$ D. No obstante, dicha operación
conlleva tener redundancias en los datos de la malla y lo que deseamos es
contar con dos instancias de la misma malla. Para duplicar elementos
utilizaremos la combinación ALT $+$ D. Con estas sencillas instrucciones
ya posees los conocimientos suficientes para componer escenas como la de
la figura \ref{fig:blender-5}.

\figura{blender-5.jpg}{scale=0.35}{Elementos del escenario ya colocados}{fig:blender-5}{H}

% Nuevos objetos
\subsubsection{Nuevos objetos}

Puedes añadir nuevos objetos que no estén en el catálogo modelados por tí
mismo o que hayas encontrado en algún banco de recursos libres. Una vez
modelado y texturizado, será necesario que lo exportes en un formato compatible
con el motor gráfico del juego \textsc{Ogre3D}, el formato binario \texttt{.mesh}.
Puedes obtener más información sobre la exportación en \wiki:\\

\url{http://wikis.uca.es/iberogre/index.php/Exportar_modelos_desde_Blender}\\

Los modelos exportados se sitúan en el directorio \texttt{[siontower]/media/meshes},
los materiales en \texttt{[siontower]/media/materials} y las texturas
en \texttt{[siontower]/media/textures}. El último paso consiste en definir
el modelo colisionable del nuevo elemento del escenario. Debes acudir al
fichero xml \texttt{[siontower]/media/levels/bodycatalog.xml} y editarlo
de forma manual.\\

Cada objeto cuenta con un nombre único y  tiene asociado un cuerpo (body)
compuesto por una o más formas (shapes). Los objetos tienen un tipo, pueden
ser 1 (suelo) o 2 (mobiliario). Cada forma es una figura geométrica
sencilla:

\begin{itemize}
    \itemsep0em
    \item \textbf{Esfera}: \textit{sphere}, definida por un centro y un
    radio.
    \item \textbf{Plano}: \textit{plane}, definido por un punto y un vector
    perpendicular.
    \item \textbf{Caja alineada}: \textit{aabb}, un hexaedro
    alineado con los ejes, definido por un centro y unas distancias hasta
    sus lados. 
    \item \textbf{Caja orientada}: \textit{obb}, hexaedro al que
    se le ha aplicado una rotación, definido por un centro, distancias
    hasta los lados y los ejes sobre los que se apoya.
\end{itemize}

Ha de añadirse una nueva entrada por cada objeto nuevo que se añada si se
desea que cuente con un modelo colisionable. La sintaxis del fichero
xml es bastante sencilla.

\begin{lstlisting}[style=xml]
<?xml version="1.0" encoding="UTF-8" ?>
<bodies>
    
    <body name="floor8x8" type="1">
        <shape type="plane">
            <position x="0" y="0" z="0" />
            <normal x="0" y="1" z="0" />
        </shape>
    </body>
    
    <body name="chair" type="2">
        <shape type="obb">
            <center x="0" y="-0.036" z="0.147"/>
            <extent x="0.2275" y="0.4675" z="0.21"/>
            <axes a00="1" a01="0" a02="0" a10="0" a11="1"
                  a12="0" a20="0" a21="0" a22="1" />
        </shape>
    </body>
    
</bodies>
\end{lstlisting}


% Efectos de partículas
\subsubsection{Efectos de partículas}

Como ya se ha mencionado anteriormente, los efectos de partículas siguen
la nomenclatura \texttt{particle -nombre.numero}. Lo ideal es representarlos
mediante pequeñas esferas de un color relacionado con el efecto que generan.
En \juego\ se incluyen tres tipos de sistemas de partículas por defecto:

\begin{itemize}
    \itemsep0em
    \item \textbf{flame}: llama del tamaño adecuado para ser utilizada en las antorchas.
    \item \textbf{flame2}: llama del tamaño adecuado para colocarse sobre los candelabros.
    \item \textbf{rain}: lluvia intensa utilizada en el menú principal del juego.
\end{itemize}

\figura{blender-6.jpg}{scale=0.35}{Esfera simbolizando un sistema de partículas}{fig:blender-6}{H}

Los sistemas de partículas se definen en scripts de extensión \texttt{.particle}
almacenados en el directorio \texttt{[siontower]/media/particles}. Si conoces
la sintaxis, puedes añadir los que desees para utilizarlos en tus propios niveles.
Hay herramientas como \textit{Ogre Particle Editor}\footnote{Ogre Particle Editor: \url{http://www.ogre3d.org/tikiwiki/OGRE+Particle+Editor&structure=Tools}}
que permite generar dichos scripts utilizando una interfaz gráfica sencilla.

% Luz
\subsubsection{Iluminación}

Añadir luces a una escena es imprescindible, de lo contrario esta aparecería
completamente oscura dentro del juego. Para hacerlo pulsa la barra
espaciadora dentro de la escena 3D y selecciona Add $\rightarrow$ Lamp
y uno de los tipos de luces aceptados por \juego: Lamp, Sun y Spot.
No tienes porqué preocuparte por ninguna convención de nombrado en este caso.
Una vez hayas añadido la luz, puedes moverla por la escena como desees.\\

Es posible configurar varios parámetros de cada punto de luz como su intensidad
y color accediendo al panel de sombreado (F5) y al subpanel de controles
de luz. Lo recomendable es que ninguna luz ascienda de $1.0$ en intensidad
y que los colores sean suaves.

\figura{blender-7.jpg}{scale=0.4}{Panel de configuración de luces}{fig:blender-7}{H}


% Malla de navegación
\subsubsection{Malla de navegación}

La malla de navegación es un conjunto de triángulos conectados que definen
un área transitable para los enemigos. En \juego\ los monstruos no podrían
encontrar el camino hacia el personaje si no fuera por la malla de navegación.
Es similar a contar con un tablero repleto de casillas conectadas. Una vez
tengamos claro dónde se colocarán los obstáculos podemos ir creando la malla.
Recuerda que el objeto debe llamarse \texttt{navMesh} (se distingue entre
mayúsculas y minúsculas). El nombre de la propia malla debe ser único
entre las mallas del juego, lo normal es llamarla \texttt{navMesh-número}
donde número es la posición del nivel en la historia.

\figura{blender-8.jpg}{scale=0.35}{Creación de la malla de navegación}{fig:blender-8}{H}

Para comenzar a crear la malla puedes añadir un plano pequeño con la barra
espaciadora, Add $\rightarrow$ Mesh $\rightarrow$ Plane. Entra en modo edición
utilizando el tabulador y selecciona uno de los vértices con el botón derecho
para eliminarlo con X. Asegúrate de que la malla siempre permanece en el suelo,
es importante para que los enemigos no se eleven misteriosamente. Debe estar
formada exclusivamente por triángulos, no pueden existir cuadriláteros.\\

Extender la malla es sencillo, selecciona un vértice y pulsa E para extruirlo
creando uno nuevo y una arista de unión. Ve repitiendo la operación para
cubrir las zonas transitables del escenario y recuerda dejar un margen
prudencial con los obstáculos. Debes crear caras entre los grupos de tres
vértices que formen los triángulos, para hacerlo selecciona los tres vértices
y pulsa F. Al finalizar es imprescindible que todas las caras de la malla
miren hacia el mismo lado (arriba o abajo), para conseguirlo, selecciona
todos los vértices con A y pulsa Ctrl $+$ N.\\

Finalmente, debes exportar la malla para que el motor de \juego\ pueda
procesarla. Selecciona la malla fuera del modo edición (pulsa el tabulador)
y accede a File $\rightarrow$ Export $\rightarrow$ OGRE Meshes.

\figura{blender-9.jpg}{scale=0.25}{Exportación de la malla de navegación}{fig:blender-9}{H}

El nombre del material es irrelevante porque no nos hará falta, asegúrate
de marcar la opción \textit{Fix Up Axis to Y} y elegir el directorio
\texttt{[siontower]/media} como destino de la exportación.

% Enemigos
\subsection{Enemigos y jugador}

Los enemigos suelen representarse mediante una pirámide de base cuadrada
en posición horizontal. Como ya hemos mencionado, la nomenclatura para los
enemigos es \texttt{enemy-tipo-t.numero} siendo los tipos posibles: \textit{goblin},
\textit{demon} y \textit{golem}. Para que el diseño de niveles sea más cómodo, los goblins
llevan asignado el color verde, los demonios el rojo y los gólems el azul
aunque no es obligatorio de cara a la exportación.\\

Tras añadir un enemigo, puedes duplicarlo utilizando la combinación SHIFT
$+$ D, moverlo con G y rotarlo con R (aunque lo normal es hacerlo sobre
el eje vertical únicamente). La pirámide de los enemigos debe colocarse
atravesando el suelo, de forma que el vértice superior esté en el punto 0
del eje Z.

\figura{blender-10.jpg}{scale=0.35}{Disposición de enemigos en el nivel}{fig:blender-10}{H}

Debes añadir otro elemento simbólico como una esfera o cono para indicar
la posición inicial del jugador. Simplemente has de nombrarlo \texttt{player}.


\subsection{Integración del nivel}

Es conveniente que poco a poco vayas guardando tus progresos diseñando el
nivel ya que no es extraño equivocarse o dejar el trabajo a media. Para ello
selecciona File $\rightarrow$ Save, elige una ruta y presiona el botón de
guardado. A continuación se adjuntan las indicaciones para integrar el nivel
que acabas de crear en \juego.

% Exportación del nivel
\subsubsection{Script de exportación}

Tras haber guardado el nivel como fichero \texttt{.blend} es el momento
de proceder a la exportación a formato \textit{DotScene}. Para ello selecciona
File $\rightarrow$ Export $\rightarrow$ Ogre Scene. Se desplegará un panel
con todos los elementos de la escena, si quieres que todos aparezcan
en el nivel, pulsa sobre All. Recuerda activar la opción \textit{Fix Up Axis
To Y} y selecciona el directorio de niveles como destino de la exportación:
\texttt{[siontower]/media/levels}.

\figura{blender-11.jpg}{scale=0.25}{Exportación de la escena}{fig:blender-11}{H}

Por cada nivel existen dos ficheros xml, el primero es el de la escena
y tiene como nombre \texttt{codigo\_scene.xml}. Por su parte, el segundo
contiene información complementaria como el nombre del nivel, descripción
y música que sonará, su nomenclatura es \texttt{codigo\_info.xml}. Por ejemplo,
si el código del nivel que deseamos exportar es \texttt{level05}, el nombre
del fichero escena debería ser \texttt{level05\_scene.xml}.

% Añadir el nivel al juego
\subsubsection{Añadir el nivel al juego}

% level-info
Para integrar el nuevo nivel dentro de \juego\ debes crear un fichero
\texttt{codigo\_info.xml} en el directorio \texttt{[siontower]/media/levels}.
La sintaxis es muy sencilla, a continuación se muestra el ejemplo para
el primer nivel del juego. El fichero de música debe estar en formato \textsc{Ogg}
y estar situado en \texttt{[siontower]/media/music}.

\begin{lstlisting}[style=xml]
<?xml version="1.0" encoding="UTF-8" ?>
<basicInfo>
    <name>The Hall</name>
    <description>Some enemies have assaulted the main hall of\nthe Sacred Tower. Stop the invasion!</description>
    <song name="nivel1.ogg" group="" />
</basicInfo>
\end{lstlisting}

% Icono
Como ya has visto en la página \pageref{sec:levelselection}, en la pantalla
de selección de nivel, cada escenario viene acompañado de un icono. Si no
haces nada, el hueco quedará vacío para tu nuevo nivel. No obstante, si lo deseas,
puedes añadir un fichero con nombre \texttt{codigo.png} al directorio
\texttt{[siontower]/media/textures}.

% Fichero levels
Finalmente, debes añadir tu nuevo nivel al fichero que indexa los niveles,
\texttt{[siontower]/media/levels/levels.xml}. Por ejemplo, si añadiésemos
el quinto nivel de nuestro ejemplo anterior, el fichero quedaría de la
siguiente manera.

\begin{lstlisting}[style=xml]
<?xml version="1.0" encoding="UTF-8" ?>
<levels>
    <level id="level01"></level>
    <level id="level02"></level>
    <level id="level03"></level>
    <level id="level04"></level>
    <level id="level05"></level>
</levels>
\end{lstlisting}

Con esto habrás conseguido añadir un nivel completamente nuevo y personalizado
a \juego. Para comprobar que todo está correcto, inicia el juego y trata
de seleccionar el escenario. Antes de compartirlo con tus amigos, es recomendable
que compruebes que ni es demasiado complicado ni demasiado sencillo, ¡de lo
contrario no sería divertido!

\end{document}
