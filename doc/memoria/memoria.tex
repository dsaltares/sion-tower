% -*-memoria.tex-*-
% Este fichero es parte de la plantilla LaTeX para
% la realización de Proyectos Final de Carrera, protejido
% bajo los términos de la licencia GFDL.
% Para más información, la licencia completa viene incluida en el
% fichero fdl-1.3.tex

% Copyright (C) 2009 Pablo Recio Quijano 

%-------------------------------------------------------
% ---- Plantilla para libros / memorias PFC -----

% Realizada por Pablo Recio Quijano y Noelia Sales Montes 
% Formato de portada y primera página tomado del PFC de
% Francisco Javier Vázquez Púa, en su proyecto 'libgann'
% -------------------------------------------------------

\documentclass[a4paper,11pt]{book}

\usepackage{./estilos/estiloBase} % Basicamente son todas las
                                  % librerias usadas. En caso de que
                                  % falten librerias se van añadiendo
                                  % al fichero.
\usepackage{./estilos/colores}  % Algunos colores ya generados, para
                                % los algunos estilos más avanzados.
\usepackage{./estilos/comandos} % Algunos comandos personalizados

\graphicspath{{./imagenes/}} % Indicamos la ruta donde se encuentran
                             % las imagenes, para ahorrarnos la ruta
                             % completa, y solo modificar aquí si en
                             % un momento dado lo movemos

\begin{document}

% Renombramos las figuras y las tablas
\renewcommand{\figurename}{Figura}
\renewcommand{\listfigurename}{Indice de figuras}
\renewcommand{\tablename}{Tabla}
\renewcommand{\listtablename}{Indice de tablas}

\pagestyle{empty}
% Fichero con la portada %%%%%%

\thispagestyle{empty}
%\begin{picture}(0,0)
%	\put(-73,-35){\includegraphics[scale=0.42]{\plpath/img/cabecera.png}}
%\end{picture}\\[4cm]
	
	\begin{center}
		\makeatletter
		{\bf {\Huge \@title}}
		\\[2cm]
		{\bf {\huge \subtitlename}}\\[4cm]
		\@date\\[2cm]
		{\footnotesize Revisión \revname}
		\\[7cm]
		\begin{tabular}[t]{c} \@author \end{tabular}\\[3cm]
		\makeatother
		
		\begin{center}
			\includegraphics[scale=0.9]{\plpath/img/by-nc-sa.png}
		\end{center}
	\end{center}

\cleardoublepage



\cleardoublepage

% -*-primerahoja.tex-*-
% Este fichero es parte de la plantilla LaTeX para
% la realización de Proyectos Final de Carrera, protejido
% bajo los términos de la licencia GFDL.
% Para más información, la licencia completa viene incluida en el
% fichero fdl-1.3.tex

% Fuente tomada del PFC 'libgann' de Javier Vázquez Púa

\begin{center}

  \includegraphics[scale=0.2]{logo_uca.jpg} \\

  \vspace{2.0cm}

  \Large{ESCUELA SUPERIOR DE INGENIERÍA} \\

  \vspace{1.0cm}

  \large{INGENIERO TÉCNICO EN INFORMÁTICA DE GESTIÓN} \\

  \vspace{2.0cm}

  \large{IBEROGRE, WIKI DE OGRE3D EN ESPAÑOL\\Y SION TOWER, VIDEOJUEGO DE ESTRATEGIA} \\

  \vspace{1.0cm}

\end{center}

\begin{itemize}
\item \large{Departamento: Lenguajes y sistemas informáticos}
\item \large{Directores del proyecto: Manuel Palomo Duarte y Francisco Palomo Lozano}
\item \large{Autor del proyecto: David Saltares Márquez}
\end{itemize}

\vspace{1.0cm}

\begin{flushright}
  \large{Cádiz, \today} \\

  \vspace{2.5cm}

  \large{Fdo: David Saltares Márquez}
\end{flushright}

\cleardoublepage
\pagestyle{plain}

\frontmatter % Introducción, índices ...

% -*-previo.tex-*-
% Este fichero es parte de la plantilla LaTeX para
% la realización de Proyectos Final de Carrera, protejido
% bajo los términos de la licencia GFDL.
% Para más información, la licencia completa viene incluida en el
% fichero fdl-1.3.tex

% Copyright (C) 2009 Pablo Recio Quijano 

\section*{Agradecimientos}

Me gustaría agradecer y dedicar este texto a:

\begin{itemize}
    \itemsep0em
    \item Manuel Palomo y Francisco Palomo por sus consejos y labor de tutorización.
    \item Mis compañeros Javier, Jose, Daniel, José Tomás y Alberto por su ayuda
    en los momentos de mayor presión y sus empujoncitos para que se encienda
    la bombilla.
    \item Daniel Pellicer, Antonio Caro, Francisco Martín, Antonio
    Rodríguez y Daniel Belohlavek por el gran trabajo artístico que han
    realizado.
    \item Ainhoa por apoyarme tanto en las épocas de éxito como de
    frustración así como por su labor de testeo.
    \item Toda mi familia por el cariño incondicional en todo momento.
    \item La fantástica comunidad de usuarios que han ofrecido su colaboración
    y sugerencias.
\end{itemize}

\cleardoublepage

\section*{Licencia} % Por ejemplo GFDL, aunque puede ser cualquiera

Este documento ha sido liberado bajo Licencia GFDL 1.3 (GNU Free
Documentation License). Se incluyen los términos de la licencia en
inglés al final del mismo. No obstante, las imágenes de videojuegos
comerciales están sujetas a copyright y no se distribuyen bajo licencia
libre.\\

Copyright (c) 2011 David Saltares Márquez.\\

Permission is granted to copy, distribute and/or modify this document under the
terms of the GNU Free Documentation License, Version 1.3 or any later version
published by the Free Software Foundation; with no Invariant Sections, no
Front-Cover Texts, and no Back-Cover Texts. A copy of the license is included in
the section entitled "GNU Free Documentation License".\\

\cleardoublepage

\section*{Notación y formato}

Con el objetivo de mantener un estilo uniforme y cómodamente legible, a lo
largo de esta memoria de \textbf{Proyecto Fin de Carrera} se ha utilizado la
siguiente notación:

\begin{itemize}
    \item Para referirnos a nombres de ficheros, órdenes del sistema o
    funciones de un lenguaje utilizaremos: \texttt{ogre.cfg}.
    \item Cuando se nombre a una tecnología o biblioteca se hará uso del
    formato: \textsc{Ogre3D}.
    \item Los nombres de las clases se escriben en cursiva: \textit{Game}.
    \item Cuando se mencione el nombre de un programa se hará con el
    siguiente formato: \textit{Blender}.
    \item En caso de adjuntar un fragmento de código se utilizarán bloques
    como el siguiente:
    \lstinputlisting[style=C++]{codigo/holamundo.cpp}
\end{itemize}

\cleardoublepage

\tableofcontents
\listoffigures
\listoftables

\mainmatter % Contenido en si ...

\chapter{Introducción}
\label{chap:introduccion}
\paragraph{}
Este es el documento de diseño de \juego. El videojuego para PC que 
ejemplifica todos los contenidos de \wiki, la wiki en español sobre desarrollo de videojuegos
en 3D utilizando Ogre como motor de renderizado. Este escrito tiene como
objetivo principal plasmar los elementos que debe incluir \juego y servir
de carta de presentación en caso de buscar colaboradores en un futuro.

%\paragraph{}
%Se darán detalles sobre trasfondo, objetivos, jugabilidad, mecánicas etc.
%Finalmente se elaborará una lista de recursos artísticos necesarios: modelos,
%texturas, efectos de partículas, sonido, animaciones...

\subsection{Concepto del juego}
\label{sec:int-concepto}

\paragraph{}
\juego es un videojuego en el que controlamos a un pequeño mago iniciado
que permanece a cargo de una torre sagrada mientras sus compañeros y 
maestros han acudido a un celebrar un rito. Durante su guardia, la torre
se ve asaltada por criaturas malignas y nuestro pequeño mago debe detener
la invasión a toda costa utilizando sus limitadas habilidades.

\subsection{Características principales}
\label{sec:int-caracteristicas}

%\paragraph{}
%Hemos realizado ciertas pinceladas sobre el juego pero a continuación
%mencionaremos las características principales de \juego. En secciones
%posteriores se profundizará con detalle en todos los puntos:

\paragraph{}
El juego se basa en los siguientes pilares:

\begin{itemize}
    \item \textbf{Planteamiento sencillo}: la historia mencionada es muy
    simple, una mera excusa para el desarrollo del juego pero lo suficientemente
    explícita para que el \jugador sienta que tiene un objetivo.
    \item \textbf{Táctica}: detener a las oleadas de enemigos debe ser
    imposible si se comienza a atacar indiscriminadamente. La gestión de
    nuestras limitadas capacidades de forma inteligente será imprescindible.
    \item \textbf{Dinamismo}: al contrario que algunos juegos de estrategia,
    \juego debe ser dinámico y provocar una sensación de tensión en el
    \jugador.
    \item \textbf{Ampliación}: \juego debe ser ampliable con nuevos niveles
    y enemigos de forma sencilla. El motor será todo lo independiente posible
    del contenido. De esta forma los artistas podrán generar nuevos niveles,
    habilidades o enemigos.
\end{itemize}

\subsection{Género}
\label{sec:int-genero}

\paragraph{}
\juego supone una unión de varios géneros. A continuación se listan los géneros
de los que toma elementos y sus motivos:

\begin{itemize}
	\item \textbf{Tower Defense}: un subgénero de la estrategia basado
	en detener oleadas de enemigos colocando de forma estratégica obstáculos y trampas.
        En \juego manejamos los mismos elementos aunque aparece la figura
	de un protagonista y no es una \emph{`mano invisible'} la que gestiona la acción.
	
	\item \textbf{Acción en tercera persona}: juegos dinámicos y directos
	en el que el \jugador experimenta una descarga de adrenalina.
	La cámara suele situarse cerca del personaje. En \juego tenemos el
	componente de la acción aunque la cámara se situará de forma que podamos
	ver gran parte de la escena.
\end{itemize}

\subsection{Propósito y público objetivo}
\label{sec:int-publico}

\paragraph{}
El principal objetivo de \juego es ofrecer a los lectores de \wiki un ejemplo
real de videojuego en 3D desarrollado utilizando el engine de renderizado
Ogre. Es un complemento dentro del contenido didáctico del wiki. No obstante,
debe ser un producto jugable y divertido. Su interés no sólo debe radicar
en el código fuente y su proceso de desarrollo sino en el propio juego
y sus mecánicas.

\paragraph{}
\juego está dirigido a jugadores de un amplio rango de edades con un tiempo
limitado que dedicar al ocio electrónico. Por ello, se apuesta por un sistema
de partidas cortas y recompensas rápidas. La historia es sencilla, lo que
permite poder jugar de forma esporádica.

\subsection{Jugabilidad}
\label{sec:int-jugabilidad}

\paragraph{}
Cada nivel de \juego ofrece un piso de la torre sagrada que está siendo
asaltada por enemigos. Tenemos que impedir que las bestias lleguen a un
determinado punto del nivel. Para ello nos valdremos de los siguientes elementos:

\begin{itemize}
    \item \textbf{Movilidad}: al contrario que en otros Tower Defense, en
    \juego controlamos un personaje. Nos desplazaremos por el escenario
    atendiendo los focos de peligro que consideremos oportunos.
    \item \textbf{Trampas y obstáculos}: podemos dirigir al personaje a
    un punto del nivel y colocar una trampa u obstáculo si tenemos
    suficientes recursos. Esto eliminará o ralentizará a los enemigos.
    \item \textbf{Hechizos}: el combate directo utilizando hechizos también
    es una opción.
    \item \textbf{Puntos fuertes y débiles}: cada enemigo tendrá puntos
    fuertes y débiles. La mejor forma de hacerles frente es utilizar la
    herramienta adecuada en cada momento.
    \item \textbf{Mejoras}: el \jugador deberá recibir recompensas cada
    poco tiempo para que sienta que progresa en el juego. Tendremos habilidades
    desbloqueables.
\end{itemize}

\subsection{Estilo visual}
\label{sec:int-estilo}

\paragraph{}
\juego tendrá un estilo sencillo, no demasiado detallista para encajar con
su carácter amigable y accesible. El estilo visual que más encaja con este
concepto es el de dibujo animado o cómic. Los personajes serán caricaturescos,
los colores serán vivos y las texturas muy simples. Se estudiará aplicar un
efecto \emph{`cel-shading'} para reforzar esta idea.

\subsection{Alcance}
\label{sec:int-alcance}

\paragraph{}
El objetivo principal es desarrollar un sistema de juego sólido al que podamos
introducirle contenidos sin dificultad. En primera instancia se desarrollará
un pack de contenidos básicos que será ampliado en un futuro.



\chapter{Organización temporal}
\label{chap:calendario}
En este capítulo se expone la planificación de tareas que se ha seguido
para desarrollar \wiki\ y \juego. En primer lugar se adjunta el diagrama
de Gantt completo para, más adelante, complementarlo con un breve comentario de
cada tarea.

\section{Diagrama de Gantt}

Como puede observarse en el diagrama de Gantt, en tareas relacionadas con
el apartado artístico de \juego\ participan más personas. Esto se debe a que
no poseía ni poseo los conocimientos o destrezas requeridos para crear
modelos tridimensionales animados, piezas musicales ni grabar y procesar
efectos de sonido. Por ello, he contactado con expertos en dichas materias
dispuestos a colaborar en el desarrollo de un videojuego completamente libre.
Finalmente, los participantes adicionales son:

\begin{itemize}
    \itemsep0em
    \item Antonio Jiménez Rodríguez: artista 3D encargado de diseñar, modelar,
    texturizar al protagonista y a los enemigos del juego.
    \item Daniel Belohlavek: artista 2D encargado de los iconos de los hechizos.
    \item Daniel Pellicer García y Antonio Caro Oca: encargados de componer
    y producir la banda sonora completa del juego.
    \item Francisco Martín Márquez: técnico de sonido encargado de grabar
    y procesar los efectos de sonido para el juego.
\end{itemize}

Por cuestiones de espacio y formato, el diagrama de Gantt ha sido divido
en dos. En la figura \ref{fig:planificacion1} puede observarse la planificación
del proyecto correspondiente al intervalo comprendido entre julio de 2010
y enero de 2011. Por otro lado, la figura \ref{fig:planificacion2} muestra
el resto de la planificación, desde febrero de 2011 hasta septiembre de 2011.

\figura{planificacion1.jpg}{scale=0.35}{Planificación del proyecto desde julio de 2010 hasta enero de 2011}{fig:planificacion1}{H}


\figura{planificacion2.jpg}{scale=0.35}{Planificación del proyecto desde febrero de 2011 hasta septiembre de 2011}{fig:planificacion2}{H}


\section{Etapas de desarrollo del proyecto}

\begin{enumerate}
    \itemsep0em
    \item \textbf{Planificación}
    
    Dada la envergadura del proyecto, era necesaria
    una etapa de planificación en la que se ha estudiado de forma cuidadosa
    el alcance del mismo y las posibles dificultades a encontrar durante
    el desarrollo.\\
    
    \item \textbf{Formación}
    
    Al comienzo del proyecto desconocía por completo
    el uso de bibliotecas imprescindibles como \textsc{Ogre3D}, \textsc{OIS} o
    \textsc{MyGUI} y herramientas importantes como \textit{Blender}. Tampoco
    conocía con la necesaria profundidad los fundamentos del desarrollo
    de videojuegos en tres dimensiones. Fue necesaria, por tanto,
    una larga etapa de formación personal utilizando varios recursos bibliográficos
    y pequeñas pruebas prácticas.\\
    
    \item \textbf{IberOgre}
    
    Tras el periodo de formación inicial, dio comienzo el montaje de la wiki
    \wiki. La Oficina de Software Libre de la Universidad de Cádiz (OSLUCA)
    llevó a cabo la instalación de la plataforma sobre sus servidores y,
    desde dicho momento, se pudo comenzar a trabajar en la documentación
    de \textsc{Ogre3D}.
    
    \begin{enumerate}
        \itemsep0em
        \item \textit{Plantillas}
        
        Fue necesario transformar la apariencia que incluye una plataforma
        \textit{Wikimedia} por defecto para conseguir un aspecto propio
        en \wiki. Para ello se modificaron las hojas de estilo \textsc{CSS},
        se trabajó sobre la portada y se crearon varias plantillas. Estas
        plantillas proporcionan una forma sencilla de incluir estructuras
        en varios puntos del sitio como ejemplos, tablas especiales, etc.\\
        
        \item \textit{Sección: Primeros pasos}
        
        Tras preparar la wiki, se procedió a la redacción del artículo
        correspondiente a la primera sección de \wiki. En dicho artículo
        se daba una bienvenida a la documentación y se explicaban sus
        motivaciones, objetivos y estructura en cuanto a lecciones. De esta
        manera, el lector podría conocer la filosofía de trabajo de la plataforma.\\
        
        \item \textit{Sección: \textsc{Ogre3D}}
        
        Posteriormente se comenzó a elaborar los artículos pertenecientes
        a la sección del motor \textsc{Ogre3D} junto a sus ejemplos prácticos. 
        Estos textos cubren prácticamente la totalidad del uso del motor
        en cuanto a aspectos básicos se refiere: instalación, arquitectura,
        inicialización, configuración, creación de escenas, animación, iluminación,
        materiales, efectos de partículas, etc.\\
        
        \item \textit{Sección: Otras tecnologías}
        
        \textsc{Ogre3D} no es un motor de videojuegos completo, sólo es un
        motor de renderizado. Por tanto, el uso de tecnologías complementarias
        es prácticamente obligado. En \wiki\ se han elaborado artículos
        para asistir a los lectores en dichas tecnologías como \textsc{OIS}
        para gestión de dispositivos de entrada o \textsc{SDL mixer} para
        música y efectos de sonido.\\
        
        \item \textit{Sección: Matemáticas}
        
        En la sección de matemáticas se presentan conceptos fundamentales
        para el desarrollo e videojuegos. Sobre todo se cubre de manera
        ligera temas relacionados con la geometría del espacio: puntos,
        vectores, matrices y cuaternos. Siempre se ofrecen aplicaciones
        prácticas dentro del ámbito que nos ocupa.\\
        
    \end{enumerate}
    \item \textbf{Sion Tower}
    
    Tras el periodo de aprendizaje, casi de manera simultánea al trabajo
    en \wiki, se comenzó con el videojuego \juego. El primer paso fue solicitar
    la creación de un nuevo proyecto en la \textit{Forja de RedIRIS} y 
    esperar su aprobación. Dicha forja proporciona un repositorio
    \textit{Subversion} y herramientas web que ayudan en la gestión de
    un proyecto libre: gestor de tareas, subida de ficheros, publicación
    de noticias, etc.
    
    \begin{enumerate}
        \itemsep0em
        \item \textit{Documento de diseño}
        
        En el documento de diseño de un videojuego se detallan elementos
        como la historia, género, personajes, mecánicas de juego, objetos
        o enemigos entre otros muchos. En definitiva, es el documento
        que define de forma más o menos concisa cómo será el videojuego.
        Se trata de un escrito muy importante ya que ayuda a que todo el
        equipo albergue la misma idea sobre el videojuego y pueda trabajar
        de forma más compenetrada.\\
        
        \item \textit{Análisis}
        
        La fase de análisis dio comienzo tras la redacción y revisión del
        documento de diseño. Se procedió con la toma de requisitos a partir
        de dicho documento, se confeccionaron los casos de uso, se elaboró
        el modelo conceptual de datos y se detalló el modelo de comportamiento.\\
        
        \item \textit{Diseño}
        
        Durante la fase de diseño, que siguió de forma inmediata al análisis,
        se elaboraron los diagramasde clases de diseño.\\
        
        \item \textit{Implementación}
        
        La fase de implementación de \juego\ fue, con diferencia, la más
        extendida de todo el desarrollo. Quizás viniese motivada por
        la inexperiencia y por el gran número de subsistemas que se implementaron
        desde cero.\\
        
        \item \textit{Pruebas}
        
        Durante la implementación se fueron realizando pruebas de módulos
        individuales pero fue tras finalizar ficha fase cuando tuvieron
        lugar las pruebas de integración. No sólo se trabajó para que
        el código fuese correcto sino que \juego\ fue probado de forma
        extensiva por colaboradores distintos al desarrollador principal con
        el claro objetivo de pulir ciertos detalles relacionados con el balanceo
        de características (personajes muy débiles o poderosos) o la comodidad
        del control entre otros.\\
        
        \item \textit{Mantenimiento}
        
        Tras reunir sugerencias del público interesado en el proyecto se
        realizaron pequeñas mejoras y ajustes que en pocas ocasiones implicaron
        cambios en el código fuente. Sobre todo, estaban relacionados
        con el balanceo de personajes.\\
        
        \item \textit{Distribución}
        
        Se crearon y publicaron paquetes descargables tanto para sistemas GNU/Linux
        como para Windows. La dificultad que conllevaba empaquetar una
        aplicación \textsc{Ogre3D} en Debian provocó que se tuviera
        que distribuir un paquete con el código fuente en GNU/Linux. Para
        Windows existe un paquete con el ejecutable \texttt{.exe}, los ficheros
        multimedia y las dependencias en forma de bibliotecas dinámicas.\\
        
        \item \textit{Arte}
        
        El proceso de creación de todo el arte necesario para \juego\ comenzó
        nada más acabar la redacción del documento de diseño. Desde ese momento
        se conocía el estilo visual del juego, las pantallas y los personajes
        que aparecerían. Por su complejidad, el trabajo se extendió prácticamente
        durante todo el desarrollo. Al ser la única tarea en la que 
        participaron colaboradores, fueron necesarias labores de supervisión
        y coordinación: estilo, formato de entrega, corrección de fallos, etc.\\
        
    \end{enumerate}
    \item \textbf{Memoria}
    
    Tras el desarrollo de \wiki\ y \juego\ se procedió a la redacción
    del presente documento que incluye apéndices adicionales como el 
    manual de usuario del videojuego.\\
    
    \item \textbf{Presentación}
    
    Como última tarea en la organización temporal figura la elaboración
    de la exposición de cara a la presentación del \textbf{Proyecto fin
    de Carrera}. Se llevó a cabo tratando de plasmar el trabajo realizado
    y los objetivos conseguidos con el desarrollo de este proyecto.\\
    
    \item \textbf{Comunidad}
    
    Una de las partes fundamentales de este proyecto es su objetivo de servir
    a la comunidad. Hasta el momento no existía una documentación extensa
    sobre \textsc{Ogre3D} y el desarrollo de videojuegos tridimensionales
    en castellano. A lo largo de todo el desarrollo ha estado presente la
    atención e interacción con la comunidad a través de diversos medios
    como redes sociales, la propia wiki y el blog.\\
    
    \item \textbf{Presentación (hito)}
    
    Finalmente, el día 1 de septiembre de 2011 se entrega toda la documentación
    del proyecto y se preparó la presentación final que tendría lugar,
    aproximadamente, una semana después.\\
    
\end{enumerate}


\chapter{Desarrollo de IberOgre}
\label{chap:desarrollo-iberogre}
\section{Metodología}

\wiki\ es un proyecto de documentación y, por tanto, no hemos podido
aplicar una metodología de desarrollo típica de sistemas software como el
RUP (\textit{Rational Unified Process}) empleado en \juego\ más adelante.
Podemos distinguir dos partes bien diferenciadas en este proyecto aunque
inseparables: el texto de los artículos y los ejemplos que los acompañan.\\

% Etapas para los artículos
Para la composición de todos los artículos se ha seguido un proceso
uniforme para asegurar la calidad de los mismos:\\

\begin{enumerate}
    \itemsep0em
    \item \textbf{Estudio de necesidades}: búsqueda de los puntos más
    relevantes a la hora de aprender a desarrollar videojuegos en tres dimensiones
    así como los apartados más fundamentales de la biblioteca \textsc{Ogre3D}.
    \item \textbf{Planificación}: estudio para encontrar la mejorar forma
    de abordar la materia escogida. División del contenido en una
    estructura uniforme con el resto de la plataforma de aprendizaje. 
    \item \textbf{Redacción y formateado}: recopilación de información y
    redacción del artículo completo. Adaptación del texto al formato
    de la plataforma, composición del artículo con imágenes que amenicen
    la lectura y publicación.
    \item \textbf{Desarrollo del ejemplo}: creación de un ejemplo descargable
    que ilustre todo el contenido expuesto en el texto del artículo de forma
    práctica. Creación de documentación adicional en el propio artículo
    para explicar el funcionamiento del ejemplo.
    \item \textbf{Evaluación}: tras la publicación del artículo se buscan
    las impresiones de los lectores y se anotan todas las sugerencias para
    mejorar el texto.
\end{enumerate}

\section{Análisis}
\label{iberogre-analisis}

En esta sección expondremos el análisis de la documentación sobre desarrollo
de videojuegos en tres dimensiones \wiki. En primer lugar realizaremos
una breve especificación de requisitos del sistema, posteriormente
hablaremos de los tipos de usuarios que accederán a la plataforma y finalmente
haremos un listado de los artículos necesarios.\\

\subsection{Especificación de requisitos del sistema}

\subsubsection{Requisitos de interfaces externas}

Los lectores accederán al contenido de \wiki\ mediante su monitor, empleando
el navegador de su elección. Por tanto, es importante que la plataforma emplee
estándares web soportados por todos los navegadores mayoritarios: Internet
Explorer, Mozilla Firefox, Google Chrome, Opera y Safari. Para conseguirlo,
será necesario escoger un sistema de publicación compatible con HTML 4.0
y CSS 2. En la medida de lo posible, deben seguirse las directrices del
\textit{World Wide Web Consortium} (W3C) \cite{website:w3c}.\\

El contenido deberá poder ser visualizado de forma clara en monitores
de varias resoluciones y relaciones de aspecto. El texto debe aparecer
de forma limpia, clara y poco recargada. No obstante, debe estar acompañado
de imágenes que amenicen la lectura de los temas más áridos. En caso de
que se adjunte código fuente, éste deberá poder visualizarse en la misma
plataforma empleando un resaltado de sintaxis apropiado.\\

La navegación debe ser intuitiva y no precisar de aclaraciones de ningún tipo.
El contenido debe estar perfectamente organizado en secciones temáticas
y ordenadas en dificultad creciente, de forma prácticamente secuencial.\\

\subsection{Requisitos sobre usuarios}

No todos los usuarios que accedan a \wiki\ tendrán los mismos privilegios.
Existirán tres categorías de usuarios cuyos privilegios se detallan a
continuación.\\

\begin{itemize}
    \itemsep0em
    \item \textbf{Lector}: usuario ocasional no registrado en la plataforma.
    \begin{itemize}
        \itemsep0em
        \item Acceso de lectura a los artículos publicados y a las secciones de la wiki.
        \item Posibilidad de enviar sugerencias de mejora.
    \end{itemize}
    \item \textbf{Usuario registrado}: 
    \begin{itemize}
        \itemsep0em
        \item Acceso de lectura a los artículos publicados y a las secciones de la wiki.
        \item Posibilidad de enviar sugerencias de mejora.
        \item Acceso de escritura a artículos existentes con el objetivo de
        ampliarlos, corregirlos y/o mejorarlos.
        \item Creación de nuevos artículos siguiendo la estructura general
        de la wiki.
    \end{itemize}
    \item \textbf{Administrador}: 
    \begin{itemize}
        \itemsep0em
        \item Acceso de lectura a los artículos publicados y a las secciones de la wiki.
        \item Posibilidad de enviar sugerencias de mejora.
        \item Acceso de escritura a artículos existentes con el objetivo de
        ampliarlos, corregirlos y/o mejorarlos.
        \item Creación de nuevos artículos siguiendo la estructura general
        de la wiki.
        \item Control sobre la estructura de la plataforma.
        \item Instalación de nuevas extensiones para mejorar la plataforma.
        \item Gestión del resto de usuarios.
    \end{itemize}
\end{itemize}

Cualquier persona que acceda al sitio web de la plataforma será considerado
como lector no registrado. En el momento en el que rellene el formulario
y envíe sus datos se convertirá en usuario registrado. Lo habitual es que
sólo exista un administrador del sistema o exista un grupo reducido de
administradores.\\

\subsection{Artículos}

% Bloques temáticos
Los artículos de \wiki\ están organizados en varios bloques temáticos
para abarcar todos los objetivos que nos hemos marcado en la introducción
de esta memoria. Esta organización también busca facilitar la navegación
al usuario de forma que acceda al contenido que desee en el menor tiempo
posible. Los bloques principales de \wiki\ son:\\

\begin{itemize}
    \itemsep0em
    \item \textbf{Primeros pasos}: bloque introductorio en el que el lector
    debe poder conocer la estructura, filosofía y metodología de la plataforma
    de aprendizaje. Tras leerlo, el usuario debe poder saber qué esperar
    de \wiki\ y cómo afrontar los contenidos posteriores.
    \item \textbf{Programación de videojuegos 3D}: sección que se dedicará
    a ofrecer los conocimientos matemáticos mínimos para desarrollar
    videojuegos en 3D. Se centrará en conceptos generales de geometría
    del espacio y álgebra lineal desde una perspectiva eminentemente práctica.
    Las matemáticas son un medio, una herramienta más para conseguir los objetivos
    propuestos. Al no ser el fin en sí mismo, priman las aplicaciones prácticas
    por encima de la rigurosidad (sin llegar a ser incorrectos).
    \item \textbf{Ogre3D}: bloque central que desglosa los apartados más
    relevantes del motor de renderizado \textsc{Ogre3D}. Tras terminar con
    este bloque, el lector debe ser capaz de utilizar la biblioteca
    para crear aplicaciones tridimensionales con elementos como modelos
    animados, sistemas de partículas y otros efectos de iluminación. Lo más
    importante es que se comprenda el funcionamiento del motor y se sea
    capaz de ampliar conocimientos de forma autónoma.
    \item \textbf{Otras tecnologías}: como ya se ha mencionado anteriormente,
    \textsc{Ogre3D} es sólo un motor de renderizado, no proporciona todos
    los elementos necesarios para construir un videojuego como gestión 
    de entrada del usuario, sonido, detección de colisiones o juego en red.
    Por ello, esta sección está enfocada a introducir en el uso de
    tecnologías complementarias a \textsc{Ogre3D} como \textsc{OIS} (entrada),
    \textsc{libSDL mixer} (audio) u \textsc{OgreBullet} (físicas).
    \item \textbf{Videojuegos}: en esta sección se adjuntará documentación
    sobre el desarrollo de videojuegos libres que hagan uso de \textsc{Ogre3D}.
    \juego\ contará con el primer artículo de esta sección. El videojuego pretende
    ser una aplicación práctica lo más realista posible del uso del motor y de la creación
    de videojuegos 3D en general. Adjuntaremos toda la documentación
    que se ha generado al respecto para que el lector pueda acceder a la misma
    de forma sencilla. Sin duda, es el bloque que se presta en mayor medida
    a ampliaciones futuras por parte de la comunidad.
\end{itemize}

\figura{bloques-iberogre.png}{scale=0.65}{Bloques temáticos de IberOgre}{fig:bloques-iberogre}{h}

% Lista de artículos

La sección \textit{Primeros pasos} únicamente cuenta con un artículo
introductorio.

\begin{itemize}
    \itemsep0em
    \item \textbf{Comenzando en IberOgre}: en este artículo a modo de preámbulo
    desgranamos todos los objetivos de la plataforma de aprendizaje. Posteriormente
    se explica la filosofía de trabajo que deseamos tomar con software compatible
    con varias plataformas y de carácter abierto.
\end{itemize}

En el bloque \textit{Programación de videojuegos 3D} se encuentran tres
artículos.

\begin{itemize}
    \itemsep0em
    \item \textbf{Introducción, puntos y vectores}: en el primer artículo
    sobre programación de videojuegos en 3D se hace una introducción al
    planteamiento que vamos a seguir durante todo el bloque. Posteriormente
    tratamos los puntos y los sistemas de coordenadas. Más adelante abarcamos
    los vectores, las operaciones que podemos realizar con ellos y sus aplicaciones
    en la materia. Finalmente se dedica una pequeña sección a hablar sobre
    interpolaciones lineales para suavizar movimientos.
    \item \textbf{Matrices}: en este artículo definimos el concepto de matriz
    desde el punto de vista matemático para después hacer un repaso por sus
    operaciones y finalmente tratar su utilidad a la hora de representar
    transformaciones en tres dimensiones (traslaciones, rotaciones
    y escalas).
    \item \textbf{Cuaternos}: los cuaternos son una extensión de los números
    reales que se suelen utilizar a la hora de representar rotaciones en un
    espacio tridimensional. En el artículo explicamos su definición, la forma
    de encadenar rotaciones y deshacerlas así como la operación necesaria
    para realizar interpolaciones y rotar un elemento de forma progresiva.
\end{itemize}

En el bloque principal, \textit{Ogre3D} tenemos los siguientes artículos.

\begin{itemize}
    \itemsep0em
    \item \textbf{Conociendo Ogre3D}: en el primer artículo relacionado
    con el motor de renderizado debe explicarse qué es capaz de hacer
    por nosotros \textsc{Ogre3D} y cuáles son sus limitaciones. Se expone
    una lista de características que incorpora así como los requisitos
    de hardware que exige para poder utilizarse. Comentamos su licencia
    permisiva y los lenguajes con los que es compatible (aunque nosotros
    sólo utilizaremos C++). Por último, hablamos brevemente de varias alternativas
    libres a \textsc{Ogre3D} como \textsc{Panda 3D} o \textsc{Irrlicht}.
    \item \textbf{Conceptos generales}: en este texto hacemos un recorrido
    por la arquitectura del motor, sus subsistemas más relevantes y la filosofía
    que se ha tomado a la hora de diseñarlos. Explicamos de forma rápida
    su objeto raíz, el gestor de recursos y el grafo de escena.
    \item \textbf{Instalación de Ogre3D 1.7 en GNU/Linux}: el lector
    debe poder acceder a una sencilla, rápida y directa guía de cómo instalar
    el entorno de desarrollo en su equipo para poder continuar con las lecciones
    de la plataforma de aprendizaje. En este artículo se expone el proceso
    de instalación para sistemas GNU/Linux. Se incluye un apartado
    especial para la popular distribución Ubuntu, ya que muchas de las dependencias
    están en los repositorios oficiales o en repositorios \textit{PPA}
    \cite{website:ppa}.
    \item \textbf{Instalación de Ogre3D 1.7 en Windows}: los usuarios de
    Windows también deben contar con una guía de instalación del entorno
    de desarrollo completo para poder seguir los contenidos de \wiki. En este
    caso es necesario instalar el compilador \textit{MinGW} y las bibliotecas
    \textsc{Ogre3D}, \textsc{Boost} y \textsc{DirectX}. Se precisa adjuntar un proyecto
    de prueba para comprobar la corrección del proceso.
    \item \textbf{Creación de un entorno de trabajo multiplataforma}: es
    altamente deseable que los usuarios de GNU/Linux y Windows trabajen
    de la misma forma siguiendo el contenido de la wiki. Para ello, los proyectos
    debían tener la misma estructura, el mismo código compatible y 
    un proceso de compilación similar. En este artículo se explica la
    jerarquía de directorios que tomaremos y los makefiles para compilar
    ejemplos y proyectos personales.
    \item \textbf{Inicialización y cierre de Ogre3D}: en este artículo
    creamos por primera vez una aplicación básica con \textsc{Ogre3D}. Se
    repasa la secuencia de inicialización y cierre del motor. Así mismo,
    conocemos las posibilidades de extensión del motor gracias al uso
    de plugins y complementos externos. Posteriormente aprendemos a configurar
    aspectos como la resolución, la frecuencia, la sincronización vertical
    utilizando ficheros de configuración. Finalmente se explica el sistema
    de logs de \textsc{Ogre3D} para depurar nuestras aplicaciones.
    \item \textbf{Gestión de recursos}: el motor incluye un sistema de gestión
    de recursos para optimizar el consumo de memoria evitando duplicidades
    y proporcionar un método uniforme de acceder a elementos como mallas
    tridimensionales, materiales o esqueletos. Los usuarios del motor
    deben conocer como gestionar el ciclo de vida de los recursos para poder
    cargar modelos entre otras muchas cosas.
    \item \textbf{Creación básica de escenas}: este artículo retoma la inicialización
    de \textsc{Ogre3D} y continua con el objetivo de ilustrar la creación
    de escenas. Debe mostrar como crear una ventana, un gestor de escenas,
    una cámara y un punto de vista. Una vez esté preparado el grafo de la escena
    para que se le añadan elementos, se hace un recorrido por la creación
    y gestión de nodos de escena con modelos tridimensionales estáticos.
    Se hace una breve introducción a la iluminación y se detallan las distintas
    aproximaciones para abordar el bucle de renderizado (bucle de juego).
    \item \textbf{Materiales}: nuestra biblioteca cuenta con un sistema
    de materiales que, básicamente, definen cómo un objeto interacciona
    con la luz que recibe. En \textsc{Ogre3D} existe una pequeña sintaxis
    para definir materiales. En este artículo hablamos de los detalles
    de los scripts para definir materiales y sus posibles parámetros así
    como de las maneras para cargar dichos materiales y aplicarlos a entidades.
    \item \textbf{Manipulación de nodos}: las escenas de \textsc{Ogre3D}
    están formadas por nodos que contienen mallas tridimensionales, 
    puntos de luz, cámaras u otros elementos. Para dotar de dinamismo
    a las escenas estos nodos deben trasladarse, ser rotados o escalados.
    En definitiva, deben ser dotados de movimiento. En este artículo
    hacemos un repaso por cómo se representan los vectores y transformaciones
    en el motor y comenzaremos a gestionar los nodos de la escena. Se
    exponen las diferencias entre los distintos espacios de transformación
    (local, parental y global) y aprendemos a aplicar transformaciones
    a los nodos (traslación, rotación y escalado). Finalmente se trata el tema
    de la interpolación entre puntos para producir movimiento suavizado.
    \item \textbf{Luces, sombras y entorno}: todos los videojuegos tridimensionales
    utilizan efectos de iluminación para crear la ambientación deseada.
    Era imprescindible incluir un artículo al respecto en \wiki. En la primera
    sección se habla sobre las distintas técnicas de sombreado disponibles
    en el motor. Posteriormente abarcamos el tema de la iluminación
    desde conceptos básicos como la reflexión difusa y especular hasta
    la creación de puntos que emitan luz en \textsc{Ogre3D}. El siguiente
    apartado corresponde al efecto de niebla, muy utilizado para ahorrar
    recursos o para crear una atmósfera terrorífica. Por último, se habla
    de los fondos en dos dimensiones para representar elementos como el cielo.
    \item \textbf{Animación}: la animación es un tema muy complejo pero imprescindible
    en todo videojuego. El artículo comienza con los conceptos básicos sobre
    la animación como el uso de fotogramas clave e interpolaciones entre los mismos. 
    Posteriormente explicamos los tres tipos de animación que existen en
    \textsc{Ogre3D} y cual es en la que nos centraremos durante el resto
    del texto. A continuación explicamos la técnica para cargar animaciones
    en el motor, reproducirlas y manipularlas. Por último, se hace una introducción
    a la mezcla de animaciones (\textit{animation blending}).
    \item \textbf{Sistemas de partículas}: en este artículo se abordan
    los sistemas de partículas de \textsc{Ogre3D} empleados para reproducir
    efectos como fuego, explosiones, humos, estelas mágicas, etc. Estos
    efectos se definen en scripts similares a los de los materiales y explicamos
    su estructura y sintaxis. Tras hacerlo comentamos el método para crear
    y destruir sistemas de partículas. Finalmente mencionamos un potente
    editor que nos ahorra tener que modificar de forma manual estos efectos.
    \item \textbf{Sistema de Overlays}: \textsc{Ogre3D} carece de sistema
    para crear elementos avanzados de interfaz como botones, etiquetas o
    formularios completos. En cambio, posee un sistema de overlays para mostrar
    elementos bidimensionales en una escena tridimensional. Estos paneles
    se definen también empleando scripts cuya estructura, sintaxis y propiedades
    detallamos a lo largo del artículo. Una vez terminadas las propiedades
    mostramos cómo cargar y gestionar overlays dentro del juego. Este sencillo
    sistema puede ser útil para usuarios que no requieran interfaces demasiado
    complejas.
\end{itemize}

Dentro de la sección \textit{Otras tecnologías} tenemos los siguientes
artículos.

\begin{itemize}
    \itemsep0em
    \item \textbf{Manejo básico de OIS}: es una biblioteca libre de gestión
    de dispositivos de entrada como joystick, ratón o teclado. Es la recomendada
    por \textsc{Ogre3D}, de hecho se distribuyen juntas. En este artículo
    aprendemos a consultar el estado de los dispositivos de entrada así como
    a responder a los eventos que estos producen en el momento adecuado empleando
    el patrón Observer \cite{gamm77}.
    \item \textbf{Exportar modelos desde Blender}: cuando creamos un modelo
    tridimensional posiblemente animado en una herramienta como \textit{Blender}
    debemos exportarlo al formato que emplea \textsc{Ogre3D}. En este artículo
    se detalla el proceso con todas las opciones que nos proporcionan los
    scripts de exportación.
    \item \textbf{Colisiones y físicas con OgreBullet}: como ya se ha mencionado,
    \textsc{Ogre3D} no proporciona un sistema de detección de colisiones ni
    simulaciones físicas. Por tanto, este artículo está destinado a documentar
    la instalación y el uso de la biblioteca \textsc{OgreBullet}, un envoltorio
    de la popular \textsc{Bullet}.
    \item \textbf{Extender la gestión de recursos, audio}: el sistema de gestión
    de recursos de \textsc{Ogre3D} permite ser extendido con sencillez para
    añadir nuevos tipos de recursos. Aprovechando esta capacidad explicamos
    el proceso incluyendo un sistema de audio utilizando la biblioteca
    \textsc{libSDL} en conjunción con \textsc{libSDL mixer}.
\end{itemize}



\section{Diseño}
\label{iberogre-diseno}

Una vez completado el análisis con los requisitos de \wiki\ nos centraremos
en el diseño de sus componentes. En primer lugar decidiremos la estructura
interna de cada uno de los artículos que compondrán la plataforma educativa
de desarrollo de videojuegos 3D. Posteriormente pasaremos al diseño de la
navegabilidad del sitio y de su apariencia visual. Finalmente especificaremos
los ejemplos con los que contarán los artículos.\\

\subsection{Estructura de los artículos}

La estructura lógica de los artículos queda reflejada en la figura \ref{fig:bloques-articulos}
y está pensada para que el lector adquiera de la manera más sencilla posible
los conocimientos necesarios para continuar. Dicha estructura ha ido siendo
refinada a lo largo del desarrollo gracias al continuo flujo de opciones
por parte de los lectores. La estructura consta de las siguientes partes.\\

\begin{itemize}
    \itemsep0em
    \item \textbf{Introducción}: breve resumen del campo que abarcará
    el artículo para que el lector sepa qué esperar cuando lo siga. Siempre
    se pretende ofrecer una perspectiva práctica dando ejemplos concretos
    de aplicaciones reales del contenido.
    \item \textbf{Requisitos previos}: breve sección con los artículos
    que el lector debería conocer antes de continuar. La estructura de 
    \wiki\ es aproximadamente secuencial pero en muchas ocasiones existen
    dependencias adicionales. Es importante que el lector sepa qué conocimientos
    debe poseer antes de enfrentarse a un texto sin entender nada.
    \item \textbf{Desarrollo}: bloque principal del artículo, que a su vez
    puede estar compuesto de secciones, en el que se desarrolla el tema a tratar.
    Se intercalan las explicaciones teóricas con pequeños fragmentos de
    código mostrando de forma práctica las técnicas comentadas. 
    \item \textbf{Ejemplo final}: en todos los artículos se adjunta un ejemplo
    final en forma de aplicación descargable que ilustra el contenido
    expuesto en el desarrollo. Al fichero descargable lo acompaña una explicación
    sobre cómo se ha implementado el ejemplo y se hacen sugerencias para
    que el lector modifique elementos. Este aspecto es muy relevante ya que
    la práctica es mucho más efectiva en el proceso de aprendizaje.
    \item \textbf{Conclusiones}: finalmente se incluye una sección que hace
    las veces de resumen de los contenidos tratados y lista las tareas
    que el lector debe ser capaz de realizar tras comprender el contenido
    del artículo.
\end{itemize}

\figura{bloques-articulos.png}{scale=0.65}{Estructura de los artículos en IberOgre}{fig:bloques-articulos}{h}

\subsection{Navegabilidad}

Para \wiki\ se ha elegido el motor \textit{MediaWiki} \cite{website:wikimedia} por
ser uno de los más empleados en este tipo de plataformas, contar con funcionalidades
más atractivas de personalización o edición y disponer de una comunidad
más numerosa dispuesta a ayudar en caso de problemas. Su diseño visual por
defecto ha sido modificado de forma que se adapte en la mayor medida posible
a la estructura de \wiki\ y a su enfoque.\\

En esta sección haremos un recorrido por el diseño de \wiki\ comenzando
por los bloques de su página principal. La pantalla de bienvenida prácticamente
al completo puede observarse en la figura \ref{fig:iberogre-diseno}, en
ella se aprecia claramente la división por bloques.\\

\figura{layout-iberogre.jpg}{scale=0.25}{Diseño de IberOgre}{fig:iberogre-diseno}{H}

\subsubsection{Bloque de bienvenida}

En el bloque de bienvenida se muestra el logo de \wiki\ y se hace una pequeña
introducción en pocas líneas sobre el tema que trata la plataforma. Es sencilla
y directa para captar el interés del lector potencial. Puede verse con mayor
detalle en la figura \ref{fig:iberogre-bienvenida}.\\

\figura{iberogre-bienvenido.jpg}{scale=0.25}{Bloque de bienvenida a IberOgre}{fig:iberogre-bienvenida}{h}

\subsubsection{Navegación}

El bloque de navegación se sitúa a la izquierda de la página y es útil para
acceder de forma directa a distintas secciones de interés para los usuarios
de la plataforma. Puede verse en la figura \ref{fig:iberogre-navegacion}
y se compone de los siguientes enlaces:

\begin{itemize}
    \itemsep0em
    \item \textbf{Página principal}: accede a la bienvenida de la wiki.
    \item \textbf{Portal de la comunidad}: página con información de interés
    para los lectores sobre el trabajo actual en \wiki.
    \item \textbf{Actualidad}: sección de noticias.
    \item \textbf{Cambios recientes}: permite ver cuáles son las páginas nuevas
    o las últimas ediciones, en qué han consistido y quién las ha realizado.
    \item \textbf{Página aleatoria}: nos lleva a cualquiera de los artículos
    de la wiki de forma aleatoria.
    \item \textbf{Ayuda}: breves consejos para aquellos interesados en colaborar
    con la plataforma.
\end{itemize}

\figura{iberogre-navegacion.jpg}{scale=0.5}{Panel de navegación en IberOgre}{fig:iberogre-navegacion}{h}

\subsubsection{Búsqueda}

El cuadro de búsqueda nos permite encontrar artículos que hablen sobre
las palabras claves introducidas. En primer lugar trata de buscar un artículo
cuyo nombre coincida exactamente con la clave introducida, posteriormente
busca coincidencias parciales en el título y, por último, dentro del texto.
Muestra los resultados por orden de relevancia de forma que el lector tenga
más probabilidades de encontrar lo que busca en la zona alta de la página
de resultados. El cuadro es sencillo y puede verse en la figura \ref{fig:iberogre-busqueda}.\\

\figura{iberogre-busqueda.jpg}{scale=0.55}{Cuadro de búsqueda en IberOgre}{fig:iberogre-busqueda}{h}

\subsubsection{Bloque de contenido}

El bloque más importante de \wiki\ es el de contenido, en él se listan todos
los artículos organizados por categorías y en orden creciente de dificultad.
A pesar de que las dependencias no son exactamente lineales en todos los casos,
es buena idea que el lector pueda empezar por el primero y continuar hacia
abajo por la lista. En la figura \ref{fig:iberogre-contenido} puede observarse
con más detalle este bloque aunque no se muestra en su totalidad por razones
de espacio.\\

\figura{iberogre-contenido.jpg}{scale=0.35}{Contenido de IberOgre}{fig:iberogre-contenido}{h}

\subsubsection{Bloque otros}

El bloque \textit{Otros} se divide en dos secciones diferenciadas. La primera
está relacionada con la difusión del proyecto y la comunicación con los usuarios
(\textit{IberOgre en otros medios}) mientras que la segunda contiene enlaces
a manuales de referencia con información para aquellos que deseen colaborar
o profundizar.

\begin{itemize}
    \itemsep0em
    \item \textbf{IberOgre en otros medios}: 
        \begin{itemize}
            \itemsep0em
            \item \textit{Blog oficial}: blog dedicado al desarrollo de \wiki\
            y \juego. Contiene tanto noticias sobre los avances como artículos
            documentando el desarrollo de subsistemas concretos.
            \item \textit{Forja}: enlace al repositorio de la forja de RedIRIS
            que aloja al proyecto. Contiene noticias, lista de tareas, lista
            de correo, ficheros descargables y el repositorio de código
            Subversion \cite{website:svn}.
            \item \textit{Twitter}: enlace a la cuenta oficial del proyecto
            en la conocida red social de microblogging. Como puede leerse
            en el capítulo \nameref{chap:comunidad}, se ha hecho un uso extensivo
            de esta herramienta durante todo el desarrollo. De esta manera,
            se establece un contacto más directo con los lectores.
            \item \textit{E-mail}: medio para contactar con la wiki y enviar
            sugerencias, críticas y otras opiniones para cambiar elementos
            e introducir mejoras.
        \end{itemize}
    \item \textbf{Ayuda}: 
    \begin{itemize}
            \itemsep0em
            \item \textit{Manual oficial de Ogre3D en español}: traducción completa
            del manual oficial de referencia de \textsc{Ogre3D} en español.
            El fichero \textit{PDF} ha sido ofrecido por el colaborador
            Mario Velázquez Muñoz. Es un manual de consulta que complementa
            perfectamente al carácter didáctico de los artículos.
            \item \textit{Edición en MediaWiki}: guía sencilla sobre edición
            en sistemas \textit{MediaWiki} creada por Noelia Sales Montes
            y Emilio José Rodríguez Posada. Aquellos lectores interesados
            en colaborar pueden hacerlo aunque no tengan experiencia con plataformas
            de este tipo utilizando este sencillo manual.
        \end{itemize}
\end{itemize}

\figura{iberogre-otros.jpg}{scale=0.35}{Otros temas relacionados con IberOgre}{fig:iberogre-otros}{h}

\subsubsection{Artículos}

Cada artículo comienza con una introducción breve como ya hemos mencionado
anteriormente. A esta introducción le sigue un índice de contenidos con
enlaces que llevan directamente a las subsecciones del texto. Esta
funcionalidad
facilita el acceso a aquellos lectores que deseen buscar un apartado concreto.\\

\subsection{Ejemplos}

Los ejemplos en \wiki\ son una parte imprescindible del aprendizaje, proporcionan
una aplicación práctica del contenido teórico del texto que los preceden.
Estos ejemplos tratan de aglutinar lo que debería aprenderse tras leer el
artículo en una sola y sencilla aplicación descargable. Recordemos que en este
caso es más relevante la sencillez que la vistosidad y la calidad del código.
Tras cada ejemplo, se anima al lector a estudiar el código y realizar
modificaciones para demostrar su comprensión del mismo y dominio de la materia.
A continuación hacemos una breve descripción sobre cada uno de los ejemplos
de \wiki.\\

\subsubsection{Inicialización y cierre de Ogre}

Tras el artículo \textit{Inicialización y cierre de Ogre} el lector no sabe
mostrar nada en pantalla, ni siquiera crear una ventana en la que renderizar
elementos. Simplemente conoce la secuencia de inicialización del motor y
la forma para salir de una aplicación de forma ordenada y correcta. El ejemplo
se limita a crear el objeto principal de \textsc{Ogre3D} habiendo configurado
previamente el sistema de logs y habiendo mostrado un diálogo de configuración
al usuario. Tal y como se muestra en la figura \ref{fig:ejemplo-inicializacion},
simplemente inicializamos el motor para cerrarlo seguidamente.\\

\figura{ejemplo-inicializacion.jpg}{scale=0.5}{Ejemplo de inicialización y cierre de Ogre3D}{fig:ejemplo-inicializacion}{h}

\subsubsection{Creación básica de escenas}

En este artículo aprendíamos a crear una ventana de renderizado, configurar
el gestor de escenas, utilizar la gestión de recursos, cargar elementos
dentro del grafo de la escena y controlar el bucle de renderizado empleando
varias aproximaciones. El ejemplo aglutina todos estos conceptos creando
una sencilla aplicación que inicializa \textsc{Ogre3D}, carga un personaje
en pantalla y permanece a la espera de algún evento de salida. El resultado
puede verse claramente en la figura \ref{fig:ejemplo-escenas}. Se requieren
conocimientos de gestión de eventos de entrada con \textsc{OIS}, por lo que
es recomendable leer dicho artículo y seguir su ejemplo en primer lugar.\\

\figura{ejemplo-escenas.jpg}{scale=0.35}{Ejemplo de creación de escenas}{fig:ejemplo-escenas}{h}

\subsubsection{Materiales}

En el artículo sobre materiales en \textsc{Ogre3D} aprendemos a escribir
scripts definiendo materiales, a cargarlos en nuestra aplicación y a aplicarlos
a entidades como mallas tridimensionales. En el ejemplo para este artículo
tenemos una sencilla escena con una paredes y una esfera en el centro iluminada
por varios puntos de luz. Empleando el teclado numérico podemos cambiar el material
de la esfera para ver los efectos producidos. De esta manera se resume la
creación de materiales mediante scripts, su carga en memoria y aplicación
sobre entidades, puede verse una captura del ejemplo en ejecución en la 
figura \ref{fig:ejemplo-materiales}.

\figura{ejemplo-materiales.jpg}{scale=0.4}{Ejemplo de materiales}{fig:ejemplo-materiales}{h}

\subsubsection{Manipulación de nodos}

En el artículo sobre manipulación de nodos aprendemos a gestionar la información
básica de los nodos como visibilidad, nombre, sus descendientes, etc.
Partimos de la escena con el personaje vista en el
ejemplo de la creación básica de escenas. Utilizando la gestión de eventos
proporcionada por \textsc{OIS} capturamos las pulsaciones de las teclas
W, A, S y D para desplazar y rotar al personaje por el mundo. Empleando
la rueda del ratón podemos escalarlo. Puede verse una captura en la figura
\ref{fig:ejemplo-nodos}.\\

\figura{ejemplo-nodos.jpg}{scale=0.6}{Ejemplo de manipulación de nodos}{fig:ejemplo-nodos}{h}

\subsubsection{Luces, sombras y entorno}

En el artículo sobre iluminación aprendemos a gestionar varios efectos
lumínicos y de ambientación como la niebla, las fuentes de luz, las técnicas
de sombreado y los fondos. El ejemplo trata de aglutinar todos estos conceptos
en una aplicación interactiva que nos permite activar y desactivar varios
de los efectos anteriormente mencionados. Se crea una escena con un
plano texturizado a modo de suelo y un personaje sobre él. Los controles
son los siguientes y el resultado puede observarse en la figura \ref{fig:ejemplo-iliuminacion}.

\begin{itemize}
    \itemsep0em
    \item \textbf{s}: alterna entre distintas técnicas de sombreado. Es
    la mejor forma de ver la diferencia entre ambas.
    \item \textbf{d}: activa o desactiva el fondo.
    \item \textbf{n}: activa o desactiva la niebla.
    \item \textbf{1}: apaga o enciende la luz número 1 (roja).
    \item \textbf{2}: apaga o enciende la luz número 2 (azul).
    \item \textbf{3}: apaga o enciende la luz número 3 (verde).
    \item \textbf{4}: apaga o enciende la luz número 4 (amarilla).
\end{itemize}

\figura{ejemplo-iluminacion.jpg}{scale=0.35}{Ejemplo de iluminación}{fig:ejemplo-iliuminacion}{h}

\subsubsection{Animación}

En el artículo de animación se exponen las distintas técnicas para animar
entidades en \textsc{Ogre3D} así como la forma de activar y mezclar
varias animaciones al mismo tiempo. En el ejemplo cargamos a la mascota del
motor llamada \textit{Simbad} sobre un sencillo escenario y ofrecemos
controles al usuario para animarlo. Utilizando las teclas de dirección
\textit{Simbad} se desplaza con una animación y pulsando \textit{D} comienza
o para de bailar. Puede verse el resultado en la figura \ref{fig:ejemplo-animacion}.\\

\figura{ejemplo-animacion.jpg}{scale=0.55}{Ejemplo de animaciones}{fig:ejemplo-animacion}{h}

\subsubsection{Sistemas de partículas}

En este artículo se explican las técnicas de definición de sistemas de partículas
a través de scripts en texto plano así como su carga dentro de una escena.
En el ejemplo simplemente cargamos varios sistemas de partículas entre
los que el usuario puede alternar empleando las teclas numéricas del 1 al
9. El resultado puede verse en la figura \ref{fig:ejemplo-particulas}.\\

\figura{ejemplo-particulas.jpg}{scale=0.30}{Ejemplo de sistemas de partículas}{fig:ejemplo-particulas}{h}

\subsubsection{Sistema de Overlays}

La lección sobre el sistema de Overlays de \textsc{Ogre3D} incluye la sintaxis
y propiedades de los scripts que los definen así como su carga dentro del juego.
En el ejemplo se incluyen varios scripts para definir estas plantillas bidimensionales
ofreciendo cierta interactividad. Partimos del resultado del ejemplo de materiales
y añadimos un panel informativo con el nombre del material seleccionado
y el número de cuadros por segundo (\textit{FPS}) a la que se ejecuta
la aplicación. Si cambiamos de material con el teclado numérico, veremos
el cambio reflejado en el panel como se aprecia en la figura \ref{fig:ejemplo-overlays}.\\

\figura{ejemplo-overlays.jpg}{scale=0.25}{Ejemplo de Overlays}{fig:ejemplo-overlays}{h}

\subsubsection{Manejo básico de OIS}

\textsc{OIS} es la biblioteca que utilizamos en \wiki\ para la gestión
de dispositivos de entrada de usuario. En el artículo se detalla el proceso
de inicialización, configuración y cierre de la biblioteca así como el manejo
de dispositivos (joysticks, ratones y teclados) para consultar su estado
en un momento dado. También se explica la aproximación empleando \textit{Listeners}
\cite{gamm77} para dar respuesta a eventos. En el ejemplo se crea una aplicación
de \textsc{Ogre3D} sin ventana y permanecemos a la espera de eventos que
mostramos por la terminal.\\

\subsubsection{Extender la gestión de recursos, audio}

En este artículo se explica el proceso de extensión del sistema de gestión
de recursos de \textsc{Ogre3D} a través del desarrollo de un subsistema
de audio empleando la biblioteca \textsc{libSDL mixer}. En el ejemplo
se carga una escena con un escenario, una silla y un micrófono a modo
de teatro para monólogos. El usuario se pone en el papel de regidor
indicando qué sonido o melodía debe reproducirse en un momento dado empleando
el teclado numérico, se muestra un sencillo panel con la leyenda.

\begin{itemize}
    \itemsep0em
    \item \textbf{1}: sintonía de comienzo del programa.
    \item \textbf{2}: sintonía del contenido del programa.
    \item \textbf{3}: sintonía del final del programa.
    \item \textbf{4}: aplausos de un público ficticio.
    \item \textbf{5}: risas enlatadas.
    \item \textbf{6}: abucheos del público ficticio.
\end{itemize}

El resultado puede verse en la figura \ref{fig:ejemplo-audio}.\\

\figura{ejemplo-audio.jpg}{scale=0.3}{Ejemplo de extensión del sistema de recursos, audio}{fig:ejemplo-audio}{h}


\section{Implementación}

Tras enumerar los usuarios, artículos, detallar la estructura de los mismos
y la navegabilidad de la plataforma pasamos a la fase de implementación.
Nos centraremos en los puntos más reseñables de esta fase. En primer lugar
hablaremos de la elección del motor de wikis \textit{MediaWiki} y de sus
características principales. Posteriormente trataremos las plantillas que
hemos empleado para formatear la documentación. Finalmente abordaremos
la estructura general de los ejemplos. La escritura de los artículos 
en sí mismos es trivial y por ello no es comentada en esta sección.\\

% MediaWiki
\subsection{El motor MediaWiki}

\textit{MediaWiki} es un popular motor para aplicaciones wikis desarrollado
por la propia Fundación Wikimedia y utilizado en todos sus sitios como
Wikipedia, Wiktionary y Wikinews entre otros muchos. Está escrito en el
lenguaje de programación PHP y precisa de una base de datos de tipo SQL
para almacenar la información. Su primera versión fue liberada en enero
de 2002 y actualmente, la 1.17.0 es la última versión estable y fue lanzada
en junio de 2011. \cite{website:mediawiki}.\\

\figura{mediawiki.png}{scale=0.3}{Logo del motor MediaWiki}{fig:mediawiki}{h}

\textit{MediaWiki} es, con diferencia, el software para creación y gestión
de wikis más utilizado en todo el mundo. Es fácilmente extensible mediante
complementos y existen bots que automatizan tareas como la prevención
y corrección de vandalismos. Uno de estos bots es AVBot desarrollado
por Emilio José Rodríguez Posada \cite{website:avbot}.\\

Escribir en la sintaxis utilizada por el motor es bastante sencillo, ya que
utiliza un lenguaje mucho más simple que HTML. Esto es muy importante ya
que buscamos colaboradores para la plataforma educativa. Cuanto más accesible
sea, más probabilidades de éxito. Por ello se ha adjuntado la guía
redactada por Noelia Sales Montes y Emilio Rodríguez Posada \textit{Edición
de wikis WikiMedia} \cite{pdf:wikimedia}.\\

El sistema proporciona un método de edición sencillo y una navegación
cómoda con contenido multimedia. Podemos crear espacios de nombres y
secciones bien diferenciadas. Es altamente configurable y podemos ajustarlo
para que encaje perfectamente con los objetivos de \wiki. Del proceso de
instalación y configuración se ha encargado la Oficina de Software Libre
de la Universidad de Cádiz ya que dicha instalación se ha realizado en
sus servidores.\\

Su extensibilidad, posibilidades de configuración, la cantidad de documentación
disponible y la enorme y activa comunidad han hecho que nos decantemos
por el uso de \textit{MediaWiki} como software para dar soporte a \wiki.\\

% Plantillas
\subsection{Plantillas}

Las plantillas son fragmentos de código que aceptan parámetros a los que
podemos llamar desde otras páginas utilizando una sintaxis especial. Esto nos
ayuda a generalizar ciertas estructuras y ahorrar en recursos por parte del
servidor. En esta sección haremos un recorrido por todas las plantillas utilizadas,
sus parámetros y el resultado que producen al utilizarlas.\\

\begin{itemize}
    \item \textbf{Artículo}
    
    Esta plantilla se utiliza por cada artículo en el bloque de contenido.
    Incluye el nombre con el enlace al texto, una descripción y un icono
    representativo. Su código es el siguiente, pueden verse los parámetros
    identificados por su orden de aparición en la llamada a la plantilla
    y encerrados en llaves triples.
    
    \begin{lstlisting}[style=wiki]
{| style= "border: 0;
           margin: 0;
           background-color: inherit;"
           cellpadding="3"
           
 | valign="top" | [[Imagen:{{{1}}}|64px|link={{{2}}}]]
 | valign="top" | 
 | '''[[{{{2}}}]]'''<br>{{{3}}}
|}
    \end{lstlisting}
    
    Una llamada como esta:

    \begin{lstlisting}[style=wiki]
{{ Articulo
|fuego.png
|Sistemas de particulas
|Aprende a crear efectos especiales con Ogre como llamas, explosiones
 o nubes de humo para mejorar tus videojuegos.
}}
    \end{lstlisting}

    Produce el resultado de la figura \ref{fig:articulo}:\\
    
    \figura{articulo.jpg}{scale=0.4}{Plantilla Artículo}{fig:articulo}{h}

    
    \item \textbf{RecursoExterno}
    
    Para referenciar enlaces externos de una forma similar a los artículos
    se emplea esta plantilla. Cuenta con un título que lleva al destino
    y un icono relacionado que hace lo propio. Además, se añade una descripción.
    El código que consigue el efecto es el siguiente.
    
    \begin{lstlisting}[style=wiki]
{| style= "border: 0;
           margin: 0;
           background-color: inherit;"
           cellpadding="3"
           
| valign="top" | [[Imagen:{{{1}}}|64px|link={{{2}}}]]
| valign="top" | 
| '''[{{{2}}} {{{3}}}]'''<br>{{{4}}}
|}
    \end{lstlisting}
    
    Una llamada como la siguiente:
    
\begin{lstlisting}[style=wiki]
{{RecursoExterno
|cuaderno.png
|http://siondream.com/blog/category/proyectos/pfc/
|Blog oficial
|Noticias, experiencias y avances en IberOgre
}}
    \end{lstlisting}
    
    Produce el resultado que puede apreciarse en la figura \ref{fig:recursoexterno}.\\
    
    \figura{recursoexterno.jpg}{scale=0.45}{Plantilla RecursoExterno}{fig:recursoexterno}{h}
    
    \item \textbf{BloqueSeccion}\\
    
    Esta plantilla se emplea para conseguir el bloque de que preceden
    a las secciones de la wiki dentro del bloque contenido. Su código
    es el siguiente.
    
    \begin{lstlisting}[style=wiki]
<div style="padding-top:2px;
            padding-bottom:2px;
            padding-left:5px;
            padding-right:2px;
            border:1px solid #092C00;
            margin: 4px;
            background-color: #F0FFF0;">
'''{{{1}}}'''
</div>
    \end{lstlisting}
    
    \item \textbf{Calendario}
    
    En diversas páginas interesaba colocar un calendario con la fecha
    de la edición. Por ello, se ha creado esta plantilla que coloca la fecha
    actual en un formato de calendario clásico. Su código wiki 
    y HTML es el siguiente.
    
    \begin{lstlisting}[style=wiki]
<div style= "border:solid #ccc;
             background: #fff;
             border-width: 1px 3px 3px 1px;
             text-align: center;
             padding-top:3px;
             float:left;
             font-size: smaller;
             line-height: 1.3;
             margin-right: 4px;
             width: 7em">
             
    [[{{CURRENTDAYNAME}}]]
    [[{{CURRENTDAY}} de {{CURRENTMONTHNAME}}|
    <span style= "font-size: x-large;
                  width: 100%;
                  display: block;
                  padding:6px 0px">
        {{CURRENTDAY}}
    </span>]]
    
    <span style="display: block;"> [[{{CURRENTMONTHNAME}}]]</span>
    
    <span style= "background: #aaa;
                  color: #000;
                  display: block;">
    '''[[{{CURRENTYEAR}}]]'''
    </span>
</div>
    \end{lstlisting}
    
    El resultado producido es el que podemos ver en la figura \ref{fig:calendario}.\\
    
    \figura{calendario.jpg}{scale=0.45}{Plantilla Calendario}{fig:calendario}{h}
    
    \item \textbf{FicheroDescargable}
    
    Los ejemplos finales de cada artículo de \wiki\ pueden descargarse
    en forma de paquete para ser compilados, ejecutados o modificados
    por los lectores. Para ello, empleamos una plantilla que añade
    un enlace al fichero descargable, el título del artículo una
    descripción del ejemplo. Su código es el siguiente:
    
    \begin{lstlisting}[style=wiki]
{| style= "border: 0;
           margin: 0;
           background-color: inherit;"
           cellpadding="3"
| valign="top" | [[Imagen:{{{1}}}|64px|]]
| valign="top" | 
| '''[[media:{{{2}}}|{{{3}}}]]'''<br>{{{4}}}
|}
    \end{lstlisting}
    
    Empleado la siguiente llamada:
    
        \begin{lstlisting}[style=wiki]
{{FicheroDescargable
|ejemplo.png
|ogre_escenas.zip
|Ejemplo de creacion basica de escenas
|Pequena aplicacion que inicia Ogre y carga un personaje en la escena
}} 
    \end{lstlisting}
    
    Obtenemos el siguiente bloque (ver figura \ref{fig:ficherodescargable}).\\
    
    \figura{ficherodescargable.jpg}{scale=0.55}{Plantilla FicheroDescargable}{fig:ficherodescargable}{h}
    
\end{itemize}

% Estructura general de los ejemplos
\subsection{Estructura general de los ejemplos}

Todos los ejemplos de \wiki\ se han implementado de manera similar aunque
cada uno tiene sus particularidades en los aspectos concretos. Esta decisión
viene motivada por la intención de conseguir uniformidad tanto en el estilo
como en las soluciones. De esta forma, los lectores comprenderán los ejemplos
cada vez con mayor facilidad y se centrarán únicamente en las partes cambiantes,
lo realmente interesante y lo que se imparte en cada lección.\\

El esquema general puede verse en el diagrama de clases de diseño en la
figura \ref{fig:ejemplo-clases}. La clase \textit{AplicacionOgre} es la encargada
de iniciar el motor \textsc{Ogre3D} y la biblioteca \textsc{OIS} a través
de los métodos privados \textit{iniciarOgre()} e \textit{iniciarOIS()}.
Hereda de las clases observadoras \cite{gamm77} \textit{WindowEventListener},
\textit{KeyListener} y \textit{MouseListener}. La primera se encarga de
capturar eventos de pantalla (movimiento, cierre y redimensión), la segunda
toma eventos de teclado (pulsar y soltar tecla) mientras que la tercera
hace lo propio con el ratón (movimiento, pulsar botón y liberar botón).
Como puede verse, existe un gestor de evento para cada uno de ellos.\\

\figura{ejemplo-clases.jpg}{scale=0.45}{Diagrama de clases para los ejemplos}{fig:ejemplo-clases}{h}

Con una llamada a \textit{buclePrincipal()} comienza el bucle de juego
hasta que se presione la tecla escape o se cierre la ventana. Esta clase
está preparada para proporcionar lo básico de una aplicación interactiva
que use \textsc{Ogre3D}. Está pensada para que heredemos de ella en cada
ejemplo que creemos. Por este motivo, el bucle de juego y los gestores
de eventos son virtuales, podemos sobrecargar su comportamiento.\\

Efectivamente, en cada ejemplo creamos una clase hija que suele llevar
por nombre \textit{EscenaSimple}. Esta clase se encarga de crear, gestionar
la escena que se desarrolle en el ejemplo y controlar su lógica (interactividad).
Siempre cuenta con varios métodos de configuración: \textit{prepararRecursos()},
\textit{configurarSceneManager()}, \textit{crearCamara()} y \textit{crearEscena()}.
En cada ejemplo concreto se pueden añadir más métodos y atributos según
sea necesario.\\

Esta aproximación centrada en la reutilización de código y la sencillez
ha facilitado la implementación de ejemplos de forma rápida y directa. Además,
como ya hemos mencionado, facilita el aprendizaje a los lectores. Para
saber más sobre los ejemplos concretos, puede accederse a cualquier artículo
de \wiki\ y descargar el paquete correspondiente.\\

\section{Pruebas}

Un plan de pruebas elaborado nos garantiza un mínimo de calidad para
el proyecto. En este caso no podemos aplicar el concepto de prueba
de caja blanca ni prueba de caja negra ya que hablamos de un proyecto
de documentación (exceptuando los ejemplos finales de cada artículo).
En esta sección enumeraremos las pruebas que se han realizado sobre la 
plataforma de aprendizaje en su totalidad.\\

Es necesario mencionar que no se han realizado pruebas en un servidor de
preproducción ya que la instalación y configuración corría a cargo de la
Oficina de Software Libre de la Universidad de Cádiz. No obstante, el motor
\textit{MediaWiki} ha sido probado de forma extensiva y es muy estable
a día de hoy.\\

La plataforma era pública (aunque no abierta a ediciones) para todo el mundo
y con los primeros artículos llegaron los primeros lectores. \wiki\ ha sido
sometido a pruebas por parte de muchos usuarios. Éstos han podido ofrecer
su opinión a través del blog de desarrollo, de Twitter o del correo
electrónico.\\

A continuación adjuntamos la lista de pruebas realizadas junto a los
resultados obtenidos.\\

\begin{enumerate}
    \itemsep0em
    \item \textbf{Estructura de los artículos}
    
    \textit{¿Se cumple con la estructura acordada? ¿Cumple la estructura
    con los objetivos?}\\
    
    \wiki\ se ha dividido en las cuatro secciones establecidas durante la
    fase de análisis (capítulo \ref{iberogre-analisis} y página \pageref{iberogre-analisis}).
    Los lectores han manifestado que la división en introducción, matemáticas
    \textsc{Ogre3D} y otras tecnologías les facilitó la organización
    de su aprendizaje. Por tanto, se han cumplido con los objetivos y el
    resultado de la prueba es positivo.\\
    
    \item \textbf{Errores en el texto}
    
    \textit{¿Existen errores ortográficos, de sintaxis o expresión en los
    textos?}\\
    
    Cada artículo ha sido revisado tanto por su redactor inicial como por
    varios usuarios que se han prestado a colaborar con las correcciones.
    El uso del corrector ortográfico se ha combinado con lecturas humanas
    para detectar y eliminar los fallos de este tipo.\\
    
    \item \textbf{Plantillas}
    
    \textit{¿Las plantillas se visualizan correctamente? ¿Existen plantillas
    que devuelvan algún error?}\\
    
    Se ha comprobado que todas las llamadas a plantillas funcionan como
    se esperaba de ellas. Todas las llamadas se visualizan de forma correcta.\\
    
    \item \textbf{Etiquetado}
    
    \textit{¿Están todos los artículos debidamente etiquetados?}\\
    
    En \textit{MediaWiki} los artículos pueden etiquetarse de forma que
    encontrarlos sea más sencillo. En definitiva facilita la navegación
    ya que se crea una página de etiquetas listando a los artículos
    que contienen. En \wiki\ todos los artículos han sido etiquetados
    por lo que el resultado del test es positivo.\\
    
    \item \textbf{Usuario no registrado}
    
    \textit{¿Son los permisos de estos usuarios los correctos? ¿Pueden
    editar artículos o crear otros nuevos?}\\
    
    Como se dice en el análisis de la wiki, los usuarios no registrados
    pueden acceder a todo el contenido de \wiki\ aunque no pueden hacer
    ningún tipo de modificación. Se ha comprobado cerrando la sesión de
    nuestro usuario que, sin estar registrados en el sistema, no podemos
    crear ni modificar página alguna. Por tanto, el resultado es positivo.\\
    
    \item \textbf{Usuario registrado}
    
    \textit{¿Pueden los usuarios registrados crear o modificar contenido?
    ¿Tienen acceso de administración?}\\
    
    Tal y como se menciona en la prueba anterior, ya habíamos definido
    los permisos que debía tener cada usuario en el análisis. El usuario
    registrado debía poder acceder tanto al contenido en modo lectura
    como en modo escritura. Asimismo debía contar con la capacidad
    para crear nuevos artículos. No obstante, no se le permite el acceso
    a la configuración del sistema.\\
    
    Se han probado que puede trabajar con el contenido pero no así con la
    configuración de \wiki. Sólo el usuario administrador cuenta con
    acceso al servidor de producción y a la base de datos.\\
    
    \item \textbf{Usuario administrador}
    
    \textit{¿Cuenta el usuario administrador con todos los permisos?}\\
    
    El usuario administrador cuenta con los privilegios del usuario registrado
    y, además, debe poder acceder a parámetros de configuración así como
    poder instalar nuevas extensiones o gestionar las actuales. Durante el
    desarrollo, fueron necesarias varias tareas administrativas y este usuario
    pudo resolverlas sin problemas confirmando que el esquema de permisos
    funciona como se había planeado.\\
    
    \item \textbf{Ayuda}
    
    \textit{¿Es la ayuda lo suficientemente clara para que los lectores
    puedan empezar a colaborar? ¿Es ampliable la plataforma? ¿Los usuarios
    pueden enviar opiniones?}\\
    
    La Ayuda en \wiki\ es de extrema importancia ya que debe informar a los
    usuarios de las posibilidades que tienen para colaborar y animarles
    a hacerlo. La guía de edición en \textit{WikiMedia} de Noelia Sales Montes
    y Emilio José Rodríguez Posada cumplió su cometido ya que, lectores
    que nunca habían editado una wiki con este motor pudieron hacerlo.\\
    
    La plataforma se ha mostrado ampliable en gran medida. Tras la conclusión
    de este proyecto, la wiki cuenta con temas en los que se puede profundizar
    aún más. Algunos usuarios ya se han mostrado interesados en colaborar.\\
    
    En la página inicial de \wiki\ se adjuntan enlaces tanto al blog oficial
    de desarrollo, como a la cuenta oficial de Twitter como al correo electrónico.
    Los usuarios pueden ponerse en contacto (y de hecho lo han hecho en numerosas
    ocasiones) a través de cualquiera de estos medios.\\
    
    \item \textbf{Ejemplos}
    
    \textit{¿Funcionan correctamente los ejemplos? ¿Abarcan el contenido
    de los artículos? ¿Están debidamente documentados?}\\
    
    Esta prueba se ha aplicado a todos los ejemplos de \wiki\ y en la totalidad
    de los casos el resultado ha sido satisfactorio. Lo especificado en la
    fase de diseño de los ejemplos (sección \ref{iberogre-diseno}, página
    \pageref{iberogre-diseno}) se cumple con exactitud.\\
    
    Su diseño tiene como objetivo ilustrar todo el contenido del texto que
    precede a cada ejemplo y siempre se logra cumplir todo el temario. Cada
    ejemplo viene acompañado de una explicación que informa de su cometido,
    su diseño y funcionamiento. Esto, junto a las críticas positivas por
    parte de los usuarios, nos hace considerar que están debidamente
    documentados.\\
    
    \item \textbf{Multiplataforma}
    
    \textit{¿Es todo el código expuesto en la wiki multiplataforma?
    ¿Funcionan todos los ejemplos tanto en Windows como en GNU/Linux?}\\
    
    Uno de los objetivos principales de \wiki\ era ofrecer directrices
    sobre desarrollo de software multiplataforma, al menos para GNU/Linux
    y Windows. Todo el código de la plataforma de aprendizaje ha sido
    probado en ambos sistemas operativos con éxito.\\
    
    Cada ejemplo ha sido compilado y ejecutado en Ubuntu (GNU/Linux) y
    en Windows 7 utilizando el método expuesto en el artículo
    de creación de un entorno de desarrollo multiplataforma \cite{website:multiplataforma}
    obteniendo resultados positivos.\\
    
\end{enumerate}


\chapter{Desarrollo de Sion Tower}
\label{chap:desarrollo-siontower}
\section{Metodología}

\juego\ será un sistema software complejo y para su desarrollo se ha decidido
emplear la metodología \textit{RUP} (Rational Unified Process) \cite{larm02}. Así mismo,
se ha escogido \textit{UML} como la notación para modelar nuestro sistema.\\

Mediante la metodología \textit{RUP} podemos realizar un desarrollo orientado
a objetos de forma iterativa. Las ventajas que nos proporciona el paradigma
de la orientación a objetos consisten en una mayor reusabilidad (uno
de los objetivos principales del juego), escalabilidad, legibilidad y
sencillez de mantenimiento.\\

\section{Análisis}
\label{siontower-analisis}

\subsection{Especificación de requisitos del sistema}

En esta sección haremos un recorrido por los requisitos que debe cumplir
\juego\ tanto en el plano de interfaces, como de rendimiento o de
funcionalidades entre otros.\\

\subsubsection{Requisitos de interfaces externas}

En este apartado describiremos los requisitos relacionados con la conexión
entre el hardware y el software además de la interfaz con el usuario. En lo
referente a la conexión entre hardware y software utilizaremos varias tecnologías
en forma de bibliotecas libres. Como motor de renderizado nos decantamos por \textsc{Ogre3D},
para capturar la entrada del usuario, emplearemos \textsc{OIS}
(\textit{Object Oriented Input System}) \cite{website:ois}, para reproducir
sonidos emplearemos \textsc{libSDL} y su extensión \textsc{libSDL mixer} \cite{website:sdl}.
Finalmente, para gestionar los elementos de la interfaz a un nivel de abstracción
superior se hará uso de \textsc{MyGUI} \cite{website:mygui}.\\

\juego\ deberá soportar varias resoluciones de pantalla siempre que se mantenga
su resolución de aspecto 16:9. Además, debe ser posible establecer un modo
a pantalla completa si el usuario lo desea. Varias resoluciones aceptadas
por el videojuego deben ser:

\begin{itemize}
    \itemsep0em
    \item 640 x 360
    \item 960 x 544
    \item 1280 x 720
    \item 1920 x 1080
\end{itemize}

La navegación por los menús debe ser intuitiva y se realizará utilizando
el ratón. En la figura \ref{fig:navegacion-pantallas} pueden observarse
todas las pantallas de \juego\ dispuestas en un diagrama de flujo para
mostrar su navegación.\\

\figura{navegacion-pantallas.jpg}{scale=0.35}{Diagrama de flujo de las pantallas de Sion Tower}{fig:navegacion-pantallas}{h}

El \textbf{Menú principal} (figura \ref{fig:esquema-menu}) debe presentar
al jugador una escena de la Torre Sagrada con los enemigos aproximándose.
Mediante la opción \emph{Jugar} podrá dirigirse a la pantalla de selección
de perfil, utilizando \emph{Créditos} le será posible llegar a la pantalla
de mismo nombre y pulsando sobre \emph{Salir} se le llevará de vuelta al
sistema operativo.\\

\figura{esquema-menuprincipal.png}{scale=0.35}{Boceto del menú principal}{fig:esquema-menu}{H}

En la pantalla de \textbf{Selección de perfil} (figura \ref{fig:esquema-selperfil})
se mostrará una escena tridimensional del interior de la Torre. En un panel
se listarán todos los usuarios registrados en el sistema. El jugador podrá
seleccionar uno de ellos y pulsar \emph{Aceptar} para dirigirse a la
pantalla de selección de nivel. Si pulsa sobre \emph{Eliminar} el perfil
será destruido de los registros. Para crear un nuevo perfil debe introducir
el nombre deseado en el panel inferior y presionar \emph{Crear}. Desde este
menú debe ser posible regresar al menú principal.\\

\figura{esquema-selperfil.png}{scale=0.35}{Boceto de la pantalla de selección de perfil}{fig:esquema-selperfil}{H}

Dentro de la pantalla de \textbf{Selección de nivel} (figura \ref{fig:esquema-selnivel})
el usuario debe ver una escena del exterior de la Torre. Se mostrará un panel
con la lista de niveles de todo el juego. Cada nivel vendrá acompañado de una
miniatura, su nombre y descripción. El usuario irá desbloqueando niveles a medida
que avanza por el juego. Los niveles bloqueados aparecerán en tonos grisáceos
y se indicará que son inaccesibles. Pulsando sobre la miniatura de un nivel
disponible, el usuario accederá a la pantalla de juego. Puede volver atrás
pulsando el botón \emph{Atrás}.\\

\figura{esquema-selnivel.png}{scale=0.35}{Boceto de la pantalla de selección de nivel}{fig:esquema-selnivel}{H}

En la pantalla de \textbf{Juego} (figura \ref{fig:esquema-juego}) destacará
la escena 3D con el escenario, el personaje principal y los enemigos.
La barra inferior mostrará información relevante como la energía vital y
maná restantes, la experiencia obtenida y la lista de hechizos disponibles.
El hechizo seleccionado aparecerá resaltado y el usuario podrá seleccionar
otro distinto pulsando sobre su icono. Al pasar el ratón por encima de
un hechizo se mostrarán sus atributos: nombre, descripción y maná necesario.
El botón \emph{Pausa} abrirá el menú de pausa que nos permitirá volver al
juego, regresar a la pantalla de selección de nivel o desplazarnos
directamente al menú principal.\\

\figura{esquema-juego.png}{scale=0.35}{Boceto de la pantalla de juego}{fig:esquema-juego}{H}

Cuando terminemos el nivel de forma exitosa el usuario verá la pantalla de
\textbf{Fin de nivel} (figura \ref{fig:esquema-finnivel}). Se mostrará
al personaje celebrando su victoria y un panel resumiendo la actuación
del jugador durante la partida. La información será: enemigos eliminados,
vida restante, maná utilizado, tiempo empleado y los puntos obtenidos
en cada categoría. Para volver a jugar el usuario podrá pulsar sobre
\emph{Volver a jugar}, para seleccionar un nuevo nivel lo hará sobre
\emph{Seleccionar nivel} y si desea volver al menú principal, presionará
\emph{Volver al menú}.\\

\figura{esquema-finnivel.png}{scale=0.35}{Boceto de la pantalla de fin de nivel (victoria)}{fig:esquema-finnivel}{H}

La \textbf{Pantalla de créditos} (figura \ref{fig:esquema-creditos})
consiste en una pequeña escena 3D dentro de la Torre y un panel mostrando
a los creadores de \juego. Utilizando el botón \emph{Volver al menú} el
usuario volverá al menú principal.\\

\figura{esquema-creditos.png}{scale=0.35}{Boceto de la pantalla de créditos}{fig:esquema-creditos}{H}

\subsubsection{Requisitos funcionales}

\juego\ cuenta con la siguiente lista de requisitos funcionales:

\begin{itemize}
    \item Salir de la aplicación cerrando la ventana en cualquier momento.
    \item Gestionar varios perfiles de jugadores, cada uno con su último
    nivel desbloqueado y experiencia acumulada. Se permite la selección,
    creación y eliminación de perfiles.
    \item Seleccionar un nivel disponible para combatir contra la inteligencia
    artificial.
    \item Mover al personaje por el escenario.
    \item Lanzamiento de hechizos siempre y cuando se disponga del maná necesario.
    \item Selección del hechizo a lanzar.
    \item Capacidad de mover la cámara alrededor del personaje.
    \item Posibilidad de pausar y reanudar el juego.
\end{itemize}

\subsubsection{Requisitos de rendimiento}

Para disfrutar de \juego\ de forma satisfactoria será necesaria, al menos,
una resolución de 640 x 360 y una tasa de fotogramas por segundo igual
o superior a 25. El juego deberá estar lo suficientemente optimizado
en términos de uso de la CPU y consumo de memoria para alcanzar dicho
rendimiento en un equipo de características similares al siguiente:

\begin{itemize}
    \itemsep0em
    \item \textbf{Sistema operativo}: GNU/Linux, Windows XP Service Pack
    2, Windows Vista o Windows 7.
    \item \textbf{Procesador}: Pentium 2GHz o AMD.
    \item \textbf{Memoria}: 100 MB de memoria RAM disponibles.
    \item \textbf{Tarjeta de vídeo}: 128 MB de memoria y aceleración 3D.
    \item \textbf{Espacio en disco}: 100 MB.
    \item \textbf{Control}: ratón y teclado.
\end{itemize}

Además, la tasa de cuadros por segundo debe mantenerse estable. Es preferible
contar con una tasa media estable que con una alta que oscile en gran medida.\\

\subsubsection{Restricciones de diseño}

En un videojuego es mucho más relevante la fluidez que su consumo de memoria.
Actualmente, los equipos suelen contar con grandes cantidades de memoria
principal con velocidades aceptables pero no todos incorporan tarjetas
gráficas demasiado potentes. A la hora de diseñar \juego\ nos centraremos
en optimizar el uso de la CPU y del procesador de la tarjeta gráfica, recursos
que tienen mucho más impacto en la percepción del usuario acerca del juego.\\

\subsubsection{Requisitos del sistema software}

El sistema software deberá adherirse a los siguientes requisitos:

\begin{itemize}
    \itemsep0em
    \item Todo el código debe ser multiplataforma ofreciendo exactamente
    el mismo comportamiento en todos los sistemas operativos para los
    que se compile.
    \begin{itemize}
        \item \textbf{Windows}: el juego se probará en Windows XP, Windows Vista y Windows 7.
        \item \textbf{GNU/Linux}: el juego será probado en Ubuntu 10.10.
    \end{itemize}
    \item Se hará uso del ratón y el teclado. No obstante, cuando sea posible,
    se proporcionarán teclas de acceso rápido para un manejo más eficiente.
    \item El control deberá ser intuitivo, el jugador debería poder utilizar
    \juego\ sin necesidad de haber acudido al manual de usuario. La interfaz
    también debe presentar un manejo lógico y sencillo.
    \item El juego debe ser fácilmente ampliable con nuevos niveles y elementos.
\end{itemize}

\subsection{Modelo de casos de uso}

Para el modelo de casos de uso se ha seguido, como se ha mencionado
anteriormente la notación \textit{UML}. Los pasos para obtener el modelo
de casos de uso han sido los siguientes:

\begin{enumerate}
    \itemsep0em
    \item Identificar a todos los posibles usuarios del sistema y sus roles.
    \item Para cada rol, determinar todas las maneras posibles que éste cuenta
    para interactuar con el sistema.
    \item Creación de un caso de uso por cada objetivo que deba cumplir
    el sistema.
    \item Estructurar todos los casos de uso, por ejemplo, empleando relaciones
    de inclusión o extensión.
\end{enumerate}

\subsubsection{Diagrama de casos de uso}

En la figura \ref{fig:casosuso} se muestra el diagrama de casos de uso
para \juego.\\

\figura{casosuso.jpg}{scale=0.6}{Diagrama de casos de uso}{fig:casosuso}{H}

\subsubsection{Descripción de los casos de uso}

En esta sección adjuntaremos las descripciones de todos los casos de uso
anteriormente expuestos. Para ello emplearemos una notación en texto utilizando
un formato completo con plantilla. Se pretende que su lectura sea sencilla
y resulte accesible a la vez que directo.\\

\textbf{Caso de uso: Menú principal}

\begin{description}
    \item [Caso de uso] Menú principal
    \item [Descripción] Se le muestra al \jugador\ el menú principal desde
    el cual es posible acceder a la selección de perfil o a la pantalla de
    créditos.
    \item [Actores] \jugador.
    \item [Precondiciones] Ninguna.
    \item [Postcondiciones] Ninguna.
    \item [Escenario principal] $\quad$
        \begin{enumerate}
            \item El \jugador\ inicia la aplicación.
            \item El \sistema\ inicia el motor del juego y muestra el menú
            principal en la pantalla.
            \item El \jugador\ selecciona la opción Jugar.
            \item El \sistema\ accede a la pantalla de Selección de perfil.
        \end{enumerate}
    \item[Extensiones --- flujo alternativo] $\quad$
        \begin{description}
            \item [*a] El \jugador\ cierra la ventana.
                \begin{enumerate}
                    \item El \sistema\ libera los recursos y sale de la aplicación.
                \end{enumerate}
            \item [3a] El \jugador\ selecciona la opción Créditos.
                \begin{enumerate}
                    \item El \sistema\ accede a la pantalla de Créditos.
                \end{enumerate}
            \item [3b] El \jugador\ selecciona la opción Salir.
                \begin{enumerate}
                    \item El \sistema\ libera los recursos y sale de la aplicación.\\
                \end{enumerate}
        \end{description}
    
\end{description}

\textbf{Caso de uso: Selección de perfil}

\begin{description}
    \item [Caso de uso] Selección de perfil
    \item [Descripción] Se le muestra al \jugador\ la pantalla de selección
    de perfil. Es posible seleccionar, crear o eliminar perfiles. Una vez
    seleccionado un perfil, se le conduce hasta la pantalla de selección
    de nivel. También es posible volver al menú principal.
    \item [Actores] \jugador.
    \item [Precondiciones] Ninguna.
    \item [Postcondiciones] Se selecciona un perfil.
    \item [Escenario principal] $\quad$
        \begin{enumerate}
            \item El \jugador\ desea acceder a la pantalla de selección 
            de perfil.
            \item El \sistema\ muestra la pantalla de selección de perfil
            y carga los perfiles existentes en la lista.
            \item El \jugador\ selecciona un perfil de la lista y pulsa sobre
            la opción Aceptar.
            \item El \sistema\ accede a la pantalla de selección de nivel.
        \end{enumerate}
    \item[Extensiones --- flujo alternativo] $\quad$
        \begin{description}
            \item [*a] El \jugador\ cierra la ventana.
                \begin{enumerate}
                    \item El \sistema\ libera los recursos y sale de la aplicación.
                \end{enumerate}
            \item [3a] El \jugador\ selecciona un perfil de la lista y pulsa
            sobre la opción Eliminar.
                \begin{enumerate}
                    \item El \sistema\ elimina el perfil seleccionado.
                \end{enumerate}
            \item [3b] El \jugador\ decide crear un nuevo perfil. Introduce
            el nombre del nuevo usuario y pulsa sobre Crear.
                \begin{enumerate}
                    \item El \sistema\ comprueba que el nombre introducido
                    no existe y crea el nuevo perfil.
                \end{enumerate}
            \item [3c] El \jugador\ decide crear un nuevo perfil. Introduce
            el nombre del nuevo usuario y pulsa sobre Crear.
                \begin{enumerate}
                    \item El \sistema\ comprueba que el nombre introducido
                    existe y muestra el correspondiente error.
                \end{enumerate}
            \item [3d] El \jugador\ selecciona la opción Volver.
                \begin{enumerate}
                    \item El \sistema\ vuelve al menú principal.\\
                \end{enumerate}
        \end{description}
    
\end{description}

\textbf{Caso de uso: Selección de nivel}

\begin{description}
    \item [Caso de uso] Selección de nivel
    \item [Descripción] El \jugador\ ve la pantalla de selección de nivel.
    Puede escoger uno de entre los disponibles y acceder a la pantalla de
    juego.
    \item [Actores] \jugador.
    \item [Precondiciones] Se ha seleccionado un perfil.
    \item [Postcondiciones] Se selecciona un nivel.
    \item [Escenario principal] $\quad$
        \begin{enumerate}
            \item El \sistema\ carga los niveles (icono, nombre y descripción)
            en la lista de niveles distinguiendo los desbloqueados de los
            accesibles y muestra la pantalla. 
            \item El \jugador\ selecciona un nivel desbloqueado.
            \item El \sistema\ guarda el nivel como seleccionado y accede
            a la pantalla de juego.
        \end{enumerate}
    \item[Extensiones --- flujo alternativo] $\quad$
        \begin{description}
            \item [*a] El \jugador\ cierra la ventana.
                \begin{enumerate}
                    \item El \sistema\ libera los recursos y sale de la aplicación.
                \end{enumerate}
            \item [2a] El \jugador\ selecciona la opción Volver.
                \begin{enumerate}
                    \item El \sistema\ regresa a la pantalla de selección
                    de nivel.\\
                \end{enumerate}
        \end{description}
    
\end{description}


\textbf{Caso de uso: Jugar}

\begin{description}
    \item [Caso de uso] Jugar
    \item [Descripción] El \jugador\ comienza un nivel, interactúa con el
    y puede que gane o pierda.
    \item [Actores] \jugador.
    \item [Precondiciones] Se ha seleccionado un nivel.
    \item [Postcondiciones] Se completa un nivel (con éxito o no).
    \item [Escenario principal] $\quad$
        \begin{enumerate}
            \item El \sistema\ carga el nivel, los enemigos, la música y el
            personaje principal.
            \item El \sistema\ inicializa las estadísticas: enemigos eliminados
            y puntos por haber eliminado dichos enemigos, maná utilizado,
            vida restante y tiempo transcurrido.
            \item El \jugador\ y el \sistema\ interactúan durante la partida.
            \item El \sistema\ detecta que el \jugador\ ha eliminado todos los enemigos.
            \item El \sistema\ informa al \jugador\ de su victoria y se dirige
            hacia la pantalla de victoria.
        \end{enumerate}
    \item[Extensiones --- flujo alternativo] $\quad$
        \begin{description}
            \item [*a] El \jugador\ cierra la ventana.
                \begin{enumerate}
                    \item El \sistema\ libera los recursos y sale de la aplicación.
                \end{enumerate}
            \item [3a] El \sistema\ comprueba que un enemigo ha atacado al personaje
            y le ha impactado.
                \begin{enumerate}
                    \item El \sistema\ le resta una cantidad de vida al personaje
                    correspondiente con el poder del enemigo.
                \end{enumerate}
            \item [5a] El \sistema\ comprueba que la vida del protagonista
            llega a 0 y pierde la partida.
                \begin{enumerate}
                    \item El \sistema\ le muestra un mensaje de derrota:
                    \emph{¡Has fallado! ¡Pulsa espacio para intentarlo
                    de nuevo!}.
                    \item El \jugador\ pulsa espacio.
                    \item El \sistema\ carga la pantalla de selección de nivel.\\
                \end{enumerate}
        \end{description}
    
\end{description}


\textbf{Caso de uso: Pausar}

\begin{description}
    \item [Caso de uso] Pausar
    \item [Descripción] El \jugador\ selecciona pausar el nivel y puede
    reanudarlo o volver a otro menú.
    \item [Actores] \jugador.
    \item [Precondiciones] Se está jugando un nivel.
    \item [Postcondiciones] Ninguna.
    \item [Escenario principal] $\quad$
        \begin{enumerate}
            \item El \jugador\ pulsa el botón Pausa o presiona la tecla
            escape.
            \item El \sistema\ detiene todos los elementos del juego y
            muestra un menú de pausa.
            \item El \jugador\ selecciona el botón Reanudar.
            \item El \sistema\ vuelve a reanudar la partida.
        \end{enumerate}
    \item[Extensiones --- flujo alternativo] $\quad$
        \begin{description}
            \item [*a] El \jugador\ cierra la ventana.
                \begin{enumerate}
                    \item El \sistema\ libera los recursos y sale de la aplicación.
                \end{enumerate}
            \item [3a] El \jugador\ selecciona el botón Selección de nivel.
                \begin{enumerate}
                    \item El \sistema\ termina la partida y vuelve a la pantalla
                    de selección de nivel.
                \end{enumerate}
            \item [3b] El \jugador\ selecciona el botón Menú principal.
                \begin{enumerate}
                    \item El \sistema\ termina la partida y regresa a la
                    pantalla del menú principal.\\
                \end{enumerate}
        \end{description}
    
\end{description}


\textbf{Caso de uso: Lanzar hechizo}

\begin{description}
    \item [Caso de uso] Lanzar hechizo
    \item [Descripción] El \jugador\ selecciona un hechizo de entre los
    disponibles y lo lanza.
    \item [Actores] \jugador.
    \item [Precondiciones] Se está jugando un nivel.
    \item [Postcondiciones] Ninguna.
    \item [Escenario principal] $\quad$
        \begin{enumerate}
            \item El \jugador\ selecciona un hechizo de la barra inferior
            o a través de las teclas de acceso rápido 1, 2 o 3.
            \item El \jugador\ apunta hacia donde desea lanzar el hechizo
            con el ratón y presiona el botón izquierdo del mismo para
            lanzarlo.
            \item El \sistema\ comprueba que el personaje posee maná (energía
            mágica) suficiente para lanzar el hechizo seleccionado y lanza
            el hechizo al mundo.
            \item El \sistema\ detecta que el hechizo ha colisionado con
            un enemigo y hace explotar el proyectil.
            \item El \sistema\ le resta una cantidad de vida el enemigo
            correspondiente al poder del hechizo lanzado pero el enemigo
            no muere.
        \end{enumerate}
    \item[Extensiones --- flujo alternativo] $\quad$
        \begin{description}
            \item [*a] El \jugador\ cierra la ventana.
                \begin{enumerate}
                    \item El \sistema\ libera los recursos y sale de la aplicación.
                \end{enumerate}
            \item [3a] El \sistema\ detecta que el personaje no posee el maná
            suficiente para lanzar el hechizo.
                \begin{enumerate}
                    \item El \sistema\ informa del error al \jugador\ y finaliza
                    el caso de uso.
                \end{enumerate}
            \item [4a] El \sistema\ detecta que el hechizo ha colisionado con
            un objeto del escenario.
                \begin{enumerate}
                    \item El \sistema\ hace explotar al proyectil y es eliminado
                    del mundo.
                \end{enumerate}
            \item [4b] El \sistema\ no detecta que el hechizo haya colisionado.
            \item [5a] El \sistema\ le resta la vida al enemigo correspondiente
            al poder del hechizo y esta llega a 0 o menos.
                \begin{enumerate}
                    \item El \sistema\ elimina al enemigo del juego.\\
                \end{enumerate}
        \end{description}
    
\end{description}


\textbf{Caso de uso: Mover personaje}

\begin{description}
    \item [Caso de uso] Mover personaje
    \item [Descripción] El \jugador\ desplaza al personaje por el escenario.
    \item [Actores] \jugador.
    \item [Precondiciones] Se está jugando un nivel.
    \item [Postcondiciones] Ninguna.
    \item [Escenario principal] $\quad$
        \begin{enumerate}
            \item El \jugador\ presiona una de las teclas de movimiento.
            W (hacia donde mira la cámara), A (hacia la izquierda de la cámara),
            S (hacia la cámara) y D (hacia la derecha de la cámara).
            \item El \sistema\ mueve al personaje en dicha dirección
            teniendo en cuenta la orientación de la cámara y la velocidad
            del personaje. El \sistema\ orienta al personaje en dicha dirección.
            \item El \sistema\ comprueba que el personaje no ha colisionado
            con ningún elemento del escenario o enemigo.
        \end{enumerate}
    \item[Extensiones --- flujo alternativo] $\quad$
        \begin{description}
            \item [*a] El \jugador\ cierra la ventana.
                \begin{enumerate}
                    \item El \sistema\ libera los recursos y sale de la aplicación.
                \end{enumerate}
            \item [3a] El \sistema\ detecta que el personaje ha colisionado
            con un enemigo o alguna zona del escenario.
                \begin{enumerate}
                    \item El \sistema\ restaura la posición anterior del personaje.\\
                \end{enumerate}
        \end{description}
    
\end{description}


\textbf{Caso de uso: Mover cámara}

\begin{description}
    \item [Caso de uso] Mover cámara
    \item [Descripción] El \jugador\ rota la cámara alrededor del personaje principal.
    \item [Actores] \jugador.
    \item [Precondiciones] Se está jugando un nivel.
    \item [Postcondiciones] Ninguna.
    \item [Escenario principal] $\quad$
        \begin{enumerate}
            \item El \jugador\ pulsa el botón derecho del ratón y lo mueve
            en una dirección (arriba, abajo, izquierda o derecha).
            \item El \sistema\ desplaza y rota la cámara de forma que gire
            alrededor del personaje.
            \item El \sistema\ comprueba que la cámara no se haya salido
            de sus límites de rotación.
        \end{enumerate}
    \item[Extensiones --- flujo alternativo] $\quad$
        \begin{description}
            \item [*a] El \jugador\ cierra la ventana.
                \begin{enumerate}
                    \item El \sistema\ libera los recursos y sale de la aplicación.
                \end{enumerate}
            \item [3a] El \sistema\ percibe que la cámara se ha salido de sus
            límites de rotación.
                \begin{enumerate}
                    \item El \sistema\ corrige la rotación de la cámara.\\
                \end{enumerate}
        \end{description}
    
\end{description}

\textbf{Caso de uso: Victoria}

\begin{description}
    \item [Caso de uso] Victoria
    \item [Descripción] El \jugador\ ve un resumen con su actuación en la
    partida y acumula puntos de experiencia. Según el nivel que se haya
    jugador, es posible que desbloquee un nuevo nivel. Puede volver a jugar,
    volver a la pantalla de selección de nivel o regresar al menú principal.
    \item [Actores] \jugador.
    \item [Precondiciones] El \jugador\ acaba de finalizar un nivel.
    \item [Postcondiciones] Se actualiza la experiencia del jugador.
    \item [Escenario principal] $\quad$
        \begin{enumerate}
            \item El \sistema\ calcula la experiencia obtenida en la
            partida y la muestra desglosada en los apartados: enemigos eliminados,
            tiempo empleado, vida restante y maná utilizado. Actualiza
            la experiencia total del perfil seleccionado.
            \item El \sistema\ comprueba que el nivel jugado era el último
            nivel desbloqueado para el perfil seleccionado. Desbloquea
            el siguiente nivel y muestra el mensaje: \emph{Has desbloqueado
            un nuevo nivel}. Se actualiza el último nivel desbloqueado
            para el perfil actual.
            \item El \jugador\ selecciona la opción Selección de nivel.
            \item El \sistema\ acude a la pantalla de selección de nivel.
        \end{enumerate}
    \item[Extensiones --- flujo alternativo] $\quad$
        \begin{description}
            \item [*a] El \jugador\ cierra la ventana.
                \begin{enumerate}
                    \item El \sistema\ libera los recursos y sale de la aplicación.
                \end{enumerate}
            \item [2a] El \sistema\ comprueba que el nivel jugado no era
            el último desbloqueado para el perfil seleccionado y muestra
            el mensaje: \emph{Has completado este nivel de nuevo}.
            \item [2b] El \sistema\ comprueba que el nivel jugador era el
            último nivel desbloqueado para ese perfil pero ya no quedan
            niveles adicionales, por tanto muestra el mensaje: \emph{Has
            conseguido salvar la Torre Sagrada}.
            \item [3a] El \jugador\ selecciona la opción Volver a jugar.
                \begin{enumerate}
                    \item El \sistema\ inicia la pantalla de juego con
                    el mismo nivel seleccionado.
                \end{enumerate}
            \item [3b] El \jugador\ selecciona la opción Volver al menú.
                \begin{enumerate}
                    \item El \sistema\ regresa al menú principal.\\
                \end{enumerate}
        \end{description}
    
\end{description}


\textbf{Caso de uso: Créditos}

\begin{description}
    \item [Caso de uso] Créditos
    \item [Descripción] Se muestra la pantalla de créditos con los creadores
    del juego, el \jugador\ puede volver al menú principal.
    \item [Actores] \jugador.
    \item [Precondiciones] Ninguna.
    \item [Postcondiciones] Ninguna.
    \item [Escenario principal] $\quad$
        \begin{enumerate}
            \item El \sistema\ muestra la pantalla de créditos.
            \item El \jugador\ selecciona la opción Volver.
            \item El \sistema\ regresa al menú principal.
        \end{enumerate}
    \item[Extensiones --- flujo alternativo] $\quad$
        \begin{description}
            \item [*a] El \jugador\ cierra la ventana.
                \begin{enumerate}
                    \item El \sistema\ libera los recursos y sale de la aplicación.\\
                \end{enumerate}
        \end{description}
    
\end{description}

\subsection{Modelo conceptual de datos}

En esta sección del análisis expondremos el modelo conceptual de datos
del sistema. Se listarán las clases conceptuales y las relaciones que
existen entre ellas junto a una breve descripción. Para ello emplearemos
un diagrama de clases siguiendo
la notación \textit{UML}. Nótese que los nombres de las clases están en inglés,
se decidió tomar dicha aproximación con el objetivo de poder conseguir
colaboradores en un futuro. Cualquier desarrollador de habla no hispana
tendría dificultades para colaborar aunque el proyecto le interesase.\\

\begin{description}
    \item [State] clase que modela un estado genérico del juego, cada estado
    correspondería a una pantalla diferente. Los estados cuentan con una
    canción (\textit{Song}) como música ambiental y un número no determinado de efectos
    de sonido (\textit{SoundFX}). 
    \item [StateMenu] estado que corresponde al menú principal del juego
    (ver figura \ref{fig:esquema-menu}). Cuenta con una escena 3D y los
    elementos de interfaz    necesarios expuestos en el boceto anteriormente
    referenciado.
    \item [StateProfileSelection] estado que modela la pantalla de selección
    de perfil cuyo boceto puede verse en la figura \ref{fig:esquema-selperfil}.
    Además de la escena tridimensional y los elementos de la interfaz, permite
    poder gestionar perfiles (\textit{Profile}).
    \item [StateLevelSelection] estado que representa la pantalla de selección
    de nivel, su boceto aparece en la figura \ref{fig:esquema-selnivel}.
    Cuenta con una escena 3D, elementos de interfaz y una lista de niveles
    para seleccionar (\textit{Level}). Los niveles estarán bloqueados o
    desbloqueados en función del perfil seleccionado (\textit{Profile}).
    \item [StateGame] estado de juego, su boceto puede verse en la figura
    \ref{fig:esquema-juego}. El estado de juego gestiona un nivel (\textit{Level}),
    las apariciones de enemigos (\textit{EnemySpawn}), el jugador (\textit{Player}),
    los enemigos que actúan durante un momento determinado (\textit{Enemy})
    y los hechizos lanzados (\textit{Spell}). Además, lleva las estadísticas
    de juego (\textit{GameStats}). Se encarga de gestionar el HUD
    (\textit{Heads Up Display} o controles de juego) y de renderizar la escena.
    Por último, hace un seguimiento de la cámara y los dispositivos de
    entrada (teclado y ratón).
    \item [StateVictory] es el estado que modela la pantalla de victoria
    tras haber completado un nivel de juego y puede observarse su esquema
    en la figura \ref{fig:esquema-finnivel}. Además de la escena 3D y los
    elementos necesarios de la interfaz, se encarga de actualizar el perfil
    seleccionado (\textit{Profile}) añadiéndole experiencia y desbloqueando
    un nuevo nivel en caso de ser necesario. 
    \item [StateCredits] corresponde al estado que gestiona la pantalla de
    visualización de créditos. Simplemente cuenta con una escena 3D y elementos
    sencillos de interfaz. Su boceto puede verse en la figura \ref{fig:esquema-creditos}.
    \item [Profile] representa un perfil de usuario dentro de \juego. Cada
    perfil almacena información como el nombre de usuario, la cantidad de
    experiencia acumulada y el número del último nivel desbloqueado. Al crear
    un nuevo perfil la experiencia es igual a cero y el nivel desbloqueado
    es el primero.
    \item [Level] modela y gestiona toda la información de un nivel. De cada
    nivel necesitamos saber su identificador, nombre y descripción. Así mismo,
    también almacena información sobre el nombre de la canción que debe reproducirse
    mientras se juega a dicho nivel y la posición inicial del protagonista.
    Contiene todas las apariciones de los enemigos (\textit{EnemySpawn}) y
    la malla de navegación (\textit{NavigationMesh}). Por supuesto, también
    contiene los detalles sobre los objetos del escenario y la iluminación.
    \item [Enemy Spawn] modela la aparición de un enemigo a lo largo del
    desarrollo de una partida. De cada aparición interesa saber el tipo
    de enemigo al que se refiere, su posición y orientación iniciales
    así como el momento (en segundos desde el inicio del nivel) en el que
    hará aparición.
    \item [NavigationMesh] la malla de navegación representa la zona transitable
    por los enemigos (\textit{Enemy}) y está formada por una colección
    de vértices y triángulos conexos. En definitiva es un grafo conexo sobre
    el que se realizan operaciones de inteligencia artificial y búsqueda
    de caminos.
    \item [PointPath] representa una ruta formada por puntos en el espacio
    que lleva de un punto del escenario a otro.
    \item [GameObject] modela de forma genérica objetos del juego con un
    modelo colisionable (\textit{Body}). Cuentan con una posición, escala
    y orientación determinadas. 
    \item [Actor] esta clase engloba de forma genérica a los elementos dinámicos
    y activos del juego como el jugador (\textit{Player}) y los enemigos
    (\textit{Enemy}). Hereda de \textit{GameObject} pero además aporta
    energía vital máxima, energía vital actual, poder, velocidad actual,
    aceleración, velocidad lineal máxima y velocidad angular. Incluye efectos
    de sonido (\textit{SoundFX}) como el de aparición, el de ataque
    y el de recibir daño.
    \item [Player] hereda de \textit{Actor} y representa al personaje
    protagonista de la aventura. Aporta su energía mágica actual, el
    hechizo seleccionado, su posición anterior (en caso de que colisione
    con un obstáculo, ésta sería restaurada) y el tiempo de recuperación
    de la energía mágica o maná.
    \item [Enemy] los enemigos cuentan con su tipo concreto (goblin, diablillo
    o gólem de hielo) y un tiempo de recuperación entre ataques. Cuando 
    un enemigo desea ir de un punto a otro del escenario obtiene una ruta
    de puntos \textit{PointPath} que ha de seguir poco a poco.
    \item [Spell] representa los hechizos o proyectiles mágicos del juego
    que lanza el protagonista (\textit{Player}). De cada hechizo interesa saber su tipo
    (Bola de fuego, Furia de Gea o Ventisca), la dirección hacia la que
    se dirige, su nombre, descripción, poder, coste en maná, velocidad,
    tiempo que dura su explosión y los efectos de sonido (\textit{SoundFX})
    que se producen al ser lanzado y al impactar.
    \item [GameStats] modela las estadísticas de juego que se generan con
    la actuación del protagonista (\textit{Player}) durante la partida.
    Se cada partida es necesario conocer el código del nivel que se juega,
    el número de enemigos eliminados (\textit{Enemy}), los puntos que se ganan al eliminar
    a cada tipo de enemigo, los puntos por haber destruido enemigos, el maná
    utilizado y los puntos por ello. También queremos conocer la vida restante
    y los puntos que se ganan con ella. Por último es necesario recuperar
    la duración de la partida y los puntos que recibe el jugador por haber
    vencido en dicho tiempo.
    \item [SoundFX] representa un efecto de sonido de corta duración hace
    las veces de abstracción de la biblioteca de audio \textsc{libSDL mixer}.
    Estos efectos deben estar en formato \texttt{.wav}.
    \item [Song] modela una pista de audio de mayor duración y se utiliza
    para reproducir la banda sonora. También abstrae ciertos aspectos
    de la biblioteca \textsc{libSDL mixer} y sólo acepta ficheros en formato
    \texttt{.ogg}.
    \item [Body] representa un cuerpo colisionable de algún elemento del
    juego, es decir, algo con lo que se puede colisionar y el sistema
    está preparado para detectarlo. De ellos interesa saber las formas
    que componen el cuerpo colisionable (\textit{Shape}) y su tipo. El tipo
    es importante ya que podremos detectar colisiones entre dos tipos concretos
    de cuerpos y detectar las demás, por ejemplo entre enemigos y hechizos.
    \item [Shape] modela una forma geométrica tridimensional sencilla de
    forma genérica. Son los componentes que forman los cuerpos colisionables
    (\textit{Body}). De ellas sólo interesa saber su nombre.
    \item [Sphere] clase hija de \textit{Shape} que modela una esfera sencilla.
    De cada esfera necesitamos saber su centro en coordenadas locales y
    su radio.
    \item [Plane] una nueva especialización de \textit{Shape} para modelar
    un plano infinito en tres dimensiones. Para representar el plano utilizamos
    un punto del plano (distancia con respecto al origen) y un vector normal
    (perpendicular) al plano.
    \item [AxisAlignedBox] se trata de un nuevo tipo de forma (\textit{Shape})
    que representa un hexaedro rectangular alineado con los ejes de coordenadas.
    De él necesitamos saber para poder representarlo correctamente sus vértices
    mínimo y máximo.
    \item [OrientedBox] es el cuarto y último tipo de forma (\textit{Shape})
    del modelo de datos inicial del videojuego. Es un hexaedro rectangular
    no alineado con los ejes, sino que cuenta con rotación. Es necesario
    conocer su centro, las tres distancias a las caras de cada eje local
    y los ejes locales y rotados sobre los que se basa.
    \item [CollisionManager] es el gestor de colisiones del juego, gestiona
    todos los cuerpos colisionables (\textit{Body}) y es capaz de conocer
    cuándo entran en contacto y avisar de tal evento. 
\end{description}

\figura{clasesconceptuales1.jpg}{scale=0.4}{Diagrama de clases conceptuales (parte 1)}{fig:clasesconceptuales1}{h}

Por razones de espacio y de dificultad a la hora de distribuir todas las
clases conceptuales en un mismo diagrama, se ha decidido separarlas en
dos esquemas diferentes. El primero corresponde al de la figura \ref{fig:clasesconceptuales1}
e incluye los subistemas relacionados con las pantallas de juego. Por otro
lado, el segundo diagrama puede verse en la figura \ref{fig:clasesconceptuales2}
y abarca el subsistema de objetos de juego, modelos colisionables e inteligencia
artificial.\\

\figura{clasesconceptuales2.jpg}{scale=0.55}{Diagrama de clases conceptuales (parte 2)}{fig:clasesconceptuales2}{h}


\subsection{Modelo de comportamiento del sistema}

En esta sección desarrollaremos el modelo de comportamiento del sistema.
Consideraremos al sistema como ente que engloba a todos los objetos. El modelo
se dividirá en dos partes bien diferenciadas:

\begin{itemize}
    \item Diagramas de secuencia de sistema: nos muestran la secuencia de
    eventos entre actores y sistema. También nos ayudan a identificar las
    operaciones del sistema.
    \item Contratos para las operaciones del sistema: describen en detalle
    qué hace cada operación del sistema.
\end{itemize}

No todos los posibles diagramas de secuencia ni contratos de operaciones
aparecerán especificados. En este documento nos centraremos en los más relevantes,
es decir, los que impliquen algún tipo de cambio en el sistema.\\


\textbf{Caso de uso: Menú (escenario principal)}

\figura{sequence-menu1.jpg}{scale=0.60}{Diagrama de secuencia: Menú (escenario principal)}{fig:sequence-menu1}{h}

\begin{description}
    \itemsep0em
    \item [Operación] InicioAplicacion()
    \item [Actores] \jugador, \sistema.
    \item [Responsabilidades] iniciar la aplicación y preparar el sistema
    para la ejecución del juego. Mostrar el menú principal.
    \item [Precondiciones] ninguna.
    \item [Postcondiciones] $\quad$
        \begin{itemize}
            \itemsep0em
            \item Creación de un objeto \textit{stateMenu} de la clase
            \textit{StateMenu}.
            \item Creación de un objeto \textit{song} de la clase \textit{Song}.\\
        \end{itemize}
\end{description}

\begin{description}
    \itemsep0em
    \item [Operación] SeleccionarJugar()
    \item [Actores] \jugador, \sistema.
    \item [Responsabilidades] salir del menú principal y entrar en la
    pantalla de selección de perfil.
    \item [Precondiciones] $\quad$
        \begin{itemize}
            \itemsep0em
            \item Existe un objeto \textit{stateMenu} de la clase \textit{StateMenu}.
        \end{itemize}
    \item [Postcondiciones] $\quad$
        \begin{itemize}
            \itemsep0em
            \item Destrucción del objeto \textit{stateMenu}.
            \item Se destruye el objeto \textit{song}.\\
        \end{itemize}
\end{description}

\textbf{Caso de uso: Menú (escenario 3a)}

\figura{sequence-menu2.jpg}{scale=0.60}{Diagrama de secuencia: Menú (escenario 3a)}{fig:sequence-menu2}{h}

\begin{description}
    \itemsep0em
    \item [Operación] SeleccionarCreditos()
    \item [Actores] \jugador, \sistema.
    \item [Responsabilidades] salir del menú principal y entrar en la
    pantalla de créditos.
    \item [Precondiciones] $\quad$
        \begin{itemize}
            \itemsep0em
            \item Existe un objeto \textit{stateMenu} de la clase \textit{StateMenu}.
        \end{itemize}
    \item [Postcondiciones] $\quad$
        \begin{itemize}
            \itemsep0em
            \item Destrucción del objeto \textit{stateMenu}.
            \item Se destruye el objeto \textit{song}.\\
        \end{itemize}
\end{description}


\textbf{Caso de uso: Menú (escenario 3b)}

\figura{sequence-menu3.jpg}{scale=0.60}{Diagrama de secuencia: Menú (escenario 3b)}{fig:sequence-menu3}{h}

\begin{description}
    \itemsep0em
    \item [Operación] SeleccionarSalir()
    \item [Actores] \jugador, \sistema.
    \item [Responsabilidades] salir del menú principal y cerrar la aplicación.
    \item [Precondiciones] $\quad$
        \begin{itemize}
            \itemsep0em
            \item Existe un objeto \textit{stateMenu} de la clase \textit{StateMenu}.
        \end{itemize}
    \item [Postcondiciones] $\quad$
        \begin{itemize}
            \itemsep0em
            \item Destrucción del objeto \textit{stateMenu}.\\
        \end{itemize}
\end{description}


\textbf{Caso de uso: Selección de perfil (escenario principal)}

\figura{sequence-perfil1.jpg}{scale=0.60}{Diagrama de secuencia: Selección de perfil (escenario principal)}{fig:sequence-perfil1}{h}

\begin{description}
    \itemsep0em
    \item [Operación] PantallaSeleccionPerfil()
    \item [Actores] \jugador, \sistema.
    \item [Responsabilidades] cargar y mostrar la pantalla de selección
    de perfil.
    \item [Precondiciones] ninguna.
    \item [Postcondiciones] $\quad$
        \begin{itemize}
            \itemsep0em
            \item Creación de un objeto \textit{stateProfile} de
            la clase \textit{StateProfile}.
            \item Creación de objetos de la clase \textit{Profile} por cada
            perfil que exista en memoria secundaria.
            \item Creación de un objeto \textit{song} de la clase \textit{Song}.\\
        \end{itemize}
\end{description}

\begin{description}
    \itemsep0em
    \item [Operación] SeleccionarPerfil(nombre)
    \item [Actores] \jugador, \sistema.
    \item [Responsabilidades] marcar un perfil determinado como seleccionado.
    \item [Precondiciones]$\quad$
        \begin{itemize}
            \itemsep0em
            \item Existe un perfil \textit{profile} con \textit{profile.name = nombre}.
        \end{itemize}
    \item [Postcondiciones] $\quad$
        \begin{itemize}
            \itemsep0em
            \item Marcado del perfil \textit{profile} como seleccionado.
            \item Se destruye el objeto \textit{song}.
            \item Destrucción del objeto \textit{stateProfile}.\\
        \end{itemize}
\end{description}


\textbf{Caso de uso: Selección de perfil (escenario 3a)}

\figura{sequence-perfil2.jpg}{scale=0.60}{Diagrama de secuencia: Selección de perfil (escenario 3a)}{fig:sequence-perfil2}{h}

\begin{description}
    \itemsep0em
    \item [Operación] EliminarPerfil(nombre)
    \item [Actores] \jugador, \sistema.
    \item [Responsabilidades] elimina un perfil del sistema.
    \item [Precondiciones]$\quad$
        \begin{itemize}
            \itemsep0em
            \item Existe un perfil \textit{profile} con \textit{profile.name = nombre}.
        \end{itemize}
    \item [Postcondiciones] $\quad$
        \begin{itemize}
            \itemsep0em
            \item Destrucción del objeto \textit{profile}.\\
        \end{itemize}
\end{description}


\textbf{Caso de uso: Selección de perfil (escenario 3b)}

\figura{sequence-perfil3.jpg}{scale=0.60}{Diagrama de secuencia: Selección de perfil (escenario 3b)}{fig:sequence-perfil3}{h}

\begin{description}
    \itemsep0em
    \item [Operación] CrearPerfil(nombre)
    \item [Actores] \jugador, \sistema.
    \item [Responsabilidades] crea un nuevo perfil en el sistema.
    \item [Precondiciones]$\quad$
        \begin{itemize}
            \itemsep0em
            \item No existe un perfil \textit{profile} con \textit{profile.name = nombre}.
        \end{itemize}
    \item [Postcondiciones] $\quad$
        \begin{itemize}
            \itemsep0em
            \item Creación de un objeto \textit{profile} de la clase
            \textit{Profile}.
            \item Modificación de atributos: \textit{profile.name = nombre},
            \textit{profile.unlockedLevel = 1} y \textit{profile.experience = 0}.\\
        \end{itemize}
\end{description}


\textbf{Caso de uso: Selección de nivel (escenario principal)}

\figura{sequence-nivel1.jpg}{scale=0.60}{Diagrama de secuencia: Selección de nivel (escenario principal)}{fig:sequence-nivel1}{h}

\begin{description}
    \itemsep0em
    \item [Operación] PantallaSeleccionNivel()
    \item [Actores] \jugador, \sistema.
    \item [Responsabilidades] crea y muestra la pantalla de selección de nivel.
    \item [Precondiciones] ninguna.
    \item [Postcondiciones] $\quad$
        \begin{itemize}
            \itemsep0em
            \item Creación de un objeto \textit{stateLevel} de la clase
            \textit{StateLevel}.
            \item Creación de objetos de la clase \textit{Level} por cada
            nivel encontrado en memoria secundaria.
            \item Creación de un objeto \textit{song} de la clase \textit{Song}.\\
        \end{itemize}
\end{description}

\begin{description}
    \itemsep0em
    \item [Operación] SeleccionarNivel(nivel)
    \item [Actores] \jugador, \sistema.
    \item [Responsabilidades] crea y muestra la pantalla de selección de nivel.
    \item [Precondiciones] $\quad$
        \begin{itemize}
            \itemsep0em
            \item Existe un objeto \textit{level} de la clase \textit{Level}
            de forma que \textit{level.id = nivel}.
        \end{itemize}
    \item [Postcondiciones] $\quad$
        \begin{itemize}
            \itemsep0em
            \item Marcado del objeto \textit{level} como nivel seleccionado.
            \item Se destruye el objeto \textit{song}.
            \item Destrucción del objeto \textit{stateLevel}.\\
        \end{itemize}
\end{description}


\textbf{Caso de uso: Jugar (escenario principal)}

\figura{sequence-juego1.jpg}{scale=0.60}{Diagrama de secuencia: Jugar (escenario principal)}{fig:sequence-juego1}{h}

\begin{description}
    \itemsep0em
    \item [Operación] PantallaJuego(nivel)
    \item [Actores] \jugador, \sistema.
    \item [Responsabilidades] carga y muestra la pantalla de juego con el
    nivel seleccionado, inicia el juego.
    \item [Precondiciones] ninguna.
    \item [Postcondiciones] $\quad$
        \begin{itemize}
            \itemsep0em
            \item Creación de un objeto \textit{stateGame} de la clase
            \textit{StateGame}.
            \item Creación de un objeto \textit{song} de la clase \textit{Song}.
            \item Creación de un objeto \textit{plater} de la clase \textit{Player}
            en la posición que indique el nivel.
            \item Creación de una relación entre \textit{player.body} y 
            \textit{CollisionManager}.
            \item Creación de un objeto de la clase \textit{Enemy} por cada
            enemigo que exista en el juego en la posición que indique el nivel.
            \item Creación de una relación entre cada \textit{enemy.body}
            y \textit{CollisionManager}.
            \item Creación de un objeto \textit{gameStats} de la clase \textit{GameStats}
            con las estadísticas iniciales.
            \item Creación de un objeto \textit{navigationMesh} de la clase
            \textit{NavigationMesh} con la malla inicializada.
            \item Creación de enlaces entre los objetos de la clase \textit{Enemy}
            y \textit{navigationMesh}.\\
        \end{itemize}
\end{description}

\begin{description}
    \itemsep0em
    \item [Operación] Interactuar()
    \item [Actores] \jugador, \sistema.
    \item [Responsabilidades] permite al \jugador\ interactuar con el mundo
    3D y los elementos que lo rodean.
    \item [Precondiciones]$\quad$
        \begin{itemize}
            \itemsep0em
            \item Existe un objeto \textit{stateGame}.
            \item Existe un objeto \textit{player}.
            \item Existe algún objeto de la clase \textit{Enemy}.
        \end{itemize}
    \item [Postcondiciones] ninguna.\\
\end{description}


\textbf{Caso de uso: Jugar (escenario 3a)}

\figura{sequence-juego3.jpg}{scale=0.60}{Diagrama de secuencia: Jugar (escenario 3a)}{fig:sequence-juego3}{h}


\textbf{Caso de uso: Jugar (escenario 5a)}


\figura{sequence-juego2.jpg}{scale=0.60}{Diagrama de secuencia: Jugar (escenario 5a)}{fig:sequence-juego2}{h}

\begin{description}
    \itemsep0em
    \item [Operación] ReintentarPartida()
    \item [Actores] \jugador, \sistema.
    \item [Responsabilidades] se destruye el estado de juego junto a sus
    elementos y se regresa a la pantalla de selección de nivel.
    \item [Precondiciones]$\quad$
        \begin{itemize}
            \itemsep0em
            \item Existe un objeto \textit{stateGame}.
            \item Existe un objeto \textit{player} de forma que
            \textit{player.life <= 0}.
        \end{itemize}
    \item [Postcondiciones]$\quad$
        \begin{itemize}
            \itemsep0em
            \item Se destruye el objeto \textit{navigationMesh}.
            \item Se destruye el objeto \textit{song}.
            \item Se eliminan el enlace entre \textit{player.body} y \textit{CollisionManager}.
            \item Se destruye el objeto \textit{player}.
            \item Se eliminan los enlaces entre los \textit{enemy.body}
            y \textit{CollisionManager}.
            \item Se destruyen todos los objetos de la clase \textit{Enemy}.
            \item Se eliminan los enlaces entre los \textit{spell.body} y
            \textit{CollisionManager}.
            \item Se destruyen todos los objetos de la clase \textit{Spell}.
            \item Se destruye el objeto \textit{gameStats}.
            \item Se destruye el objeto \textit{stateGame}.\\
        \end{itemize}
\end{description}


\textbf{Caso de uso: Pausar (escenario principal)}

\figura{sequence-pausa1.jpg}{scale=0.60}{Diagrama de secuencia: Pausa (escenario principal)}{fig:sequence-pausa1}{h}

\begin{description}
    \itemsep0em
    \item [Operación] PausarJuego()
    \item [Actores] \jugador, \sistema.
    \item [Responsabilidades] interrumpe la partida y muestra el menú
    de pausa.
    \item [Precondiciones]$\quad$
        \begin{itemize}
            \itemsep0em
            \item Existe un objeto \textit{stateGame} y se está jugando
            una partida.
        \end{itemize}
    \item [Postcondiciones] ninguna.\\
\end{description}

\begin{description}
    \itemsep0em
    \item [Operación] ReanudarJuego()
    \item [Actores] \jugador, \sistema.
    \item [Responsabilidades] oculta el menú de pausa y continúa la partida.
    \item [Precondiciones]$\quad$
        \begin{itemize}
            \itemsep0em
            \item Existe un objeto \textit{stateGame} y el juego estaba pausado.
        \end{itemize}
    \item [Postcondiciones] ninguna.\\
\end{description}

\textbf{Caso de uso: Pausar (escenario 3a)}

\figura{sequence-pausa2.jpg}{scale=0.60}{Diagrama de secuencia: Pausa (escenario 3a)}{fig:sequence-pausa2}{h}

\begin{description}
    \itemsep0em
    \item [Operación] PausaSeleccionNivel()
    \item [Actores] \jugador, \sistema.
    \item [Responsabilidades] oculta el menú de pausa, destruye el nivel y
    regresa a la pantalla de selección de nivel.
    \item [Precondiciones]$\quad$
        \begin{itemize}
            \itemsep0em
            \item Existe un objeto \textit{stateGame} y el juego estaba pausado.
        \end{itemize}
    \item [Postcondiciones]$\quad$
        \begin{itemize}
            \itemsep0em
            \item Se destruye el objeto \textit{navigationMesh}.
            \item Se destruye el objeto \textit{song}.
            \item Se eliminan el enlace entre \textit{player.body} y \textit{CollisionManager}.
            \item Se destruye el objeto \textit{player}.
            \item Se eliminan los enlaces entre los \textit{enemy.body}
            y \textit{CollisionManager}.
            \item Se destruyen todos los objetos de la clase \textit{Enemy}.
            \item Se eliminan los enlaces entre los \textit{spell.body} y
            \textit{CollisionManager}.
            \item Se destruyen todos los objetos de la clase \textit{Spell}.
            \item Se destruye el objeto \textit{gameStats}.
            \item Se destruye el objeto \textit{stateGame}.\\
        \end{itemize}
\end{description}

\textbf{Caso de uso: Pausar (escenario 3b)}

\figura{sequence-pausa3.jpg}{scale=0.60}{Diagrama de secuencia: Pausa (escenario 3b)}{fig:sequence-pausa3}{h}

\begin{description}
    \itemsep0em
    \item [Operación] PausaVolverMenu()
    \item [Actores] \jugador, \sistema.
    \item [Responsabilidades] oculta el menú de pausa, destruye el nivel y
    regresa al menú principal del juego.
    \item [Precondiciones]$\quad$
        \begin{itemize}
            \itemsep0em
            \item Existe un objeto \textit{stateGame} y el juego estaba pausado.
        \end{itemize}
    \item [Postcondiciones]$\quad$
        \begin{itemize}
            \itemsep0em
            \item Se destruye el objeto \textit{navigationMesh}.
            \item Se destruye el objeto \textit{song}.
            \item Se eliminan el enlace entre \textit{player.body} y \textit{CollisionManager}.
            \item Se destruye el objeto \textit{player}.
            \item Se eliminan los enlaces entre los \textit{enemy.body}
            y \textit{CollisionManager}.
            \item Se destruyen todos los objetos de la clase \textit{Enemy}.
            \item Se eliminan los enlaces entre los \textit{spell.body} y
            \textit{CollisionManager}.
            \item Se destruyen todos los objetos de la clase \textit{Spell}.
            \item Se destruye el objeto \textit{gameStats}.
            \item Se destruye el objeto \textit{stateGame}.\\
        \end{itemize}
\end{description}


\textbf{Caso de uso: Lanzar hechizo (escenario principal)}

\figura{sequence-hechizo1.jpg}{scale=0.60}{Diagrama de secuencia: Lanzar hechizo (escenario principal)}{fig:sequence-hechizo1}{h}

\begin{description}
    \itemsep0em
    \item [Operación] SeleccionarHechizo(hechizo)
    \item [Actores] \jugador, \sistema.
    \item [Responsabilidades] marca un hechizo como seleccionado.
    \item [Precondiciones] $\quad$
        \begin{itemize}
            \itemsep0em
            \item Existe un objeto \textit{stateGame} de la clase \textit{StateGame}
            y se está jugando una partida.
            \item Existe un objeto \textit{player} de la clase \textit{Player}.
        \end{itemize}
    \item [Postcondiciones] $\quad$
        \begin{itemize}
            \itemsep0em
            \item Modificación de atributo, \textit{player.selectedSpell = hechizo}.\\
        \end{itemize}
\end{description}

\begin{description}
    \itemsep0em
    \item [Operación] LanzarHechizo(direccion)
    \item [Actores] \jugador, \sistema.
    \item [Responsabilidades] hace que el personaje lance el hechizo seleccionado
    hacia una dirección determinada.
    \item [Precondiciones] $\quad$
        \begin{itemize}
            \itemsep0em
            \item Existe un objeto \textit{stateGame} de la clase \textit{StateGame}
            y se está jugando una partida.
            \item Existe un objeto \textit{player} de la clase \textit{Player}.
        \end{itemize}
    \item [Postcondiciones] $\quad$
        \begin{itemize}
            \itemsep0em
            \item Se crea un objeto \textit{spell} de la clase \textit{Spell}.
            \item Atributos: \textit{spell.type = player.selectedSpell}, 
            el resto de atributos del hechizo se determinan según el tipo
            seleccionado.
            \item Modificación de atributo: \textit{spell.direction = direccion}.\\
        \end{itemize}
\end{description}



\textbf{Caso de uso: Mover personaje (escenario principal)}

\figura{sequence-pnj1.jpg}{scale=0.60}{Diagrama de secuencia: Mover personaje (escenario principal)}{fig:sequence-pnj1}{h}

\begin{description}
    \itemsep0em
    \item [Operación] MoverPersonaje(teclas, camara)
    \item [Actores] \jugador, \sistema.
    \item [Responsabilidades] mueve al personaje por el escenario siguiendo
    las teclas y en función de la cámara.
    \item [Precondiciones] $\quad$
        \begin{itemize}
            \itemsep0em
            \item Existe un objeto \textit{stateGame} de la clase \textit{StateGame}
            y se está jugando una partida.
            \item Existe un objeto \textit{player} de la clase \textit{Player}.
        \end{itemize}
    \item [Postcondiciones] $\quad$
        \begin{itemize}
            \itemsep0em
            \item Modificación de atributos: \textit{player.position} y
            \textit{player.direction} en función de la tecla pulsada y la
            posición relativa de la cámara.\\
        \end{itemize}
\end{description}



\textbf{Caso de uso: Mover cámara (escenario principal)}

\figura{sequence-camara1.jpg}{scale=0.60}{Diagrama de secuencia: Mover camara (escenario principal)}{fig:sequence-camara1}{h}

\begin{description}
    \itemsep0em
    \item [Operación] MoverCamara(raton)
    \item [Actores] \jugador, \sistema.
    \item [Responsabilidades] mueve la cámara alrededor del personaje.
    \item [Precondiciones] $\quad$
        \begin{itemize}
            \itemsep0em
            \item Existe un objeto \textit{stateGame} de la clase \textit{StateGame}
            y se está jugando una partida.
            \item Existe un objeto \textit{player} de la clase \textit{Player}.
        \end{itemize}
    \item [Postcondiciones] $\quad$
        \begin{itemize}
            \itemsep0em
            \item Modificación de atributos: \textit{camera.position} y
            \textit{camera.orientation} en función del gesto del ratón
            y de \textit{player.position}.\\
        \end{itemize}
\end{description}


\textbf{Caso de uso: Victoria (escenario principal)}

\figura{sequence-victoria1.jpg}{scale=0.60}{Diagrama de secuencia: Victoria (escenario principal)}{fig:sequence-victoria1}{h}

\begin{description}
    \itemsep0em
    \item [Operación] PantallaVictoria()
    \item [Actores] \jugador, \sistema.
    \item [Responsabilidades] carga y muestra la pantalla de victoria.
    \item [Precondiciones] $\quad$
        \begin{itemize}
            \itemsep0em
            \item Existe un objeto \textit{gameStats} de la clase \textit{GameStats}
            con las estadísticas de juego de la partida anterior.
            y se está jugando una partida.
            \item Existe un objeto \textit{profile} de la clase \textit{Profile}
            correspondiente al perfil seleccionado.
        \end{itemize}
    \item [Postcondiciones] $\quad$
        \begin{itemize}
            \itemsep0em
            \item Creación del objeto \textit{stateVictory} de la clase \textit{StateVictory}.
            \item Creación del objeto \textit{song} de la clase \textit{Song}.
            \item Modificación de atributos: \textit{player.experiencie} y
            \textit{player.unlockedLevel} en función del objeto \textit{gameStats} y
            del nivel que se acabe de jugar.\\
        \end{itemize}
\end{description}

\begin{description}
    \itemsep0em
    \item [Operación] SeleccionarNivel()
    \item [Actores] \jugador, \sistema.
    \item [Responsabilidades] destruye la pantalla de victoria y vuelve
    a la de selección de nivel.
    \item [Precondiciones] $\quad$
        \begin{itemize}
            \itemsep0em
            \item Existe un objeto \textit{stateVictory} de la clase \textit{StateVictory}.
        \end{itemize}
    \item [Postcondiciones] $\quad$
        \begin{itemize}
            \itemsep0em
            \item Destrucción del objeto \textit{song}.
            \item Destrucción del objeto \textit{stateVictory}.\\
        \end{itemize}
\end{description}


\textbf{Caso de uso: Victoria (escenario 3a)}

\figura{sequence-victoria2.jpg}{scale=0.60}{Diagrama de secuencia: Victoria (escenario 3a)}{fig:sequence-victoria2}{h}

\begin{description}
    \itemsep0em
    \item [Operación] SeleccionarReintentar()
    \item [Actores] \jugador, \sistema.
    \item [Responsabilidades] destruye la pantalla de victoria y vuelve
    a jugar al mismo nivel en estado de juego.
    \item [Precondiciones] $\quad$
        \begin{itemize}
            \itemsep0em
            \item Existe un objeto \textit{stateVictory} de la clase \textit{StateVictory}.
        \end{itemize}
    \item [Postcondiciones] $\quad$
        \begin{itemize}
            \itemsep0em
            \item Destrucción del objeto \textit{song}.
            \item Destrucción del objeto \textit{stateVictory}.\\
        \end{itemize}
\end{description}


\textbf{Caso de uso: Victoria (escenario 3b)}

\figura{sequence-victoria3.jpg}{scale=0.60}{Diagrama de secuencia: Victoria (escenario 3b)}{fig:sequence-victoria3}{h}

\begin{description}
    \itemsep0em
    \item [Operación] SeleccionarMenu()
    \item [Actores] \jugador, \sistema.
    \item [Responsabilidades] destruye la pantalla de victoria y vuelve
    al menú principal.
    \item [Precondiciones] $\quad$
        \begin{itemize}
            \itemsep0em
            \item Existe un objeto \textit{stateVictory} de la clase \textit{StateVictory}.
        \end{itemize}
    \item [Postcondiciones] $\quad$
        \begin{itemize}
            \itemsep0em
            \item Destrucción del objeto \textit{song}.
            \item Destrucción del objeto \textit{stateVictory}.\\
        \end{itemize}
\end{description}


\textbf{Caso de uso: Créditos (escenario principal)}

\figura{sequence-creditos.jpg}{scale=0.60}{Diagrama de secuencia: Créditos (escenario principal)}{fig:sequence-creditos}{h}

\begin{description}
    \itemsep0em
    \item [Operación] PantallaCreditos()
    \item [Actores] \jugador, \sistema.
    \item [Responsabilidades] crea y muestra la pantalla de créditos.
    \item [Precondiciones] ninguna.
    \item [Postcondiciones] $\quad$
        \begin{itemize}
            \itemsep0em
            \item Creación del objeto \textit{stateCredits} de la clase \textit{StateCredits}.
            \item Creación del objeto \textit{song} de la clase \textit{Song}.\\
        \end{itemize}
\end{description}

\begin{description}
    \itemsep0em
    \item [Operación] SeleccionarMenu()
    \item [Actores] \jugador, \sistema.
    \item [Responsabilidades] destruye la pantalla de créditos y vuelve al
    menú principal.
    \item [Precondiciones] $\quad$
        \begin{itemize}
            \itemsep0em
            \item Existe un objeto \textit{stateCredits} de la clase \textit{StateCredits}.
        \end{itemize}
    \item [Postcondiciones] $\quad$
        \begin{itemize}
            \itemsep0em
            \item Destrucción del objeto \textit{song}.
            \item Destrucción del objeto \textit{stateCredits}.\\
        \end{itemize}
\end{description}





\section{Diseño}
\label{sec:siontower-diseno}

Al igual que se hizo en la fase de análisis de \juego\ (sección
\ref{siontower-analisis}), durante la fase de diseño también emplearemos
la notación \textit{UML}. Nos centraremos en cómo hace el sistema lo que
debe hacer para cumplir sus requisitos. Por supuesto, dejaremos espacio
para variaciones en la fase de implementación.\\

Mostraremos el diagrama de clases de diseño dividido por
subsistemas para una mayor claridad. Acompañaremos cada figura de una pequeña
explicación sobre el subsistema que ilustra.\\

\subsection{Diagrama de clases de diseño}

Como hemos mencionado anteriormente, separaremos los diagramas de clases
de diseño para obtener una mayor claridad. Dicha separación se basará
en el cometido de cada subsistema. El primer diagrama (ver figura
\ref{fig:clasesdiseno1}) corresponde a las clases encargadas de iniciar
el juego, controlar su funcionamiento general y proporcionar soporte a
otras clases.\\

\figura{clasesdiseno1.jpg}{scale=0.40}{Diagrama de clases de diseño: sistema general}{fig:clasesdiseno1}{H}

El segundo diagrama (ver figura \ref{fig:clasesdiseno3}) representa la gestión
de estados de \juego. La clase \textit{StateManager} controla los estados
y las transiciones entre ellos mientras que cada clase descendiente de
\textit{State} representa cada una de las pantallas de juego. Así tenemos
\textit{StateMenu}, \textit{StateProfile}, \textit{StateLevel}, \textit{StateGame},
\textit{StateVictory}, y \textit{StateCredits}. Los estados utilizan una
melodía (\textit{Song}) y reproducen efectos de sonido \textit{SoundFX}.
El estado de selección de perfiles elige un perfil utilizando el gestor
(\textit{ProfileManager}) mientras que el selector de niveles consulta
cuál es el perfil seleccionado. Así mismo, escoge un nivel entre los
disponibles a través del gestor de niveles \textit{LevelManager}. Durante
el juego se van actualizando las estadísticas de juego \textit{GameStats}
y en la pantalla de victoria se consultan para actualizar el perfil seleccionado.\\

\figura{clasesdiseno3.jpg}{scale=0.40}{Diagrama de clases de diseño: sistema de estados}{fig:clasesdiseno3}{H}

En el tercer diagrama (ver figura \ref{fig:clasesdiseno2}) se
muestran las clases relacionadas con el sistema de juego. Podemos
observar la jerarquía de herencia de los objetos de juego (\textit{GameObject})
o el uso de la malla de navegación entre otros (\textit{NavigationMesh} y
\textit{Cell}).\\

\figura{clasesdiseno2.jpg}{scale=0.40}{Diagrama de clases de diseño: sistema de juego}{fig:clasesdiseno2}{H}

En el cuarto y último diagrama (ver figura \ref{fig:clasesdiseno4} se muestra
el sistema de inteligencia artificial y algoritmos de movimiento. La clase
\textit{SteeringBehaviour} modela los algoritmos de movimiento de forma
genérica y de ella heredan implementaciones concretas. Estos algoritmos
modifican el aspecto dinámico de los personajes aplicando fuerzas y aceleraciones,
veremos más sobre este tema en el punto \ref{sec:steering} en la página
\pageref{sec:steering}.\\

\figura{clasesdiseno4.jpg}{scale=0.5}{Diagrama de clases de diseño: Steering Behaviors}{fig:clasesdiseno4}{H}

\section{Implementación}

A lo largo de toda la fase de implementación se han ido encontrando diversos
obstáculos en distintos subsistemas. Estos han surgido bien por desconocimiento
de la materia o por la dificultad que entraña la misma. En cualquier caso, 
en este capítulo haremos un repaso por los detalles más interesantes de
la implementación de \juego. En cada uno de ellos se expondrá el problema
a resolver, las dificultades encontradas y la solución propuesta adjuntando
si es necesario pequeños fragmentos de código.\\

Para consultar el código fuente completo del juego, lo mejor es acudir
al repositorio \textit{Subversion} de la forja de RedIRIS en la siguiente
dirección.\\

\url{https://forja.rediris.es/scm/?group_id=820}\\

La documentación del código generada con \textit{Doxygen} \cite{website:doxygen}
facilitará en gran medida la lectura del código. Dicha documentación
complementaria puede ser accedida desde la siguiente dirección web.\\

\url{http://siondream.com/siontower-doxygen}\\

\subsection{Gestión de estados de juego}

% Clase State
\subsubsection{Estados de juego, clase State}

Como hemos visto anteriormente, en \juego\ contamos con varias pantallas
entre menús y el estado de juego. Cada una de estas pantallas está modelada
por una clase hija de \textit{State}. Se trata de una clase virtual pura
que incluye un método de actualización \textit{update()} y varios manejadores
de eventos siguiendo el esquema de la biblioteca \textsc{OIS} \cite{website:ois} (pulsar
tecla, liberar tecla, pulsar botón del ratón, liberar botón del ratón 
y mover ratón). El comportamiento por defecto de los manejadores de eventos
consiste en no hacer nada. Si los sobrecargamos en una clase hija, podremos
indicar la respuesta deseada.\\

Los estados pueden contar con todos sus elementos cargados en memoria (estado
cargado) o pueden estar creados pero no listos para su uso. Eso permite
evitar crear y destruir nuevos estados constantemente, simplemente llamaríamos
a sus métodos \textit{load()} y \textit{clear()} según sea necesario.\\

La interfaz del juego utiliza la biblioteca \textsc{MyGUI} \cite{website:mygui}.
Esta biblioteca se basa en unos ficheros \textit{.layout}, en el fondo
son unos XML sencillos para definir los elementos de la interfaz (botones,
paneles, etiquetas, etc). Cada estado debe traducir toda su interfaz
empleando \textsc{gettext} \cite{website:gettext} a través del método
\textit{translate()}. Hablaremos de la internacionalización del proyecto
en la sección \ref{siontower-internacionalizacion} (página \pageref{siontower-internacionalizacion})\\

El método \textit{adjustFontHeight()} se utiliza para establecer el tamaño
de la letra de los elementos de la interfaz en función de la resolución
de la ventana. Recordemos que \juego\ es independiente de la resolución
(siempre y cuando esta tenga una relación de aspecto 16:10).\\

A continuación se muestra la definición de la clase \textit{State}:\\

\lstinputlisting[style=C++]{codigo/state.h}

% Clase StateManager, transiciones entre estados
\subsubsection{Pila de estados, clase StateManager}

La clase \textit{StateManager} se encarga de gestionar los estados y las
transiciones entre los mismos. Para ello cuenta con una pila de estados
interna de forma que podemos añadir un estado encima de la pila mediante
una operación \textit{push()} o sacarlo mediante \textit{pop()}. Esto
resulta muy útil cuando avanzamos o retrocedemos de forma lineal por varios
menús y nos evita tener que estar creando y destruyendo estados constantemente.
Además, podemos eliminar todos los estados de la pila utilizando \textit{popAllStates()}
o cambiar el tope de la pila por otro estado con \textit{changeState()}.
La figura \ref{fig:statestack} ilustra el proceso.\\

\figura{statestack.png}{scale=0.7}{Pila de estados}{fig:statestack}{h}

\textit{StateManager} hereda de las clases \textit{FrameListener},
\textit{WindowEventListener}, \textit{KeyListener} y \textit{MouseListener}.
De esta forma, el gestor de estados sigue el patrón de diseño \textit{Observer} \cite{gamm77}
para diversos eventos: etapas del renderizado, cambios en la ventana,
interacción con el teclado y con el ratón respectivamente. El gestor de estado
delega en el estado activo los eventos de teclado y ratón para que éste último
les de respuesta.\\

Cuando un estado detecta que se ha de producir un cambio de estado, informa
al \textit{StateManager} de ello. No obstante, no se puede destruir el estado
actual para cambiar a otro porque lo normal es que nos encontremos en mitad
de una iteración del bucle de juego (game loop) \cite{greg09}. Podría
ocurrir que destruyésemos el estado actual a la vez que se esta ejecutando su
método \textit{update()} lo que provocaria accesos de memoria a basura.
El gestor de estados almacena una pila de operaciones pendientes de forma
que cuando solicitamos una operación \textit{pop()} o \textit{push()} no
se realiza en el mismo instante. Una vez finalizada la iteración del bucle
de juego actual, se procede a realizar las operaciones pendientes, esta vez
con total seguridad (método privado \textit{perfomOperations()}).\\

Cuando se llama al método \textit{start()} se inicia el bucle de renderizado
gestionado por el motor \textsc{Ogre3D}. Antes de renderizar la escena,
se dispara el evento \textit{frameStarted} y el gestor de estados actualiza
en orden descendente la pila de estados activos.\\

A continuación se muestra la definición de la clase \textit{StateManager}:\\

\lstinputlisting[style=C++]{codigo/stateManager.h}

\subsection{Internacionalización mediante gettext}
\label{siontower-internacionalizacion}

% Introducción

\juego\ está completamente internacionalizado y se distribuye tanto en inglés
como en castellano aunque no es complicado añadir idiomas adicionales. Se
ha seguido el sistema propuesto por la biblioteca de localización libre
\textsc{gettext} \cite{website:gettext}, que es prácticamente un estándar en la materia. Para
comprender el uso de la biblioteca así como la generación y mantenimiento
de traducciones se recurrió a la publicación de José Tomás Tocino García
\textit{Traducción de proyectos con GNU gettext en 15 minutos} \cite{pdf:jtgettext}.\\

% Ficheros
\subsubsection{Ficheros necesarios}

\textsc{gettext} funciona de una manera muy sencilla. Trabaja a modo de diccionario
clave valor en el que las claves son las cadenas a traducir en un idioma base
(normalmente el inglés) y los valores son el texto traducido al idioma destino.
Estas duplas claves valor se especifican en ficheros de texto plano de extensión
\texttt{.po}. No obstante, \textsc{gettext} utiliza internamente una versión
binaria de dichos ficheros de extensión \texttt{.mo}. El proceso de traducción
consta de los siguientes pasos:

\begin{enumerate}
    \itemsep0em
    \item Creación de la jerarquía de directorios para las traducciones.
    \item Obtención de todas las cadenas a partir del código fuente y agrupación
    de las mismas en un fichero maestro \texttt{.pot}.
    \item Creación de un fichero \texttt{.po} a partir del maestro \texttt{.pot}
    por cada idioma. Para ello empleamos el comando \texttt{msginit}.
    \item Traducción de las cadenas del fichero \texttt{.po}.
    \item Binarización del fichero \texttt{.po} en uno \texttt{.mo}. Utilizaremos
    el comando \texttt{msgfmt}.
\end{enumerate}

La jerarquía de directorios resultante en \juego\ es la siguiente:\\

\begin{verbatim}
|-- [siontower]
|    |-- lang
|    |   |-- en
|    |   |   `-- LC_MESSAGES
|    |   |       `-- siontower.mo
|    |   `-- es
|    |       `-- LC_MESSAGES
|    |           `-- siontower.mo
|    |-- po
|    |   |-- en.po
|    |   |-- es.po
|    |   `-- siontower.pot
|
\end{verbatim}

En el siguiente fragmento de código se puede apreciar un ejemplo
de fichero \textit{.po}:\\

\begin{verbatim}
#: src/stateProfile.cpp:202
msgid "#ff0000Error: the profile already exists"
msgstr "#ff0000Error: el perfil ya existe"

#: src/stateProfile.cpp:195
msgid "#ff0000Error: you must enter a name"
msgstr "#ff0000Error: debes introducir un nombre"

#: src/stateLevel.cpp:269
msgid "#ff0000Locked level"
msgstr "#ff0000Nivel bloqueado"

...
\end{verbatim}

% Traducción de mensajes dentro del código
\subsubsection{Traducción de mensajes en el código}

Para traducir mensajes dentro del código hemos de incluir los ficheros de
cabecera \textit{<locale.h>} y \textit{<libintl.h>}. Al comienzo del programa
hemos de indicar al sistema qué codificación de caracteres emplearemos,
qué paquetes de traducciones utilizaremos (en nuestro
caso sería \textit{siontower}) y dónde se encontrarán las mismas (para nosotros
sería el directorio \texttt{lang}). Bastan las siguientes líneas:\\

\begin{lstlisting}[style=C++]
bind_textdomain_codeset("siontower", "UTF-8");
setlocale(LC_MESSAGES, "");
bindtextdomain("siontower", "lang" );
textdomain("siontower");
\end{lstlisting}

En cada módulo que deseemos traducir tendremos que incluir los mismos ficheros
de cabecera. En cada cadena a localizar habrá que colocar una llamada a la
función \texttt{gettext()} con la clave en el idioma base para que nos
devuelva la cadena en el idioma del sistema que esté ejecutando el software. 
Lo normal es utilizar la directiva del preprocesador \texttt{\#define \_ gettext}
para ahorrar espacio y limpiar el código. De esta manera lo que antes era:\\

\begin{lstlisting}[style=C++]
_lblState->setCaption("You have failed! Press space to try again!");
\end{lstlisting}

Ahora se convierte en:\\

\begin{lstlisting}[style=C++]
_lblState->setCaption(_("You have failed! Press space to try again!"));
\end{lstlisting}

\subsubsection{Traducción de plantillas de MyGUI}

Si ejecutamos \juego\ en un sistema cuyo idioma sea el inglés veremos
\textit{You have failed! Press space to try again!}. En cambio, si lo hacemos
en uno cuyo idioma sea el castellano podremos leer \textit{¡Has fallado!
Pulsa espacio para intentarlo de nuevo}. Esto funciona dentro de código
\textit{C++} en el juego pero no con las plantillas de la interfaz.
\textsc{MyGUI} utiliza ficheros XML de extensión \texttt{.layout} para definir
los widgets que tiene cada pantalla de menú. En el mismo fichero XML
aparecen los textos de botones, etiquetas y otros elementos que no se
mostrarán traducidos en la interfaz. Para solucionarlo hay que llevar
a cabo dos tareas:

\begin{enumerate}
    \itemsep0em
    \item Extraer todas las cadenas traducibles a un fichero \texttt{.pot}.
    \item En tiempo de ejecución indicarle a \textsc{gettext} que traduzca
    todas las cadenas de los elementos de la interfaz.
\end{enumerate}

\textsc{gettext} es capaz de extraer las cadenas traducibles de forma automática
a partir de código \textit{C++} o \textit{Python}, no obstante, desconoce
el formato xml con la sintaxis de \textsc{MyGUI} por lo que no puede llevar
a cabo dicha extracción. Para subsanar este problema, se ha implementado un
pequeño script en \textit{Python} que escanea un directorio en busca de
ficheros \texttt{.layout}, encuentra las cadenas traducibles y las una
en una plantilla \texttt{.pot}. Su sintaxis es la siguiente:\\

\texttt{python translateLayout.py layoutsDir file.pot}\\

El código completo del script es el siguiente:\\

\lstinputlisting[style=Python]{codigo/translateLayout.py}

Para solucionar el punto 2 tenemos que recordar que cada estado de juego
(\textit{State}) contaba con un método \textit{translate()} cuyo objetivo
era traducir todos los elementos de la interfaz. Las claves de las cadenas
que buscamos son las propias cadenas en el idioma base por lo que la traducción
resulta trivial (pero necesaria). Un ejemplo de método de traducción podría
ser el siguiente:\\

\begin{lstlisting}[style=C++]
void StateMenu::translate() {
    _btnPlay->setCaption(_(_btnPlay->getCaption().asUTF8_c_str()));
    _btnCredits->setCaption(_(_btnCredits->getCaption().asUTF8_c_str()));
    _btnExit->setCaption(_(_btnExit->getCaption().asUTF8_c_str()));
    _lblSubtitle->setCaption(_(_lblSubtitle->getCaption().asUTF8_c_str()));
}
\end{lstlisting}


\subsection{Sistema de audio}

% Ogre no audio, usamos SDL mixer. Sistema de gestión de recursos de Ogre.
\textsc{Ogre3D} es simplemente un motor de renderizado y carece de subsistema
de audio, de detección de colisiones, gestión de entrada o juego en red.
Que en \juego\ se reprodujesen efectos de sonido y música de fondo era
imprescindible para la inmersión y para ofrecer información complementaria
a la visual de cara al jugador sobre lo que ocurre en el mundo que simulamos.
Para resolver este problema se ha decidido emplear la biblioteca libre
\textit{Simple DirectMedia Layer} (\textsc{libSDL}) y su extensión
relacionada con el audio \textsc{libSDL mixer} \cite{website:sdl}.\\

\textsc{libSDL} es una biblioteca compatible con el lenguaje \textit{C} y,
por tanto, carece de orientación a objetos. Parte del valor del trabajo
realizado para \juego\ ha sido abstraer el bajo nivel del subsistema de audio
de la biblioteca en un sistema orientado a objetos.\\

\subsubsection{Gestión de recursos en Ogre3D}

\figura{resources-cycle.jpg}{scale=0.4, angle=90}{Ciclo de vida de los recursos en Ogre3D}{fig:resources-cycle}{H}

\textsc{Ogre3D} cuenta con un sistema de gestión de recursos muy completo
para gestionar el ciclo de vida de cada recurso (ver figura \ref{fig:resources-cycle})
y así optimizar el consumo de memoria del juego. El concepto recurso es
genérico, podemos referirnos a una malla tridimensional, a un script de
postprocesado, un conjunto de animaciones, una textura o un recurso
definido por nosotros. El carácter configurable y extensible de \textsc{Ogre3D}
permite al usuario de la biblioteca crear y gestionar nuevos tipos
de recursos de forma idéntica a como se hace con los propios del sistema.
En nuestro caso definiremos los recursos \textit{Song} (pista de música)
y \textit{SoundFX} (efecto de sonido). Esto nos reportará las siguientes
ventajas:

\begin{itemize}
    \itemsep0em
    \item Cada recurso sólo se instanciará una vez en memoria aunque se utilice
    por varias entidades con el consiguiente ahorro.
    \item Sencillo acceso al recurso sin importar su ruta dentro del sistema
    de ficheros.
    \item Gestión de su ciclo de vida de cara a optimizar el consumo de memoria
    y los tiempos de carga.
\end{itemize}

% Esquema para extender la gestión de recursos de Ogre3D
\subsubsection{Extensión de la gestión de recursos de Ogre3D}

Por cada recurso nuevo que deseemos integrar en \textsc{Ogre3D} tendremos
que crear una clase hija de \textit{Ogre::Resource}, una derivada de la
clase paramétrica \textit{Ogre::Shared\_ptr} (para asegurarnos instancia
única del recurso) y otra descendiente de \textit{Ogre::ResourceManager}
que gestionará la carga y destrucción de recursos del mismo tipo. Cada
una de estas clases deberá implementar varios métodos de forma obligatoria.
El esquema general se muestra en la figura \ref{fig:extender-recursos}.
Para conocer detalles adicionales a los expuestos en esta sección, puede
consultarse el artículo \textit{Extender la gestión de recursos, audio}
en \wiki\ \cite{website:recursos-iberogre}.\\

\figura{extender-recursos.jpg}{scale=0.40}{Esquema para extender la gestión de recursos en Ogre3D}{fig:extender-recursos}{h}

% Song, SongPtr y SongManager
\subsubsection{Song, SongPtr y SongManager}

La clase \textit{Song} hereda de \textit{Ogre::Resource} y se encarga de
reproducir pistas de música en formato \textit{OGG}. En su constructor es
necesario indicarle el gestor de recursos que la controla su nombre y grupo
al que pertenece. Donde realmente se carga el recurso y se libera son en
los métodos privados \textit{loadImpl()} y \textit{unloadImpl()} llamados
por el gestor de recursos. Así mismo, cuenta con métodos para la reproducción,
pausa, fundido de entrada o de salida. Debe proporcionar un método
\textit{calculateSize()} porque \textsc{Ogre3D} debe conocer en todo momento
cuánto ocupan sus recursos en memoria. Incluso podríamos asignarle un presupuesto
en memoria RAM a cualquiera de los gestores de recursos. A continuación
se muestra la definición de la clase.\\

\lstinputlisting[style=C++]{codigo/song.h}

\textit{SongPtr} es una clase sencilla que se limita a heredar de
\textit{Ogre::Shared\_ptr} y permite asegurarnos de que sólo habrá una instancia
de cada canción en todo el sistema. Funciona de manera similar a los
\textit{shared\_ptr} de la biblioteca \textsc{Boost} \cite{website:boost}.\\

\lstinputlisting[style=C++]{codigo/songPtr.h}

La clase \textit{SongManager} es el gestor de recursos que se encarga de manejar
pistas de audio (\textit{Song}). Sigue el patrón de diseño \textit{Singleton}
(una sóla instancia accesible desde todo el sistema)
\cite{gamm77} y cuenta con los clásicos métodos \textit{getSingleton()} y
\textit{getSingletonPtr()}. Para cargar un recurso utilizaremos \textit{load}
el cual busca el recurso y, en el caso de no existir, lo crea con su método
privado \textit{createImpl()}.

\lstinputlisting[style=C++]{codigo/songManager.h}

% SoundFX, SoundFXPtr y SoundFXManager
\subsubsection{SoundFX, SoundFXPtr y SoundFXManager}

Las clases \textit{SoundFX}, \textit{SoundFXPtr} y \textit{SoundFXManager}
funcionan de forma prácticamente idéntica a como lo hacen sus semejantes
de música.\\

Definición de la clase \textit{SoundFX}:\\

\lstinputlisting[style=C++]{codigo/soundFX.h}

Definición de la clase \textit{SoundFXPtr}:\\

\lstinputlisting[style=C++]{codigo/soundFXPtr.h}

Definición de la clase \textit{SoundFXManager}:\\

\lstinputlisting[style=C++]{codigo/soundFXManager.h}

Para detalles adicionales sobre la implementación del subsistema, lo ideal
es acudir al propio código fuente en la forja de RedIRIS.\\

% Ejemplo de uso
\subsubsection{Ejemplo de uso}

A continuación se adjunta un pequeño ejemplo de uso del sistema de audio
desarrollado para \juego.

\begin{lstlisting}[style=C++]
// Durante el inicio de la aplicacion

// Creamos el ResourceManager
SongManager* songManager = new SongManager();
SoundFXManager* soundFXManager = new SoundFXManager();

...

// Cargamos los recursos
SongPtr levelMusic = songManager->load("musicaNivel1.ogg", "Nivel1");
levelMusic->play();

SoundFXPtr explosion = soundFXManager->load("explosion.wav", "Nivel1");
explosion->play();

...

// Durante el cierre de la aplicacion

// Destruimos el ResourceManager
delete songManager;
delete soundFXManager;
\end{lstlisting}

Como ya se mencionó en los objetivos del proyecto (sección \ref{sec:objetivos} en la
página \pageref{sec:objetivos}), se pretende que los subsistemas desarrollados para
\juego\ sean reutilizables en alta medida. Por ello, se ha publicado
el sistema de audio por separado como paquete descargable en la forja de
RedIRIS. Viene acompañado de la documentación generada con \textit{Doxygen},
de las instrucciones de integración y de su licencia \textit{GPL v3}. Puede
obtenerse en la siguiente dirección.\\

\url{http://forja.rediris.es/frs/download.php/2075/siontower-3dsound-v0.1.tar.gz}\\

\subsection{Detección de colisiones}

En \juego\ es necesario detectar colisiones entre muchos de los elementos
del juego y \textsc{Ogre3D} no proporciona un subsistema que nos ayude
a llevar a cabo esta tarea. Era posible acudir a alternativas como los motores
de físicas \textsc{Bullet} y \textsc{ODE} pero eran demasiado complejos
para el reducido número de requisitos que teníamos. En \juego\ había
que detectar colisiones entre los siguientes elementos.\\

\begin{itemize}
    \itemsep0em
    \item personaje-enemigo
    \item personaje-escenario
    \item enemigo-hechizo
    \item hechizo-escenario
\end{itemize}

El área de colisión de los personajes y escenario vendría limitado por varias
formas geométricas agrupadas. Entre estas formas podrían encontrarse: cajas,
esferas y planos. Las funcionalidades que requiere el sistema de colisiones
son:\\

\begin{itemize}
    \itemsep0em
    \item Soporte para varias formas (clase \textit{Shape}) como esferas,
    planos, AABB y OBB.
    \item Tests de colisión para varias combinaciones de estas formas.
    \item Cuerpos (clase \textit{Body}) compuestos por varias formas colisionables
    de manera que se ajusten a contornos complejos.
    \item Gestor de colisiones que mantenga el control de los cuerpos existentes
    y pueda detectar colisiones entre ellos.
    \item Detección de colisiones eficiente.
    \item Filtrado de colisiones por tipo de cuerpo. Sólo nos interesan
    las especificadas en la lista anterior.
    \item Registro de funciones a disparar ante colisiones entre dos cuerpos
    de un tipo de terminado, también conocidas como \textit{callbacks}.
\end{itemize}

% Diseño general
\subsubsection{Diseño general}

Como puede verse en los diagramas de la fase de diseño en la sección
\ref{sec:siontower-diseno} en la página \pageref{sec:siontower-diseno},
el sistema de detección de colisiones está formado por:\\

\begin{description}
    \item [Shape] clase virtual que modela una forma geométrica en el espacio
    de forma genérica. De ella descienden \textit{Plane}, \textit{Sphere}, \textit{AxisAlignedBox} y
    \textit{OrientedBox}.
    \item [Body] modela un cuerpo colisionable formado por varias formas geométricas.
    \item [GameObject] agrupa un cuerpo colisionable (\textit{Body}) y un
    nodo de la escena 3D (\textit{Ogre::SceneNode}). De esta forma encapsulamos
    la faceta colisionable y visual de los objetos de juego. Posteriormente,
    los elementos que cuenten con una malla tridimensional pueden heredar
    de esta clase y añadir un \textit{Ogre::Entity} como ocurre
    en la clase \textit{GameMesh}. En cambio, si lo que desean es mostrar un
    sistema de partículas con modelo de colisión pueden heredar y añadir
    un \textit{Ogre::ParticleSystem} como ocurre en \textit{Spell}.
    \item [CollisionManager] controla todos los cuerpos colisionables
    (\textit{Body}) y detecta colisiones entre ellos en cada iteración del
    bucle de juego filtrando los tipos de colisión que carecen de callback. 
\end{description}

% Shape y collision dispatching
\subsubsection{La clase Shape y RTTI}

La clase \textit{Shape} cuenta con un método estático \textit{getCollision()}
que recibe punteros a formas genéricas y devuelve verdadero o falso
en función de si se produce intersección entre las mismas o no. De forma
interna cuenta con métodos privados para detectar colisiones entre formas
concretas, por ejemplo el test para planos y cajas alineadas sería
\textit{getCollisionPlaneAABB()}.\\

La tarea que consiste en dados dos punteros a formas genéricas, elegir
el test de colisión adecuado se llama \textit{Collision dispatching}.
Podríamos emplear muchos bloques \texttt{if} y llamadas a \texttt{dynamic\_cast}
\cite{gera09} pero el resultado sería de lo más ineficiente y el tiempo es
un recurso muy preciado en la detección de colisiones. Necesitamos una forma
de hacer detección de tipos en tiempo de ejecución (\textit{RTTI} o Real Time
Type Identification) de forma eficiente y segura.\\

La solución por la que se ha optado consiste en incluir un método virtual
puro en \textit{Shape} llamado \textit{getType())} que obligue a las clases
descendientes a implementarlo y devuelva su tipo concreto del enumerado
\textit{Shape::Type}. La clase \textit{Shape} mantiene un conjunto
desordenado (\textit{boost::unordered\_map}) \cite{website:boost}
que dados dos tipos de formas nos devuelve el test de colisión adecuado
en un objeto función \textit{boost::function}.
Buscar en estos conjuntos es mucho más rápido que utilizar el clásico \textit{std::map}
de la biblioteca estándar de plantillas (STL).\\

Los tests de colisión entre formas concretas pueden hacer un \textit{static\_cast}
para obtener un puntero a la clase hija en lugar de a la genérica. La conversión
de tipos estáticas es mucho rápida aunque peligrosa. No obstante, ya hemos determinado
el tipo con seguridad plena gracias a \textit{getType()}. Al inicio de la aplicación
es necesario llamar a \textit{configureCollisionDispathing()} para inicializar
la tabla estática de tests de colisión. Si creamos formas nuevas y deseamos
añadir nuevos tests, es posible hacerlo mediante el método \textit{addCollisionTest()}.
A continuación, se adjunta la definición completa de la clase \textit{Shape}.\\

\lstinputlisting[style=C++]{codigo/shape.h}

En el siguiente fragmento, inicializamos la tabla de tests de colisión:\\

\begin{lstlisting}[style=C++]
void Shape::configureCollisionDispatching() {
    // Completamos la tabla de chequeos de colision
    _collisionDispatcher[SPHERE][SPHERE] = boost::bind(&Shape::getCollisionSphereSphere, _1, _2);
    _collisionDispatcher[AABB][AABB] = boost::bind(&Shape::getCollisionAABBAABB, _1, _2);
    _collisionDispatcher[PLANE][PLANE] = boost::bind(&Shape::getCollisionPlanePlane, _1, _2);
    _collisionDispatcher[OBB][OBB] = boost::bind(&Shape::getCollisionOBBOBB, _1, _2);

    _collisionDispatcher[AABB][SPHERE] = boost::bind(&Shape::getCollisionSphereAABB, _1, _2);
    _collisionDispatcher[AABB][PLANE] = boost::bind(&Shape::getCollisionPlaneAABB, _1, _2);
    _collisionDispatcher[AABB][OBB] = boost::bind(&Shape::getCollisionOBBAABB, _1, _2);

    _collisionDispatcher[SPHERE][AABB] = boost::bind(&Shape::getCollisionSphereAABB, _1, _2);
    _collisionDispatcher[SPHERE][PLANE] = boost::bind(&Shape::getCollisionPlaneSphere, _1, _2);
    _collisionDispatcher[SPHERE][OBB] = boost::bind(&Shape::getCollisionSphereOBB, _1, _2);

    _collisionDispatcher[PLANE][AABB] = boost::bind(&Shape::getCollisionPlaneAABB, _1, _2);
    _collisionDispatcher[PLANE][SPHERE] = boost::bind(&Shape::getCollisionPlaneSphere, _1, _2);
    _collisionDispatcher[PLANE][OBB] = boost::bind(&Shape::getCollisionOBBPlane, _1, _2);
    
    _collisionDispatcher[OBB][SPHERE] = boost::bind(&Shape::getCollisionSphereOBB, _1, _2);
    _collisionDispatcher[OBB][PLANE] = boost::bind(&Shape::getCollisionOBBPlane, _1, _2);
    _collisionDispatcher[OBB][AABB] = boost::bind(&Shape::getCollisionOBBAABB, _1, _2);
}
\end{lstlisting}

Cuando alguien desea saber si dos formas colisionan llama a \textit{getCollision()}
con dos punteros a formas genéricas. En dicho método es cuando se busca
en la tabla de tests de colisión.\\

\begin{lstlisting}[style=C++]
bool Shape::getCollision(Shape* shapeA, Shape* shapeB) {
    // Comprobamos si la forma A esta registrada
    CollisionDispatchTable::iterator itA;
    itA = _collisionDispatcher.find(shapeA->getType());
    
    if (itA == _collisionDispatcher.end()) {
        cout << "Shape::getCollision(): no existe el tipo " << shapeA->getType() << endl;
        return 0;
    }
    
    // Comprobamos que la forma B esta registrada
    CollisionDispatchTable::iterator itB;
    itB = _collisionDispatcher.find(shapeB->getType());
    
    if (itB == _collisionDispatcher.end()) {
        cout << "Shape::getCollision(): no existe el tipo " << shapeB->getType() << endl;
        return 0;
    }

    // Comprobamos que hay un metodo de comprobacion del tipo A - B
    boost::unordered_map<int, CollisionCheckFunction>::iterator itC;
    itC = _collisionDispatcher[shapeA->getType()].find(shapeB->getType());
    
    if (itC == itA->second.end()) {
        cout << "Shape::getCollision(): no existe un metodo de comprobacion entre" << shapeA->getType() << " y " << shapeB->getType() << endl;
        return 0;
    }

    // Llamamos al metodo de comprobacion
    return itC->second(shapeA, shapeB);
}
\end{lstlisting}

% Tests de colisión
\subsubsection{Tests de colisión}

En esta sección comentaremos con cierto nivel de detalle los tests de colisión
para cada pareja de tipos de formas. Para implementar estos tests se ha
recurrido a la publicación \textit{Real Time Collision Detection} de 
Christer Ericson \cite{eric05}. La explicación de cada test vendrá acompañada
del fragmento de código correspondiente y de un diagrama explicativo. Por motivos
de claridad, los diagramas se adjuntan en dos dimensiones pero los principios
son fácilmente extensibles a las tres dimensiones.\\

La mayoría de los tests de colisiones se sustentan en el \textbf{Teorema
del eje de separación} \cite{website:ejeseparacion}. Asegura que dados dos
objetos convexos en un plano 2D existe una línea sobre la cual, las proyecciones
de los dos objetos no se solapan si y sólo si los objetos son disjuntos
(no tienen puntos en común). La línea se conoce como eje de separación.
Si nos trasladamos a las tres dimensiones, la línea de separación se convierte
en plano de separación. Podemos ver un ejemplo del teorema en la figura
\ref{fig:ejeseparacion}.\\

\figura{ejeseparacion.png}{scale=0.8}{Teorema del eje de separación}{fig:ejeseparacion}{h}

El test entre una \textbf{Sphere} y otra \textbf{Sphere} es el más sencillo
de todos. Basta con comprobar si la distancia entre los centros de ambas
esferas es menor que la suma de sus radios, en tal caso existiría colisión,
tal y como puede verse en la figura \ref{fig:test-sphere-sphere}.\\

\figura{test-sphere-sphere.png}{scale=0.8}{Test de colisión Sphere \- Sphere}{fig:test-sphere-sphere}{h}

Para calcular la distancia entre dos puntos es necesario una raíz cuadrada
pero éstas son extremadamente caras en tiempo de procesamiento. Podemos
comparar la distancia al cuadrado con el cuadrado de la suma de los radios,
una expresión equivalente y de mayor eficiencia. A continuación adjuntamos
el código del test.

\begin{lstlisting}[style=C++]
bool Shape::getCollisionSphereSphere(Shape* shapeA, Shape* shapeB) {
    // Hacemos la conversion (estamos seguros de que son esferas)
    Sphere* sphereA = static_cast<Sphere*>(shapeA);
    Sphere* sphereB = static_cast<Sphere*>(shapeB);

    // Hacemos el test
    Ogre::Vector3 s = sphereA->getCenter() - sphereB->getCenter();
    Ogre::Real totalRadius = sphereA->getRadius() + sphereB->getRadius();

    return (s.squaredLength() <= totalRadius * totalRadius);
}
\end{lstlisting}

En la intersección entre cajas alineadas (\textbf{AABB} y \textbf{AABB})
con los ejes emplearemos el teorema del plano de separación. Proyectamos
las cajas sobre cada uno de los tres ejes y si algunas de las proyecciones
no se solapan podremos asegurar que no existe colisión entre las AABB.
La figura \ref{fig:test-aabb-aabb} ilustra el test.\\

\figura{test-aabb-aabb.png}{scale=0.8}{Test de colisión AABB \- AABB}{fig:test-aabb-aabb}{h}

\begin{lstlisting}[style=C++]
bool Shape::getCollisionAABBAABB(Shape* shapeA, Shape* shapeB) {
    // Hacemos la conversion (estamos seguros de que son AABBs)
    AxisAlignedBox* aabb1= static_cast<AxisAlignedBox*>(shapeA);
    AxisAlignedBox* aabb2 = static_cast<AxisAlignedBox*>(shapeB);

    // Hacemos el test
    return (aabb1->getMaxPos().x > aabb2->getMinPos().x &&
            aabb1->getMinPos().x < aabb2->getMaxPos().x &&
            aabb1->getMaxPos().y > aabb2->getMinPos().y &&
            aabb1->getMinPos().y < aabb2->getMaxPos().y &&
            aabb1->getMaxPos().z > aabb2->getMinPos().z &&
            aabb1->getMinPos().z < aabb2->getMaxPos().z);
}
\end{lstlisting}

El test de colisión entre dos figuras de tipo \textbf{Plane} también es
sencillo. Los planos son infinitos por lo que la única situación en la
que dos planos no colisionan es cuando estos son paralelos y no están a 
la misma distancia del origen. La orientación de los planos está definida
por su vector normal. Si las dos normales son paralelas y la distancia
con respecto al origen no coincide podremos asegurar que los planos no
colisionan. Dos vectores son paralelos si su producto escalar es igual a $1$.
Lo vemos ilustrado en la figura \ref{fig:test-plane-plane}.\\
 
\figura{test-plane-plane.png}{scale=0.8}{Test de colisión Plane \- Plane}{fig:test-plane-plane}{h}

\begin{lstlisting}[style=C++]
bool Shape::getCollisionPlanePlane(Shape* shapeA, Shape* shapeB) {
    // Hacemos la conversion (estamos seguros de que son Planes)
    Plane* planeA = static_cast<Plane*>(shapeA);
    Plane* planeB = static_cast<Plane*>(shapeB);

    // Hacemos el test
    Ogre::Vector3 normalA = planeA->getNormal().normalisedCopy();
    Ogre::Vector3 normalB = planeB->getNormal().normalisedCopy();
    return (normalA.dotProduct(normalB) != 1 ||
            planeA->getPosition() == planeB->getPosition());
}

\end{lstlisting}

En el test entre \textbf{Sphere} y \textbf{AxisAlignedBox} se pueden
producir dos casos en los que existe intersección entre los objetos.
El primero se da cuando el centro de la esfera está contenida en el AABB
mientras que el segundo tiene lugar cuando el centro está fuera de la caja
pero existe intersección (el diagrama \ref{fig:test-sphere-aabb} ilustra
el segundo caso). En primer lugar comprobamos si el centro de la esfera
está dentro de la caja. Posteriormente recorremos los vértices del AABB
y elegimos el más cercano al centro de la esfera. Si la distancia entre
ambos es menor que el radio de la esfera las dos formas colisionan.

\figura{test-sphere-aabb.png}{scale=0.8}{Test de colisión Sphere \- AABB}{fig:test-sphere-aabb}{h}

\begin{lstlisting}[style=C++]
bool Shape::getCollisionSphereAABB(Shape* shapeA, Shape* shapeB) {
    // Hacemos la conversion (estamos seguros de que A es Sphere y B es AABB)
    Sphere* sphere;
    AxisAlignedBox* aabb;
    if (shapeA->getType() == SPHERE) {
        sphere = static_cast<Sphere*>(shapeA);
        aabb = static_cast<AxisAlignedBox*>(shapeB);
    } else {
        sphere = static_cast<Sphere*>(shapeB);
        aabb = static_cast<AxisAlignedBox*>(shapeA);
    }

    // Hacemos el test
    Ogre::Real s = 0;
    Ogre::Real d = 0;
    Ogre::Vector3 center = sphere->getCenter();
    Ogre::Vector3 minPos = aabb->getMinPos();
    Ogre::Vector3 maxPos = aabb->getMaxPos();

    // Comprobamos si el centro de la esfera esta dentro del AABB
    bool centerInsideAABB = (center.x <= maxPos.x &&
                             center.x >= minPos.x &&
                             center.y <= maxPos.y &&
                             center.y >= minPos.y &&
                             center.z <= maxPos.z &&
                             center.z >= minPos.z);

    if (centerInsideAABB)
        return true;

    // Comprobamos si la esfera y el AABB se intersectan
    for (int i = 0; i < 3; ++i) {
        if (sphere->getCenter()[i] < aabb->getMinPos()[i]) {
            s = sphere->getCenter()[i] - aabb->getMinPos()[i];
            d += s * s;
        } else if (sphere->getCenter()[i] > aabb->getMaxPos()[i]) {
            s = sphere->getCenter()[i] - aabb->getMaxPos()[i];
            d += s * s;
        }
    }

    return (d <= sphere->getRadius() * sphere->getRadius());
}
\end{lstlisting}

Comprobar si una \textbf{Sphere} colisiona con un \textbf{Plane} es tan
sencillo como obtener la distancia entre ambos y compararla con el radio
de la esfera como hemos hecho en otras ocasiones. La distancia entre el
centro y el punto que conocemos del plano no es la distancia real entre
ambas formas. Para calcular la distancia real tendremos que proyectar
el vector $p-c$ (punto del plano - centro de la esfera) sobre la normal
del plano. Sólo nos es necesario el cuadrado de la distancia y lo
comprobaremos con el cuadrado del radio (para evitarnos utilizar una
raíz cuadrada). Puede verse el test en la figura \ref{fig:test-sphere-plane}.\\

\figura{test-sphere-plane.png}{scale=0.8}{Test de colisión Sphere \- Plane}{fig:test-sphere-plane}{h}

\begin{lstlisting}[style=C++]
bool Shape::getCollisionPlaneSphere(Shape* shapeA, Shape* shapeB) {
    // Hacemos la conversion (estamos seguros de que A es Plane y B es Sphere)
    Plane* plane;
    Sphere* sphere;
    if (shapeA->getType() == PLANE) {
        plane = static_cast<Plane*>(shapeA);
        sphere = static_cast<Sphere*>(shapeB);
    } else {
        plane = static_cast<Plane*>(shapeB);
        sphere = static_cast<Sphere*>(shapeA);
    }

    // Hacemos el test
    
    // Distancia del centro de la esfera al plano
    Ogre::Vector3 v = sphere->getCenter() - plane->getPosition();
    Ogre::Vector3 n = plane->getNormal().normalisedCopy();
    Ogre::Real d = abs(n.dotProduct(v));

    // Si d <= radio, hay colision
    return d <= sphere->getRadius();
}
\end{lstlisting}

Para el test entre un \textbf{AxisAlignedBox} y un \textbf{Plane} calculamos
el vértice más lejano y el más cercano al plano (\textit{pmin} y \textit{pmax}
respectivamente). Si cada punto está a un lado distinto del plano podemos
asegurar que ambas formas colisionan. El proceso se ilustra en la figura
\ref{fig:test-aabb-plane}.\\

\figura{test-aabb-plane.png}{scale=0.8}{Test de colisión AABB \- Plane}{fig:test-aabb-plane}{h}

\begin{lstlisting}[style=C++]
bool Shape::getCollisionPlaneAABB(Shape* shapeA, Shape* shapeB) {
    // Hacemos la conversion (estamos seguros de que A es Plane y B es AABB)
    Plane* plane;
    AxisAlignedBox* aabb;
    if (shapeA->getType() == PLANE) {
        plane = static_cast<Plane*>(shapeA);
        aabb = static_cast<AxisAlignedBox*>(shapeB);
    } else {
        plane = static_cast<Plane*>(shapeB);
        aabb = static_cast<AxisAlignedBox*>(shapeA);
    }


    // Hacemos el test
    Ogre::Vector3 p;
    Ogre::Vector3 n;

    for (int i = 0; i < 3; ++i) {
        if (plane->getNormal()[i] >= 0) {
            p[i] = aabb->getMaxPos()[i];
            n[i] = aabb->getMinPos()[i];
        } else {
            p[i] = aabb->getMaxPos()[i];
            n[i] = aabb->getMinPos()[i];
        }
    }

    // Si p esta en un lado diferente del plano que n, hay interseccion
    Ogre::Real d1 = plane->getNormal().dotProduct(p - plane->getPosition());
    Ogre::Real d2 = plane->getNormal().dotProduct(n - plane->getPosition());

    return ((d1 <= 0 && d2 >= 0) || (d1 >= 0 && d2 <= 0));
}
\end{lstlisting}

Si deseamos conocer si un \textbf{OrientedBox} y una \textbf{Sphere} buscamos
el punto más cercano de la caja a la esfera. Para hacerlo recorremos los vértices
y vamos almacenando la distancia mínima. Una vez lo hayamos encontrado
es sencillo ya que comparamos la distancia con el radio de la esfera. Como siempre,
utilizamos distancias al cuadrado para evitar el uso de raíces en la medida
de lo posible. El proceso aparece en la figura \ref{fig:test-obb-sphere}.\\

\figura{test-obb-sphere.png}{scale=0.9}{Test de colisión OBB \- Sphere}{fig:test-obb-sphere}{h}

\begin{lstlisting}[style=C++]
bool Shape::getCollisionSphereOBB(Shape* shapeA, Shape* shapeB) {
    // Hacemos la conversion (estamos seguros de que uno es Sphere y otro OBB
    Sphere* sphere;
    OrientedBox* obb;
    if (shapeA->getType() == OBB) {
        obb = static_cast<OrientedBox*>(shapeA);
        sphere = static_cast<Sphere*>(shapeB);
    } else {
        obb = static_cast<OrientedBox*>(shapeB);
        sphere = static_cast<Sphere*>(shapeA);
    }

    Ogre::Vector3 closest = closestPointToOBB(sphere->getCenter(), obb);

    Ogre::Vector3 v = closest - sphere->getCenter();

    return v.dotProduct(v) <= sphere->getRadius() * sphere->getRadius();
}

static Ogre::Vector3 closestPointToOBB(const Ogre::Vector3& p, const OrientedBox* obb) {
    Ogre::Vector3 d = p - obb->getCenter();
    Ogre::Vector3 closest = obb->getCenter();
    Ogre::Matrix3 axes = obb->getAxes();

    for (int i = 0; i < 3; ++i) {
        Ogre::Real dist = d.dotProduct(Ogre::Vector3(axes[i][0], axes[i][1], axes[i][2]));
        if (dist > obb->getExtent()[i]) 
            dist = obb->getExtent()[i];
        if (dist < -obb->getExtent()[i]) 
            dist = -obb->getExtent()[i];

        closest += dist * Ogre::Vector3(axes[i][0], axes[i][1], axes[i][2]);
    }

    return closest;
}
\end{lstlisting}

Para el test de colisión entre un \textbf{OrientedBox} y un \textbf{Plane}
proyectamos el primero sobre el segundo.\\

\figura{test-obb-plane.png}{scale=0.9}{Test de colisión OBB \- Plane}{fig:test-obb-plane}{h}

\begin{lstlisting}[style=C++]
bool Shape::getCollisionOBBPlane(Shape* shapeA, Shape* shapeB) {
    // Hacemos la conversion (estamos seguros de que uno es Plane y otro OBB
    Plane* plane;
    OrientedBox* obb;
    if (shapeA->getType() == OBB) {
        obb = static_cast<OrientedBox*>(shapeA);
        plane = static_cast<Plane*>(shapeB);
    } else {
        obb = static_cast<OrientedBox*>(shapeB);
        plane = static_cast<Plane*>(shapeA);
    }

    Ogre::Matrix3 axes = obb->getAxes();
    Ogre::Vector3 extent = obb->getExtent();
    Ogre::Vector3 normal = plane->getNormal();

    // Radio de la proyeccion de obb en el plano L(t) = obb.center + t * plane.normal
    Ogre::Real r = extent[0] * std::abs(normal.dotProduct(Ogre::Vector3(axes[0][0], axes[0][1], axes[0][2]))) +
                   extent[1] * std::abs(normal.dotProduct(Ogre::Vector3(axes[1][0], axes[1][1], axes[1][2]))) +
                   extent[2] * std::abs(normal.dotProduct(Ogre::Vector3(axes[2][0], axes[2][1], axes[2][2])));

    // Distancia del centro de la caja al plano
    Ogre::Real s = normal.dotProduct(obb->getCenter() - plane->getPosition());
    
    return abs(s) <= r;
}
\end{lstlisting}

Detectar una colisión entre dos \textbf{OrientedBox} es el más complejo de
todos. Los parámetros de las cajas de colisión están dispuestos en función
de los ejes globales pero en este caso calculamos los parámetros de una
de las cajas para que estén en función de los ejes definidos por la otra.
Una vez hecho eso vamos tratando de trazar planos de separación entre ambas
y buscando el descarte (método más rápido). Si no conseguimos encontrar el plano
de separación, se habrá producido una colisión. Puede verse en el diagrama
\ref{fig:test-obb-obb}.\\

\figura{test-obb-obb.png}{scale=0.9}{Test de colisión OBB \- OBB}{fig:test-obb-obb}{h}

\begin{lstlisting}[style=C++]
bool Shape::getCollisionOBBOBB(Shape* shapeA, Shape* shapeB) {
    // Hacemos la conversion, estamos seguros de que son OBB
    OrientedBox* obbA = static_cast<OrientedBox*>(shapeA);
    OrientedBox* obbB = static_cast<OrientedBox*>(shapeB);

    
    // FUENTE: Real Time Collision Detection pag 101


    // Obtenemos B en funcion de los ejes locales de A
    Ogre::Real ra, rb;
    Ogre::Matrix3 R, absR;
    Ogre::Matrix3 axesA = obbA->getAxes();
    Ogre::Matrix3 axesB = obbB->getAxes();

    for (int i = 0; i < 3; ++i) {
        for (int j = 0; j < 3; ++j) {
            Ogre::Vector3 vA(axesA[i][0], axesA[i][1], axesA[i][2]);
            Ogre::Vector3 vB(axesB[j][0], axesB[j][1], axesB[j][2]);
            R[i][j] = vA.dotProduct(vB);
        }
    }

    // Vector de translacion t en los ejes de A
    Ogre::Vector3 t = obbB->getCenter() - obbA->getCenter();
    t = Ogre::Vector3(t.dotProduct(Ogre::Vector3(axesA[0][0], axesA[0][1], axesA[0][2])),
                      t.dotProduct(Ogre::Vector3(axesA[1][0], axesA[1][1], axesA[1][2])),
                      t.dotProduct(Ogre::Vector3(axesA[2][0], axesA[2][1], axesA[2][2])));
        
    for (int i = 0; i < 3; ++i) 
        for (int j = 0; j < 3; ++j)
            absR[i][j] = std::abs(R[i][j]);

    // Test ejes L = A0 L = A1 L = A2
    for (int i = 0; i < 3; ++i) {
        ra = obbA->getExtent()[i];
        rb = obbB->getExtent()[0] * absR[i][0] +
             obbB->getExtent()[1] * absR[i][1] +
             obbB->getExtent()[2] * absR[i][2];

        if (std::abs(t[i]) > ra + rb) return false;
    }

    // Test ejes L = B0 L = B1 L = B2
    for (int i = 0; i < 3; ++i) {
        ra = obbA->getExtent()[0] * absR[0][i] +
             obbA->getExtent()[1] * absR[1][i] +
             obbA->getExtent()[2] * absR[2][i];
        rb = obbB->getExtent()[i];

        if (std::abs(t[0] * R[0][i] + t[1] * R[1][i] + t[2] * R[2][i]) > ra + rb) return false;
    }
   
    // Test eje L = A0 x B0
    ra = obbA->getExtent()[1] * absR[2][0] + obbA->getExtent()[2] * absR[1][0];
    rb = obbB->getExtent()[1] * absR[0][2] + obbB->getExtent()[2] * absR[0][1];
    if (std::abs(t[2] * R[1][0] - t[1] * R[2][0]) > ra + rb) return false;

    // Test eje L = A0 x B1
    ra = obbA->getExtent()[1] * absR[2][1] + obbA->getExtent()[2] * absR[1][1];
    rb = obbB->getExtent()[0] * absR[0][2] + obbB->getExtent()[2] * absR[0][0];
    if (std::abs(t[2] * R[1][1] - t[1] * R[2][1]) > ra + rb) return false;

    // Test eje L = A0 x B2
    ra = obbA->getExtent()[1] * absR[2][2] + obbA->getExtent()[2] * absR[1][2];
    rb = obbB->getExtent()[0] * absR[0][1] + obbB->getExtent()[1] * absR[0][0];
    if (std::abs(t[2] * R[1][2] - t[1] * R[2][2]) > ra + rb) return false;

    // Test eje L = A1 x B0
    ra = obbA->getExtent()[0] * absR[2][0] + obbA->getExtent()[2] * absR[0][0];
    rb = obbB->getExtent()[1] * absR[1][2] + obbB->getExtent()[2] * absR[1][1];
    if (std::abs(t[0] * R[2][0] - t[2] * R[0][0]) > ra + rb) return false;

    // Test eje L = A1 x B1
    ra = obbA->getExtent()[0] * absR[2][1] + obbA->getExtent()[2] * absR[0][1];
    rb = obbB->getExtent()[0] * absR[1][2] + obbB->getExtent()[2] * absR[1][0];
    if (std::abs(t[0] * R[2][1] - t[2] * R[0][1]) > ra + rb) return false;
   
    // Test eje L = A1 x B2
    ra = obbA->getExtent()[0] * absR[2][2] + obbA->getExtent()[2] * absR[0][2];
    rb = obbB->getExtent()[0] * absR[1][1] + obbB->getExtent()[1] * absR[1][0];
    if (std::abs(t[0] * R[2][2] - t[2] * R[0][2]) > ra + rb) return false;

    // Test eje L = A2 x B0
    ra = obbA->getExtent()[0] * absR[1][0] + obbA->getExtent()[1] * absR[0][0];
    rb = obbB->getExtent()[1] * absR[2][2] + obbB->getExtent()[2] * absR[2][1];
    if (std::abs(t[1] * R[0][0] - t[0] * R[1][0]) > ra + rb) return false;

    // Test eje L = A2 x B1
    ra = obbA->getExtent()[0] * absR[1][1] + obbA->getExtent()[1] * absR[0][1];
    rb = obbB->getExtent()[0] * absR[2][2] + obbB->getExtent()[2] * absR[2][0];
    if (std::abs(t[1] * R[0][1] - t[0] * R[1][1]) > ra + rb) return false;

    // Test eje L = A2 x B2
    ra = obbA->getExtent()[0] * absR[1][2] + obbA->getExtent()[1] * absR[0][2];
    rb = obbB->getExtent()[0] * absR[2][1] + obbB->getExtent()[1] * absR[2][0];
    if (std::abs(t[1] * R[0][2] - t[0] * R[1][2]) > ra + rb) return false;


    return true;
}
\end{lstlisting}

La colisión entre \textbf{OrientedBox} y \textbf{AxisAlignedBox} es muy
sencilla una vez hemos logrado implementar la anterior. Se basa en convertir
la caja alineada con los ejes en una caja orientada calculando sus parámetros.
Posteriormente, realizamos el test OBB con OBB como puede verse en la figura
\ref{fig:test-obb-aabb}.\\

\figura{test-obb-aabb.png}{scale=0.9}{Test de colisión OBB \- ABB}{fig:test-obb-aabb}{h}

\begin{lstlisting}[style=C++]
bool Shape::getCollisionOBBAABB(Shape* shapeA, Shape* shapeB) {
    // Hacemos la conversion (estamos seguros de que uno es AABB y otro OBB
    AxisAlignedBox* aabb;
    OrientedBox* obb;
    if (shapeA->getType() == OBB) {
        obb = static_cast<OrientedBox*>(shapeA);
        aabb = static_cast<AxisAlignedBox*>(shapeB);
    } else {
        obb = static_cast<OrientedBox*>(shapeB);
        aabb = static_cast<AxisAlignedBox*>(shapeA);
    }

    // Convertimos aabb en obb
    Ogre::Vector3 minPos = aabb->getMinPos();
    Ogre::Vector3 maxPos = aabb->getMaxPos();
    Ogre::Vector3 extent = (maxPos - minPos) * 0.5f;
    Ogre::Vector3 center = (maxPos + minPos) * 0.5f;

    OrientedBox convertedOBB("convertedOBB", center, extent, Ogre::Matrix3::IDENTITY);

    return getCollisionOBBOBB(obb, &convertedOBB);
}
\end{lstlisting}

\subsubsection{Cuerpos colisionables, clase Body}

Como hemos mencionado anteriormente, un \textit{Body} representa la parte
colisionable de un objeto y está formado por un vector de formas (\textit{Shape})
y una transformación (traslación con respecto al origen, escala y rotación).
Los cuerpos tienen un tipo (entero) que permite agruparlos y filtrarlos en
la detección de colisiones. Para manejar las colisiones entre dos cuerpos,
debemos cruzar las formas de ambos en coordenadas del mundo, en ningún
caso locales al objeto. Aplicar la transformación de cada cuerpo a cada
forma en todas las iteraciones del bucle de juego es demasiado costoso.
Por ello, he decidido almacenar un segundo vector de formas en coordenadas
del mundo, en el eterno dilema de la eficiencia tempo/memoria ha ganado
el tiempo en este caso. A continuación, mostramos la definición de la clase.\\

\lstinputlisting[style=C++]{codigo/body.h}

% El gestor de colisiones
\subsubsection{Gestor de colisiones, clase CollisionManager}

El gestor de colisiones sigue el patrón de diseño \textit{Singleton} \cite{gamm77}
y lleva el registro de todos los cuerpos colisionables de la escena (\textit{Body}).
Es posible añadir o eliminar cuerpos según nos convenga con los métodos
\textit{addBody()} y \textit{removeBody()}.
Una vez en cada iteración del bucle de juego es recomendable llamar al método
\textit{checkCollisions} para detectar e informar de las colisiones que se
produzcan.\\

Sólo se comprobarán colisiones entre cuerpos para cuyo tipo exista un
callback. Los callbacks son objetos función de \textsc{Boost} que reciben
dos punteros a \textit{Body} y no devuelven nada. Utilizando \textit{boost:bind}
podemos crear un objeto \textit{boost:function} y añadir el callback para dos
cuerpos de un tipo determinado. Por ejemplo, podemos hacer que el método
\textit{callbackSpellEnemy} sea llamado cuando colisionen cuerpos de los
supuestos tipos \textit{Spell} (cuyo número podría ser el 4) y \textit{Enemy}
(cuyo número podría ser el 8). Podemos incluso definir callbacks para el momento
en el que empieza una colisión, para el tiempo que dure la colisión o para el
instante en el que los cuerpos se separen.\\

En el método checkCollisions no sólo se filtran los cuerpos para los que
existe un callback definido sino que no se comprueban aquellos que están
a una distancia prudencial. Es cierto que, para cantidades ingentes de
elementos este particionado no es suficiente aunque para nuestras necesidades
funciona correctamente. A continuación se adjunta la definición de la clase:\\

\lstinputlisting[style=C++]{codigo/collisionManager.h}

El sistema de colisiones de \juego\ puede obtenerse de forma completamente
independiente y debidamente documentado a través de la siguiente dirección:\\

\url{http://forja.rediris.es/frs/download.php/2091/siontower-collisions-v0.2.tar.gz}\\


\subsection{Exportación de modelos 3D}

El artista de personajes 3D de \juego, Antonio Jiménez Rodríguez trabaja
con la herramienta privativa \textit{Cinema 4D} \cite{website:cinema4d}
(ver figura \ref{fig:cinema4d-moto}).
El dilema viene cuando \textsc{Ogre3D} sólo acepta un formato binario propio
basado en un sencillo \textit{XML} de extensión \textit{.mesh}. El motor
gráfico utiliza este formato propio (pero abierto) para optimizar la organización
de los datos en los modelos 3D, siempre mirando hacia el rendimiento. La mayoría
de herramientas de modelado y animación tridimensionales como \textit{Blender}
cuentan con exportadores a \textsc{Ogre3D} que funcionan correctamente y
son de código abierto pero solo existía un exportador para \textit{Cinema 4D}, 
el cual es privativo.\\

\figura{cinema4d-moto.jpg}{scale=0.2}{Moto modelada con Cinema 4D}{fig:cinema4d-moto}{h}

El plugin de exportación se llama \textit{I/Ogre} y, siguiendo su manual,
la exportación se realiza de forma correcta. El artista utilizaba una escala
distinta a la mía (una unidad de la herramienta 3D equivale a un metro
dentro del juego). Inicialmente no ocurría nada ya que el exportador era
capaz de aplicarle escalas a los modelos pero me di cuenta de que no funcionaba
correctamente. Decidí escribir un pequeño script en \textit{Python} para corregir
este problema en los modelos exportados.\\

\subsubsection{Plugin de exportación privativo}

Si tenemos un modelo \texttt{personaje}, tras la exportación contaremos
con los ficheros:

\begin{itemize}
    \item \texttt{personaje.mesh.xml}: xml en texto plano con la información
    de los vértices y caras del personaje. Esta información incluye la posición
    y rotación exacta de cada vértice dentro del espacio local del objeto.
    Se hacen referencias al fichero con el esqueleto y a sus materiales.
    \item \texttt{personaje.skeleton.xml}: xml en texto plano con la información
    de los huesos que forman el esqueleto del personaje de cara a la animación
    y su peso sobre los vértices asociados. Así mismo, incluye los fotogramas
    claves indicando la posición de cada hueso en cada momento.
    \item \texttt{personaje.material}: fichero en texto plano que sigue
    una sintaxis especial indicando los materiales que componen al personaje
    (texturas, colores, brillo...). Se puede leer más sobre esto en el artículo
    \textit{Materiales} de \wiki\ \cite{website:materiales}.
\end{itemize}

Ejemplo abreviado de fichero \texttt{.mesh.xml}:\\

\lstinputlisting[style=xml]{codigo/personaje.mesh.xml}

Ejemplo abreviado de fichero \texttt{.skeleton.xml}:\\

\lstinputlisting[style=xml]{codigo/personaje.skeleton.xml}

Posteriormente, pasaríamos a formato binario los ficheros con la malla y el
esqueleto del personaje utilizando el script que se distribuye con \textsc{Ogre3D}
llamado \texttt{OgreXMLConverter}. Antes de aplicar este script, podemos
manipular los ficheros \textit{XML} y corregir la escala de vértices y huesos
en fotogramas clave.\\

\subsubsection{Script de corrección de escala}

El script que se ha desarrollado simplemente toma uno de estos ficheros,
recorre todos los nodos del \textit{XML} y aplica la escala deseada
a todas las medidas (posiciones, tamaños, etc). Posteriormente guarda
el resultado en un fichero auxiliar y, si lo deseamos, aplica por nosotros
el script \texttt{OgreXMLConverter}. Su sintaxis es la siguiente:\\

\texttt{python fixExport.py [mesh|skeleton] originXML destinyXML \\scaleFactor [destinyBin]}

\begin{itemize}
    \itemsep0em
    \item \texttt{[mesh|skeleton]}: indicamos si queremos corregir una
    malla o un fichero de esqueleto.
    \item \texttt{originXML}: fichero xml de origen.
    \item \texttt{destinyXML}: fichero xml destino, en el que queremos
    guardar el resultado de la transformación.
    \item \texttt{scaleFactor}: factor por el que escalaremos las medidas.
    Un factor de $0.5$ reduce el tamaño a la mitad y uno de $2$ lo duplica.
    \item \texttt{[destinyBin]}: si lo deseamos, indicamos el fichero binario
    resultante y se aplicará el script \texttt{OgreXMLConverter}.
\end{itemize}

A continuación, adjuntamos el script completo:\\

\lstinputlisting[style=Python]{codigo/fixExport.py}

\subsection{Carga de escenarios desde Blender}

% Creación de niveles con Blender y convenciones de nombrado
\subsubsection{Creación de niveles y convenciones de nombrado}

En \juego\ los niveles se diseñan con la herramienta de modelado y animación
3D \textit{Blender} \cite{hess09}. Para más detalles sobre el proceso
de creación de niveles es posible acudir al apéndice \nameref{chap:manual}
en la página \pageref{chap:manual}. Tras el diseño del nivel utilizando
\textit{Blender}, se procede su exportación al formato \textit{Dotscene},
un sencillo XML que define la posición, escala y orientación de cada
objeto en la escena.\\

El sistema de carga de niveles de \juego\ procesa dicho fichero XML y va
creando los elementos de juego. No obstante, no sólo se adjunta información
sobre el escenario sino que necesitamos especificar dentro del nivel: oleadas
de enemigos, malla de navegación para la búsqueda de caminos, sistemas de
partículas... En definitiva, elementos que no pueden ser representados mediante
una malla tridimensional sin más. Es necesaria una convención de nombrado
para que el sistema que procesa los niveles sepa a qué se refiere cada
elemento. La nomenclatura es la siguiente:

\begin{itemize}
    \itemsep0em
    \item \textbf{Escenario}: todos los elementos del escenario (paredes,
    suelo, mesas, sillas y otros muebles) siguen la regla \texttt{scene-nombre.numero}.
    El nombre del objeto es lo que lo identifica dentro del catálogo para poder
    recuperar su modelo de colisión y el número ayuda a hacerlo único dentro de la
    escena (dos objetos no pueden tener el mismo nombre). Ejemplo: \texttt{scene-door.005}.
    \item \textbf{Efectos de partículas}: puedes incluir efectos de partículas
    en cualquier punto de la escena. La regla es \texttt{particle-nombre.numero}.
    Ejemplo: \texttt{particle-flame.001}.
    \item \textbf{Luces}: el motor del juego reconoce las luces de manera automática,
    por lo que no debes preocuparte de sus nombres.
    \item \textbf{Malla de navegación}: la malla de navegación indica a los
    enemigos cuales son las zonas transitables del escenario y siempre
    debe llamarse \texttt{navMesh}. Veremos más sobre las mallas de navegación
    en secciones posteriores.
    \item \textbf{Enemigos}: los enemigos siguen la regla \texttt{enemy-tipo-t.numero}.
    La letra t representa el segundo en el que aparecerá el enemigo desde
    que se inicia la partida. Ejemplo: \texttt{enemy-goblin-25.001}.
    \item \textbf{Protagonista}: el elemento cuyo nombre sea \texttt{player}
    definirá la posición inicial del jugador dentro del nivel.
    \item \textbf{Geometría arbitraria}: toda la geometría que no siga
    la convención de nombrado será tratada como complementos del escenario.
    No se calcularán colisiones contra ellos (se podrán atravesar).
\end{itemize}

Para procesar ficheros XML se ha empleado la sencilla, ligera y rápida
biblioteca libre \textsc{pugixml} \cite{website:pugixml}.\\

% Clase LevelManager y fichero levels.xml
\subsubsection{Gestor de niveles, clase LevelManager}

La clase \textit{LevelManager} es la encargada de gestionar a todos
los niveles de juego. También sigue el patrón de diseño \textit{Singleton}
ya que sólo necesitamos una instancia accesible desde varios módulos del
sistema \cite{gamm77}. Cuando creamos el gestor de niveles al inicio
de la aplicación procesa el fichero \texttt{[siontower]/media/
levels/levels.xml}
que contiene la lista ordenada de los niveles del juego.\\

\lstinputlisting[style=XML]{codigo/levels.xml}

En ese momento, el gestor de niveles crea todos los niveles cargando en memoria
únicamente su información más básica: identificador, nombre, descripción
y nombre de la canción que ha de sonar. Esto es necesario ya que hay que mostrar
dicha información en la pantalla de selección de nivel. De esta manera, los
niveles cuentan con dos estados: creado y cargado (listo para ser jugado).
Al gestor de niveles podemos pedirle que nos devuelva la lista completa
de niveles en orden o simplemente un nivel individual. Su definición es la
siguiente:\\

\lstinputlisting[style=C++]{codigo/levelManager.h}

% Clase Level, formato DotScene y _info.xml
\subsubsection{La clase Level y el formato DotScene}

Los objetos de la clase \textit{Level} se identifican por un código único
y obtienen su información de dos ficheros XML distintos. El primero es el
que contiene la información básica y el segundo es el resultante de la exportación
de Blender y que contiene todos los datos sobre la escena. Su nomenclatura
siempre es \texttt{id\_info.xml} y \texttt{id\_scene.xml} respectivamente
y se almacenan en el directorio \texttt{[siontower]/media/levels}.\\

\textit{Level} cuenta con sencillos métodos para recuperar su información
básica así como conocer si está cargado por completo y poder cargarlo
o liberar sus objetos según sea necesario. En caso de que esté cargado
podemos recuperar sus elementos (elementos del escenario, la posición
inicial del jugador, las oleadas de enemigos \textit{EnemySpawn} o la malla
de navegación). El nivel hace, en la práctica, de clase contenedora a la
que el estado de juego pregunta por su contenido.\\

Cuando se le pide a un nivel que cargue todos sus elementos, éste procesa
el XML en formato \textit{DotScene} de nombre \texttt{id\_scene.xml} y va
distribuyendo los nodos según sean enemigos, objetos del escenario, etc.
A continuación se adjunta un fragmento de fichero \textit{Dotscene} a modo
de ejemplo.\\

\lstinputlisting[style=XML]{codigo/level01_scene.xml}

La definición de la clase \textit{Level} es la siguiente:

\lstinputlisting[style=C++]{codigo/level.h}

% Catálogo de objetos

\subsubsection{Catálogo de objetos}

Los objetos que forman parte del escenario se cargan dentro del juego y
automáticamente cuentan con su modelo colisionable de forma que ni el personaje
ni los hechizos pueden atravesarlos. Esto se debe a que existe un catálogo
de objetos gestionado por \textit{LevelManager}. Cuando un nivel se encuentra
con un objeto de tipo \texttt{wall}, le pregunta al gestor de niveles si
existe algún modelo colisionable para dicho tipo y en caso afirmativo
le asigna una copia actuando según el patrón \textit{Factory} \cite{gamm77}.\\


\subsection{Búsqueda de caminos}

% Necesidades
Los enemigos de \juego\ no pueden ver por sí mismos y no son capaces de buscar
al personaje para atacarle. Es necesario un sistema de búsqueda de caminos
que guíe a estos enemigos por el escenario sin que atraviesen obstáculos
como paredes, sillas, mesas o columnas. El sistema debía ser lo suficientemente
general como para que fuera sencillo añadir niveles al juego y que los enemigos
pudieran seguir comportándose de la forma esperada. En esta sección expicaremos
cómo se ha resulto el problema de la búsqueda de caminos en el juego.\\

% Cell y NavigationMesh (su construcción a partir de un mesh.xml)
\subsubsection{Malla de navegación}

La solución pasa por asociar a cada nivel información sobre las zonas
transitables del escenario. Una malla de navegación es ideal para almacenar
este tipo de información y puede crearse junto al nivel utilizando \textit{Blender}.
Así conseguimos que no sea necesario que un programador cree los niveles, alguien
especializado en dicho aspecto y sin conocimientos de cómo está hecho el sistema
puede hacerlo. La malla estará formada por un conjunto de triángulos agrupados de forma
que construyan un grafo conexo, dirigido y ponderado como puede verse en
la figura \ref{fig:siontower-navmesh}.\\

\figura{siontower-navmesh.jpg}{scale=0.2}{Malla de navegación dibujada en modo depuración}{fig:siontower-navmesh}{h}

Cada una de las celdas de la malla de navegación está modelada por la
clase \textit{Cell}. Básicamente contiene información sobre los puntos
que forman su triángulo, el plano delimitado por el mismo y las celdas vecinas.
Incluye un método para conocer si un punto está contenido en la celda llamado
\textit{containsPoint()} y otro para, dado un punto en un plano bidimensional,
conocer la altura en la que encaja dentro de la celda llamado \textit{getHeight()}.
Este último resulta especialmente útil para incluir desniveles dentro de un
escenario (comoe escaleras o rampas). Con el método \textit{classifyPathToCell()}
podemos saber si una línea atraviesa la celda, termina en la celda, o no
guarda ninguna relación con la misma. Se adjunta la definición de la clase
a continuación.\\

\lstinputlisting[style=C++]{codigo/cell.h}

La malla de navegación o \textit{NavigationMesh} toma el fichero XML
resultante de la exportación de \textit{Blender} a \textsc{Ogre3D} y
crea las celdas (\textit{Cell}) que forman el grafo. Incluye el método
\textit{findCell()} para que una entidad sepa en qué celda se encuentra
en cada momento. Con \textit{buildPath()} podemos obtener un camino desde
cualquier punto de la malla a otro. Utilizando \textit{lineOfSightTest()}
podemos conocer si desde un punto determinado existe una linea de visión
hacia otro. Más adelante hablaremos sobre el algoritmo utilizado para
obtener dichos caminos.\\

\lstinputlisting[style=C++]{codigo/navigationMesh.h}

% A* vs precomputación
\subsubsection{Precomputación de caminos}

Inicialmente se pensó en emplear el algoritmo A* \cite{mill09} pero dicho
sistema trata de buscar el camino demandado cada vez que éste se consulta.
Los enemigos procuran pedir nuevos caminos en el menor número de momentos
posibles, pero en muchas ocasiones es estrictamente necesario. Si tenemos
unos cinco enemigos en pantalla y en el mismo cuadro todos piden la búsqueda
de un camino diferente podemos tener problemas si el algoritmo no es muy rápido.
Por esta razón se ha tomado la decisión de emplear el algoritmo de Floyd
y precomputar los caminos mínimos. Además, el algoritmo A* no es capaz
de asegurar su optimalidad, al contrario que Floyd.\\

Cuando creamos la malla de navegación debemos extraer el grafo a partir
de sus celdas. Almacenamos el gráfo en una matriz \texttt{nxn} donde \texttt{n}
es el número de celdas y \texttt{grafo[a][b]} es igual al coste de ir desde
el nodo a al nodo b. Esta tarea se lleva a cabo en el método privado \textit{initGraph()}
y todos los costes se inicializan a infinito menos entre las celdas vecinas ya
que en tal caso sería 1. Inicialmente nos podría preocupar el coste en memoria
de esta aproximación. En realidad el coste en memoria no es demasiado elevado.
Por ejemplo, en un nivel grande formado por $200$ celdas, tendríamos una matriz de $40000$
reales, esto implica un espacio en memoria de 156KB.\\

El algoritmo de Floyd toma una matriz de costes y devuelve la matriz de costes
mínimos así como una matriz de caminos para poder reconstruir las rutas \cite{website:floyd}.
Este algoritmo calcula la ruta mínima entre todos los nodos de un grafo y resulta
ideal para precalcular todos los caminos en el juego. Básicamente cruza
todos los vértices $i$ y $j$ buscando un tercero $k$ a modo de atajo de forma
que $coste(i, k) + coste(k, j) < coste(i, j)$. Cuando encuentra un atajo
modifica la matriz de costes y actualiza la matriz de caminos para que el
nuevo nodo intermedio se vea reflejado. Su orden es $O(n^{3})$ pero sólo
debe ejecutarse una vez durante la carga del nivel. El algoritmo integrado en nuestro
sistema es el siguiente:\\

\begin{lstlisting}[style=C++]
void NavigationMesh::floyd() {
    for (int k = 0; k < _cellNumber; ++k) {
        for (int i = 0; i < _cellNumber; ++i) {
            for (int j = 0; j < _cellNumber; ++j) {
                Ogre::Real ikj = _graph[i * _cellNumber + k] +
                                 _graph[k * _cellNumber + j];
                if (ikj < _graph[i * _cellNumber + j]) {
                    _graph[i * _cellNumber + j] = ikj;
                    _paths[i * _cellNumber + j] = k;
                }
            }
        }
    }
}
\end{lstlisting}

Creando el grafo tras generar las celdas de la malla y precalculando los
caminos mínimos con Floyd, nuestro sistema ya estaría listo para usarse.
No obstante, los caminos generados producen un efecto de zig-zag no deseado.
Los enemigos podrían saltar celdas intermedias si entre el origen y el destino
existe una ruta sin obstáculos. Durante la inicialización recorremos todos
las combinaciones de caminos posibles simplificando el camino utilizando
los tests de línea de visión (\textit{lineOfSightTest()}) y actualizando
la matriz de caminos cuando sea necesario.\\

\begin{lstlisting}[style=C++]
void NavigationMesh::precomputePaths() {    
    // Recorremos todas las combinaciones inicio - destino
    for (int i = 0; i < _cellNumber; ++i) {
        for (int j = 0; j < _cellNumber; ++j) {
            // Recuperamos el camino de i a j
            CellPath cellPath;
            recoverPath(i, j, cellPath);
            
            // Simplificamos el camino de i a j
            simplifyPath(cellPath);
            
            // Actualizamos la matriz de caminos con el camino simplificado de i a j 
            for (CellPath::iterator it = cellPath.begin(); it != cellPath.end(); ++it) {
                CellPath::iterator nextIt = it;
                ++nextIt;

                if (nextIt != cellPath.end()) {
                    int idA = (*it)->getId();
                    int idB = (*nextIt)->getId();
                    
                    // El camino entre nodos consecutivos es directo
                    _paths[idA * _cellNumber + idB] = -1;
                }
            }
        }
    }
}
\end{lstlisting}

% Recuperación del camino
\subsubsection{Recuperación del camino}

Cuando cualquier enemigo hace una llamada a \textit{NavigationMesh::buildPath()}
para encontrar la ruta más corta hasta el protagonista debemos acudir a la
matriz de caminos y recuperar la ruta. Se trata de un algoritmo recursivo
que se basa en la estructura de la matriz de caminos. Si tenemos los nodos
a y b y observamos que $camino(a, b) = c$ significa que el camino más corto
entre a y b pasa por c. Crear un vector de puntos o \textit{CellPath} basándonos
en esta idea, es sencillo entoces.

\begin{lstlisting}[style=C++]
void NavigationMesh::recoverPath(int i, int j, CellPath& cellPath) {
    // Tomamos el nodo intermedio por el que pasa el camino de i a j
    int k = _paths[i * _cellNumber + j];
    
    // Si hay atajo
    if (k != -1) {
        // Recuperamos el camino de i a k
        recoverPath(i, k, cellPath);
        
        // Insertamos la celda de id k en el camino
        cellPath.push_back(_cells[k]);
        
        // Recuperamos el camino de k a j
        recoverPath(k, j, cellPath);
    }
}
\end{lstlisting}

% Suavizado del camino
\subsubsection{Suavizado del camino}

\figura{spline.jpg}{scale=0.3}{Spline cúbico de Catmull-Roll}{fig:spline}{h}

A pesar de haber reducido celdas intermedias siempre que haya sido posible,
en el camino se producen cambios de orientación demasiado bruscos cuando
se llega de un punto intermedio a otro. Es posible suavizar el camino
empleando una interpolación. Tras recuperar el camino en el método
\textit{NavigationMesh::buildPath()}, introducimos puntos intermedios
gracias al spline cúbico de Catmull-Roll \cite{website:spline} (ver figura \ref{fig:spline}).
Es un algoritmo muy barato en tiempo de computación y mejora en gran
medida la calidad del comportamiento de los enemigos al seguir un camino de puntos.\\

El sistema se ha probado con una malla de navegación formada por 100 celdas
y el tiempo de carga y precomputación ha sido de 2 segundos. Cada consulta
para recuperar un camino únicamente ha necesitado 0.2 milisegundos.\\



\subsection{Inteligencia Artificial: Steering Behaviors}
\label{sec:steering}

Hemos expuesto la técnica que utilizamos para que los enemigos \textit{vean}
los obstáculos y puedan desplazarse por el escenario. No obstante, aún
teníamos que conseguir que realizaran el movimiento de seguir la línea de puntos
de una forma realista. Además, si dos enemigos se dirigen a capturar al personaje,
podrían colisionar entre sí y tenemos que evitarlo a toda costa. En definitiva,
necesitamos modelar el comportamiento de los enemigos y su forma de desplazarse
por el entorno. En esta sección pondremos solución a dicho problema.\\

\subsubsection{Conceptos básicos de movimiento}

Los algoritmos de movimiento toman datos de entrada del propio personaje y de
su entorno como otros personajes, escenario, un camino a seguir, obstáculos
o un objeto. Finalmente, producen una salida representando el movimiento que debería
realizar la entidad.
Algunos algoritmos de movimiento requieren muy pocos datos del personaje
como su posición. En cambio, otros necesitan su posición, velocidad lineal
y circular actuales, etc. Como salida pueden producir simplemente la velocidad
deseada (en forma de vector) o una fuerza a aplicar sobre la entidad
de forma que resulte en una aceleración para alcanzar el objetivo \cite{mill09}.\\

Los \textit{Steering Behaviors} son algoritmos de movimiento dinámico,
es decir, tienen en cuenta las propiedades cinemáticas completas del personaje
y producen una aceleración para cambiar su velocidad actual. Para ir de un
punto a otro, el algoritmo dinámico hace que el personaje acelere hacia dicho
punto, cuando se va acercando trata de frenar progresivamente para detenerse
completamente en el destino \cite{mill09}.\\

Los algoritmos de movimiento dinámicos producen comportamientos mucho más
realistas. Cuando una persona va de un punto a otro su velocidad no es constante,
sobre todo al iniciar la marcha. Sería muy extraño que los enemigos de \juego\
fueran de una velocidad nula a la máxima en un sólo frame.\\


\subsubsection{Las clases Kinematic y Steering}

Como hemos dicho anteriormente, los cambios repentinos de velocidad pueden
parecer extraños y poco naturales. Es necesario almacenar más información
para cada personaje de forma que los algoritmos de movimiento puedan trabajar
sobre ella. Tendremos una clase \textit{Kinematic} que almacenará estas propiedades:
posición, orientación, velocidad lineal, velocidad angular y máxima velocidad.
La información sobre la dinámica del personaje cuenta con un método
\textit{update()} para ser actualizado por la salida producida por los
algoritmos de movimiento (clase \textit{Steering}). Además cuenta con métodos
para modificar automáticamente la orientación del personaje en función
de su velocidad con \textit{setOrientationFromVelocity()} y para mirar
en una dirección determinada con \textit{lookAt()}. A continuación se adjunta
la definición de la sencilla clase.\\

\lstinputlisting[style=C++]{codigo/kinematic.h}

La salida de los algoritmos de movimiento la modelaremos utilizando la clase
\textit{Steering}. Estos algoritmos funcionan aplicando fuerzas así que
simplemente necesitamos almacenar una aceleración lineal para afectar a la
velocidad otra angular para la rotación. Es necesario sobrecargar los operadores
aritméticos de \textit{Steering} porque será común tener que combinar salidas
de varios algoritmos para producir una salida final. La definición de la clase
es la siguiente.\\

\lstinputlisting[style=C++]{codigo/steering.h}

Para actualizar la información dinámica de un personaje necesitamos la salida
del algoritmo de movimiento y el tiempo que ha pasado desde la última iteración
del bucle de juego. Es posible utilizar las ecuaciones que se imparten en
las clases de cinemática del instituto pero lo habitual es emplear
integración de Newton Euler \cite{mill09}.

\begin{lstlisting}[style=C++]
void Kinematic::update(const Steering& steering, Ogre::Real deltaT) {
    // Actualizamos posicion y orientacion
    _position += _velocity * deltaT;
    _orientation += _rotation * deltaT;
    
    // Actualizamos velocidad y rotacion
    if (steering.getLinear() == Ogre::Vector3::ZERO)
        _velocity = Ogre::Vector3::ZERO;
    else
        _velocity += steering.getLinear() * deltaT;
        
    _rotation += steering.getAngular() * deltaT;
        
    // Comprobamos si sobrepasamos la velocidad maxima
    if (_velocity.squaredLength() > _maxSpeed * _maxSpeed) {
        _velocity.normalise();
        _velocity *= _maxSpeed;
    }
}
\end{lstlisting}


\subsubsection{Steering Behaviors}

Los \textit{Steering Behaviors} toman información del personaje y otros
elementos del juego y producen una salida gracias al método \textit{getSteering()}.
La clase \textit{SteeringBehavior} modela esto de forma genérica, si deseamos
añadir un comportamiento nuevo, simplemente creamos una clase descendiente
de esta e implementamos su método virtual puro. Podemos crear una jerarquía
de algoritmos de movimiento de manera que los más complejos utilicen a los 
más simples para sus cálculos intermedios, lo veremos en la siguiente
descripción.\\

Se han implementado más comportamientos de los necesarios para el juego
de forma que otros usuarios puedan reutilizarlos en su sistema.
Cada clase modela un comportamiento con un objetivo, los implementados son
los siguientes:\\

\begin{description}
    \itemsep0em
    \item [Seek] toma la posición del personaje y la de un objetivo que trata
    de buscar alcanzando la velocidad máxima poco a poco. Sigue acelerando
    hasta que sobrepasa el objetivo, entonces trata de acelerar en dirección
    opuesta. Es muy simple pero no es útil para llegar a un objetivo y detenerse.
    
    \item [Flee] es el opuesto a \textit{Seek}, trata de huir de un punto
    con la mayor aceleración posible.
    
    \item [Arrive] es muy similar a \textit{Seek} ya que busca alcanzar un
    punto a la velocidad máxima. No obstante, cuando está llegando a un radio
    predeterminado, comienza a frenar para detenerse en el destino.
    
    \item [Align] recibe los datos dinámicos del personaje y una orientación
    objetivo que trata de igualar. Es útil para que un personaje mire hacia
    el mismo punto que otro de forma progresiva.
    
    \item [VelocityMatch] en este caso lo que pretendemos igualar son las velocidades
    de dos entidades.
    
    \item [Pursue] este comportamiento se diferencia de \textit{Seek} y \textit{Arrive}
    en que está preparado para perseguir un objetivo en movimiento. Se basa
    en predecir la posición del objetivo dentro de un tiempo determinado y dirigirse
    hacia ese punto a velocidad máxima.
    
    \item [Evade] es el opuesto a \textit{Pursue}, trata de huir de un perseguidor
    prediciendo su comportamiento.
    
    \item [Face] está diseñado para que un personaje mire hacia otro, deja
    que \textit{Align} haga el trabajo pero calcula primera la orientación
    deseada en función de los parámetros.
    
    \item [Wander] es un algoritmo de movimiento que hace que el personaje
    vague por el escenario. Básicamente coloca un objetivo en un radio
    alrededor del personaje y va seleccionando puntos de forma aleatoria
    a los que llegar delegando en \textit{Face} y \textit{Seek}.
    
    \item [CollisionAvoidance] se trata de un algoritmo de movimiento muy
    importante ya que evita que un personaje colisione con sus vecinos. Los
    enemigos lo utilizan para no atravesarse los unos a los otros mientras
    persiguen al personaje. Consiste en que si se detecta un compañero cerca,
    se huye en dirección opuesta empleado \textit{Flee}.
    
    \item [FollowPath] el algoritmo de movimiento básico para seguir un camino
    que nos haya devuelto el sistema de búsqueda de caminos. Simplemente
    se busca el próximo punto más cercano del camino y nos dirigimos a el
    empleando \textit{Arrive}.
\end{description}

En \juego, los algoritmos de movimiento más importantes son \textit{CollisionAvoidance}
y \textit{FollowPath}. Por ello, pasamos a explicarlos en mayor profundidad 
a continuación. La definición de \textit{CollisionAvoidance} es la siguiente:\\

\begin{lstlisting}[style=C++]
class CollisionAvoidance: public Flee {
    public:
        Ogre::Real maxAcceleration;
        std::vector<Enemy*> targets;
        Enemy* myself;
        Ogre::Real radius;
        
        CollisionAvoidance(Kinematic* character,
                           const std::vector<Enemy*>& targets,
                           Enemy* myself);
        void getSteering(Steering& steering); 
};
\end{lstlisting}

El algoritmo para evitar las colisiones parece más complejo de lo que en realidad
es. Simplemente recorremos la lista de enemigos comprobando que estén a
una distancia prudencial. En caso negativo huimos en dirección opuesta. 
Si no colisionamos con ningún enemigo nos quedamos con el que, si no varía
su rumbo actual, creamos que colisionará con mayor antelación y huimos de él.
En caso de que no haya enemigos cercanos, no se hace nada. Su implementación
se adjunta a continuación.\\

\begin{lstlisting}[style=C++]
void CollisionAvoidance::getSteering(Steering& steering) {
    
    // Parametros
    Ogre::Real neededSeparation = 0.0f;
    Ogre::Real posDistance = 0.0f;
    Ogre::Real velDistance = 0.0f;
    Ogre::Real separation = 0.0f;
    
    Ogre::Real time = 0.0f;
    Ogre::Real minTime = Ogre::Math::POS_INFINITY;
    
    Ogre::Vector3 posDifference;
    Ogre::Vector3 velDifference;
    
    Kinematic* selTarget = 0;
    Ogre::Real selSeparation = 0.0f;
    Ogre::Real selNeededSeparation = 0.0f;
    Ogre::Vector3 selPosDifference;
    Ogre::Vector3 selVelDifference;
    
    // Para cada enemigo
    std::vector<Enemy*>::iterator i;
    for (i = targets.begin(); i != targets.end(); ++i) {
        if (*i != myself) {
            target = &((*i)->getKinematic());
            
            // 1. Comprobamos que no colisionamos
            
            // Minima distancia
            neededSeparation = 2 * radius;
            
            // Distancia al objetivo
            posDifference = target->getPosition() - character->getPosition();
            posDistance = posDifference.length();
            
            // Si estamos colisionando, huimos
            if (posDistance <= neededSeparation) {
                Flee::getSteering(steering);
                return;
            }
            
            // 2. Comprobamos si colisionamos en un futuro
            velDifference = target->getVelocity() - character->getVelocity();
            velDistance = velDifference.length();
            time = (posDifference.dotProduct(velDifference)) / (velDistance * velDistance);
            
            if (time > 0.5f)
                continue;
                
            // Solo si tenemos tiempo de colision y es minimo
            if (time > 0.0f && time < minTime)
                minTime = time;
            else
                continue;
            
            // Calculamos la separacion en ese momento
            separation = posDistance - velDistance * minTime;
            
            // Si es el mas corto y va a colisionar lo guardamos
            if (separation <= neededSeparation) {
                selTarget = target;
                selSeparation = separation;
                selNeededSeparation = neededSeparation;
            }
        }
    }

    if (selTarget == 0)
        return;
        
    // Cambiamos la direccion de la velocidad
    
    // 1. Tomamos un punto lo mas lejano posible del punto de colision
    // 2. Cambiamos la velocidad actual para dirigirnos a ese punto
    
    // La nueva posicion del personaje al cabo del tiempo
    Ogre::Vector3 charPos = character->getPosition() + character->getVelocity() * minTime;
    Ogre::Vector3 targetPos = selTarget->getPosition() + selTarget->getVelocity() * minTime;
    
    // Direccion desde el objetivo al personaje
    posDifference = charPos - targetPos;
    posDifference.normalise();
    
    // Nueva posicion deseada: en la direccion posDifference
    posDifference = charPos + posDifference * (selNeededSeparation - selSeparation);
    
    // Direccion deseada para llegar al punto seguro
    Ogre::Vector3 desiredDir = posDifference - character->getPosition();
    desiredDir.normalise();
    desiredDir = desiredDir * character->getVelocity().length();
    
    // Aceleracion deseada
    steering.setLinear(desiredDir - character->getVelocity());
}
\end{lstlisting}

La clase \textit{FollowPath} simplemente almacena la información
dinámica del personaje y el camino de puntos como puede verse en el siguiente
fragmento de código.

\begin{lstlisting}[style=C++]
class FollowPath: public Arrive {
    public:
        NavigationMesh::PointPath* path;
        Ogre::Real pathOffset;
        
        FollowPath(Kinematic* character, NavigationMesh::PointPath* path);
        void getSteering(Steering& steering);
        
    protected:
        Ogre::Vector3 findTargetInPath();
};
\end{lstlisting}

El mecanismo consiste en buscar el punto del camino más cercano al personaje 
y tratar de llegar hasta él utilizando \textit{Arrive}. Encontrar el punto
más cercano consiste en recorrer todos los puntos e ir calculando y comparando
distancias. A continuación se adjunta el fragmento completo.\\ 

\begin{lstlisting}[style=C++]
void FollowPath::getSteering(Steering& steering) {
    // 1. Calculamos el target para delegar en Arrive
    target = new Kinematic(findTargetInPath());
    
    // 2. Delegamos en Arrive
    Arrive::getSteering(steering);
    
    delete target;
}

Ogre::Vector3 FollowPath::findTargetInPath() {
    Ogre::Real minDistance = Ogre::Math::POS_INFINITY;
    NavigationMesh::PointPath::iterator closestPointIt = path->begin();
    
    // Recorremos la lista de puntos buscando el mas cercano
    for (NavigationMesh::PointPath::iterator i = path->begin(); i != path->end(); ++i) {
        // Calculamos la nueva distancia
        Ogre::Vector3 direction = character->getPosition() - *i;
        
        if (direction.squaredLength() < minDistance) {
            minDistance = direction.squaredLength();
            closestPointIt = i;
        }
    }
    
    NavigationMesh::PointPath::iterator targetIt = closestPointIt;
    ++targetIt;
    
    if (targetIt == path->end())
        return *closestPointIt;
    
    return *targetIt;
}
\end{lstlisting}

Para conocer detalles adicionales sobre la implementación de los
\textit{Steering Behaviors} es recomendable acudir al propio código fuente
en la forja de RedIRIS.\\

\subsubsection{Máquina de estados para los enemigos}

\figura{enemigo-estados.jpg}{scale=0.4}{Máquina de estados para los enemigos}{fig:enemigo-estados}{h}

Todos los enemigos de \juego\ se comportan de manera similar, se diferencian
en atributos como velocidad, energía vital o fuerza. Internamente se implementa
una sencilla máquina de estados de forma que utilizamos algoritmos de movimiento
para resolver cada comportamiento combinados con consultas a la búsqueda
de caminos. El diagrama de estados puede observarse en la figura \ref{fig:enemigo-estados}.\\


\section{Pruebas}

Diseñar casos de prueba para un videojuego como \juego\ es una tarea complicada
ya que estamos simulando continuamente un mundo lleno de elementos que interactúan
entre sí y hay métodos que se ejecutan más de 60 veces en un sólo segundo.
Por supuesto, las pruebas son absolutamente necesarias, ya que nos ayudan
a desarrollar software de mayor calidad.\\

La mayoría de módulos han podido ser probados de forma independiente 
en pruebas unitarias de caja blanca. Entre ellos se encuentran las clases
del sistema de detección de colisiones, el gestor de perfiles, el gestor
de niveles o la búsqueda de caminos. No obstante, existen módulos que debían
trabajar en conjunción con otros por lo que sólo han sido sometidos a pruebas
de integración. Este caso se da en los algoritmos de movimiento, el subsistema
de audio o la gestión de estados de juego.\\

Más adelante se llevaron a cabo pruebas sobre la jugabilidad de \juego. Colaboradores
externos al probaron el juego y ofrecieron sus opiniones. Los aspectos a
analizar eran la dificultad, el control del personaje, velocidades,
fuerzas y otros parámetros relacionados con el balanceo. Finalmente
se llevaron a cabo pruebas sobre la interfaz, su intuitividad y resistencia
a valores inadecuados.\\

Durante las pruebas unitarias y de integración se utilizó software de depuración
como \textit{DNU Debugger} \cite{website:gdb} y \textit{Valgrind}
\cite{website:valgrind}. El primero nos ayuda a detectar
dónde se producen fallos en tiempo de ejecución y nos permite monitorizar
el valor de variables y otros parámetros. El segundo vigila la memoria
y nos permite localizar puntos en los que no liberemos de forma correcta
los recursos así como otros momentos en los que accedamos a memoria basura.\\

El proceso, de forma resumida, fue el siguiente:

\begin{enumerate}
    \itemsep0em
    \item Pruebas unitarias durante la fase de implementación, tras finalizar
    cada módulo.
    \item Pruebas de integración a medida que se completaban pequeños
    subsistemas que debían colaborar.
    \item Pruebas de jugabilidad con las primeas versiones usables del juego
    y empleando la ayuda de colaboradores externos.
    \item Pruebas de interfaz una vez se finalizó el desarrollo del juego.
\end{enumerate}

\subsection{Pruebas unitarias}

Las pruebas unitarias se realizaron junto a la fase de implementación, a medida
que se finalizaban los módulos. Se optó por un enfoque estructural con pruebas
de caja blanca. Se buscaba tomar todas las bifurcaciones posibles, o al menos
las más propensas a fallos, en cada módulo testado. De esta forma todas las
sentencias se ejecutarían al menos una vez y los posibles fallos saldrían
a relucir con mayor facilidad. Se prestó especial atención a problemas
relacionados con el redondeo de números reales ya que se utilizan de forma
intensiva para representar posiciones, rotaciones, aceleraciones y otros
parámetros imprescindibles. Así mismo, también se procuró que el sistema
no accediese a memoria no inicializada a través de punteros inválidos.\\

El mayor número de errores encontrados estaban relacionados con los accesos
inválidos a memoria. En este caso, el depurador del proyecto GNU y
\textit{Valgrind} fueron de gran ayuda. Así mismo, el sistema de detección
de colisiones presentó varios defectos en algunos de los tests de colisión
por la complejidad que entrañaban.\\

\subsection{Pruebas de integración}

A medida que el desarrollo de varios módulos de un mismo subsistema finalizaba,
se procedían a realizar pruebas de integración entre dichos módulos empleando
una aproximación de caja negra. Interesaba que el sistema realizara la tarea
para la que había sido diseñado de forma correcta. Cuando todos los módulos
del sistema estuvieron listos se realizaron pruebas adicionales de integración
a mayor escala. No sólo entre los módulos de un mismo subsistema, sino el
funcionamiento conjunto de varios subsistemas.\\

De nuevo, el mayor número de problemas fue localizado en la gestión de memoria.
Cuando se destruía un estado de juego para crear otro había elementos que no
se liberaban de forma adecuada o que no se inicializaban correctamente
y provocaban problemas a posteriori. \textit{Valgrind} y \textit{GDB}
fueron extremadamente útiles en dicho momento.\\

\subsection{Pruebas de jugabilidad}

Una vez se compilaron las primeras versiones usables de \juego\ se pidió
a colaboradores externos que probaran el videojuego. Tras estas sesiones
se les preguntó por aspectos como la capacidad de respuesta del control,
la comodidad de juego, dificultad y otros parámetros. Posteriormente,
se analizaron las respuestas de todos los colaboradores y se llevó a cabo
un proceso de balanceo para ajustar la dificultad.\\

La mayoría de colaboradores encontró el juego bastante complicado. Cuando
el personaje es rodeado tiene poca escapatoria posible y comienza a recibir
golpes sin parar. Esto se solucionó haciendo a los enemigos más lentos,
así el jugador podía esquivarlos con mayor facilidad y evitar ser rodeado
en la mayoría de ocasiones. Se reforzó al protagonista con una mayor cantidad
de energía vital y potenciando el poder destructivo de sus hechizos. 
Se suavizó el control de la cámara para que fuera más ágil y no fueran necesario
gestos con el ratón tan bruscos. Por último, se añadieron atajos de teclado
para que el jugador pudiese seleccionar de forma más cómoda los hechizos.
Tras aplicar los cambios, los colaboradores se mostraron mucho más
satisfechos con el juego.\\

\subsection{Pruebas de interfaz}

Tras completar las pruebas de jugabilidad, se procedió a realizar pruebas
sobre la interfaz del juego. Tanto en los menús como en la propia pantalla de
juego. Estas pruebas consistieron en medir los tiempos de carga, comprobar
que la interfaz fuera intuitiva y que no aceptaba valores erróneos. Se probó
a introducir nombres inválidos como perfil, tratar de seleccionar un nivel
bloqueado y utilizar hechizos para los que no disponíamos de maná suficiente.\\

Se demostró que la interfaz del juego era bastante sólida e intuitiva.
Prácticamente no era necesario un manual de usuario para que cualquiera
pudiera comprenderla y navegar por ella. El sistema de perfiles no admite
nombres erróneos y la pantalla de selección de nivel no nos permite
jugar a un nivel bloqueado para un perfil dado.\\




\chapter{Conclusiones}
\label{chap:conclusiones}
Tras haber explicado todo el proceso de desarrollo de \wiki\ y \juego\
a lo largo del presente documento, en esta sección haremos una valoración
acerca del proyecto. En primer lugar hablaré de como me ha afectado en el plano
personal y académico para después tratar los aspectos técnicos. Finalmente
finalizaremos con las posibles ampliaciones que podría sufrir en el futuro.\\

\section{Conclusiones personales}

Con \wiki\ y \juego\ he adquirido una gran cantidad de conceptos y me he
enriquecido muchísimo como persona y desarrollador. Si bien ya había
trabajado de forma independiente en algún videojuego sencillo o proyecto
de distinta índole, este sin duda era el reto de mayor envergadura al que
me había enfrentado jamás. De ahí que la fase de aprendizaje fuese tan
larga al comienzo del proyecto aunque después se extendiera durante
todo este tiempo.\\

En los siguientes puntos repasaré de forma superficial lo que he aprendido:

\begin{enumerate}
    \itemsep0em
    \item \textbf{Juegos 3D}: hasta el momento había desarrollado juegos
    sencillos en dos dimensiones pero no conocía el mundo de las tres dimensiones.
    Consideré que el Proyecto Fin de Carrera me brindaba una oportunidad
    excelente para aprender y me decidí a intentarlo. Es una aproximación
    muy diferente, atractiva y llena de retos. Es necesario conocer todo
    un conjunto de técnicas y conceptos totalmente nuevos y me ha resultado
    de lo más interesante.\\
    
    \item \textbf{Nuevas bibliotecas}: he hecho uso de muchas bibliotecas
    que no conocía y he necesitado aprender a utilizarlas correctamente.
    Entre estas bibliotecas se encuentran la propia \textsc{Ogre3D},
    \textsc{MyGUI}, \textsc{Boost} o \textsc{pugixml}. Anteriormente
    había utilizado \textsc{libSDL mixer} pero nunca me había visto obligado
    a integrarla dentro de un sistema más grande como es la gestión
    de recursos de \textsc{Ogre3D}. \textsc{Boost} es una biblioteca de
    lo más versátil y potente que complementa en muchos aspectos a la
    estándar de C++. De hecho, muchas de sus características aparecerán
    en el nuevo C++ 0x.\\
    
    \item \textbf{Matemáticas para videojuegos}: no sólo he tenido que repasar
    conceptos matemáticos de geometría del espacio y álgebra sino que he
    tenido que incorporar otros nuevos. Tanto de cara a \wiki\ como para
    \juego\ me he visto obligado a aplicar estos conceptos con el objetivo
    de buscar soluciones a problemas de programación.\\
    
    \item \textbf{Lenguaje C++}: desde el segundo curso de Ingeniería
    Técnica en Informática de Gestión he estado utilizando C++ no sólo
    dentro del ámbito estrictamente académico. No obstante, desarrollar
    \juego\ me ha servido para profundizar en los detalles del lenguaje
    y aprender a utilizarlos a mi favor.\\
    
    \item \textbf{Python}: este lenguaje de scripting orientado a objetos
    se ha mostrado extremadamente útil a la hora de desarrollar
    pequeños programas auxiliares. Por ejemplo, lo he utilizado en el script
    que soluciona el problema de la escala en la exportación de modelos
    tridimensionales y en el que extrae las cadenas traducibles de plantillas
    de interfaz. Es un lenguaje extremadamente sencillo y fácil de aprender
    pero muy potente y que agiliza el desarrollo de herramientas simples.\\
     
    \item \textbf{Optimización}: los videojuegos son sistemas complejos
    que, de no prestar especial atención, podrían hacer un consumo irracional
    de recursos. Es necesario aplicar técnicas de optimización tanto en
    términos de uso de procesador como de consumo de memoria. Este proyecto
    me ha servido para conocer varias de estas técnicas y algoritmos concretos
    cuyo rendimiento es superior al de otros en momentos concretos. Por ejemplo,
    en primer lugar empleé el algoritmo A* para la búsqueda de caminos
    aunque la precomputación con Floyd probó ser más eficiente.\\
    
    \item \textbf{Técnicas de IA}: hasta el momento conocía algunas técnicas
    de inteligencia artificial sobre teoría de juegos, búsqueda de caminos
    o autómatas para modelar comportamientos. No obstante, no había aplicado
    dichas técnicas al mundo de las tres dimensiones. En \juego\ se realiza
    una búsqueda de caminos a partir de una malla definida en tiempo de diseño.
    Además, los enemigos hacen uso de algoritmos de movimiento (
    \textit{Steering Behaviors}) que desconocía hasta este momento.\\
     
    \item \textbf{Diseño de un videojuego}: los videojuegos que había
    desarrollado con anterioridad con contaban con un documento de diseño
    en el que se detallara la forma de jugar, personajes, historia, etc.
    En este caso ha sido muy necesario ya que ha ayudado al proceso de análisis
    y diseño. Además, los colaboradores han podido conocer las necesidades
    del proyecto en cuanto a recursos artísticos se refiere. Los integrantes
    del equipo conocíamos en todo momento el estilo de juego y la apariencia
    que debía tener \juego.\\
    
    \item \textbf{Trabajo en equipo}: en \wiki\ se ha trabajado junto
    a la comunidad en todo momento. Los lectores enviaban sus opiniones y
    éstas debían ser tenidas en cuenta. En \juego\ el trabajo en equipo
    se hizo mucho más evidente ya que colaboramos en todo momento seis
    personas especializadas en disciplinas muy distintas. Hubo que realizar
    labores de coordinación, comunicación y resolución de conflictos
    (sobre todo relacionados con los formatos empleados). Una experiencia
    muy enriquecedora y que, sin duda, me ayudará en mi futuro profesional.\\ 
    
    \item \textbf{Redacción en MediaWiki}: hasta el momento mi relación
    con el motor \textit{MediaWiki} se había limitado a ediciones muy
    esporádicas en Wikipedia que no requerían conocimientos de la sintaxis
    que se utilizaba. Me he visto obligado a conocer la sintaxis de redacción
    en \textit{Wikimedia} para redactar correctamente los artículos.\\
    
    \item \textbf{Trabajo en wikis}: no sólo basta con conocer la sintaxis
    de \textit{Wikimedia} para poder participar en la confección de una wiki.
    Existen toda una serie de convenciones de escritura, comportamiento,
    nomenclatura y estructuras que deben ser conocidas.\\
    
    \item \textbf{Aplicación de conocimientos}: el desarrollo de este
    proyecto me ha sido especialmente útil para poner en práctica aquellos
    conocimientos adquiridos durante la Ingeniería Técnica. Sobre todo
    me refiero a aquellos relacionados con la Ingeniería del Software.\\
\end{enumerate}

\section{Conclusiones técnicas}

En \wiki\ y \juego\ se han cumplido con los objetivos propuestos en el
capítulo introductorio de esta memoria de Proyecto Fin de Carrera. En la
plataforma de aprendizaje se ha conseguido:

% Objetivos cumplidos con IberOgre
\begin{itemize}
    \itemsep0em
    \item Se ha creado contenido organizado en bloques temáticos tal y como
    se propuso: introducción, matemáticas para videojuegos, \textsc{Ogre3D},
    otras tecnologías y videojuegos desarrollados con \textsc{Ogre3D}.
    \item La navegación es intuitiva y los artículos están ordenados
    de forma aproximada en dificultad ascendente.
    \item Es posible adquirir los conocimientos de geometría del espacio
    necesarios para desarrollar juegos en 3D.
    \item Se han cubierto los aspectos básicos del uso del motor de renderizado
    \textsc{Ogre3D}.
    \item Otras tecnologías enfocadas al desarrollo de videojuegos han sido
    explicadas a través de varios artículos.
    \item Los artículos están dotados de un enfoque práctico gracias a los
    ejemplos finales y a los pequeños fragmentos de código intermedios.
    \item Varios usuarios han mostrado interés, han colaborado con el proyecto
    ya sea mediante correcciones, artículos, sugerencias o ayudando a difundir
    la plataforma.
\end{itemize}

% Objetivos cumplidos con Sion Tower
En el videojuego \juego\ hemos conseguido:

\begin{itemize}
    \itemsep0em
    \item Construir un videojuego completo empleando los conocimientos de
    \wiki\ y otros conceptos profusamente documentados.
    \item Creación de un motor orientado a la creación de contenido. Es posible
    crear nuevos niveles complejos sin necesidad de tocar una sola línea
    de código.
    \item Aplicación multilenguaje gracias a \textsc{gettext}.
    \item Creación de un videojuego que entretiene gracias a las impresiones
    que han prestado los colaboradores a lo largo de todas las fases del
    desarrollo.
    \item Implementación de un motor modular fácilmente ampliable. Varios
    subsistemas han sido liberados de forma independiente y están siendo
    utilizados por otros usuarios.
\end{itemize}

Se han generado estadísticas sobre el uso del repositorio \textit{Subversion}
empleando la herramienta libre \textit{StatsSVN} \cite{website:statssvn}.
En total el proyecto está compuesto por 619.000 líneas de código aunque
entre ellas se encuentran las distintas ramas (con código duplicado)
y líneas de documentación (esta memoria, por ejemplo). Cabe destacar
que se han realizado más de 500 commits, lo que permite volver hacia atrás
de forma sencilla. Es posible acceder al informe de estadísticas desde
la siguiente dirección.\\

\url{http://siondream.com/iberogre-siontower-statsvn}\\

\section{Trabajos futuros}

Es cierto que los objetivos que nos marcamos al comienzo del desarrollo
han sido cumplidos satisfactoriamente, no obstante se han detectado puntos
en los que el proyecto podría mejorar. A continuación hacemos una lista
de las posibles mejoras de \wiki.

% Mejoras en IberOgre
%   - Artículos en Ogre
%   - Artículos en otras tecnologías
%   - Artículos en videojuegos
\begin{itemize}
    \item Consolidación de una comunidad de redactores y lectores para
    la plataforma de pruebas.
    \item Nuevos artículos sobre videojuegos desarrollados con \textsc{Ogre3D}
    en los que el propio desarrollador comente la experiencia del desarrollo
    y proporcione o enlace documentación de interés.
    \item Artículos en la sección \textsc{Ogre3D} que documenten los sistemas
    de \textit{shading} con los que cuenta el motor.
    \item Documentar alguna biblioteca o sistema para incluir juego en red en
    la sección de otras tecnologías.\\
\end{itemize}

En \juego\ podrían incluirse las siguientes mejoras:

% Mejoras en Sion Tower
%   - Más hechizos
%   - Experiencia, subida de nivel, aumento tde características
%   - Idiomas adicionales
%   - IA distinta para cada enemigo

\begin{itemize}
    \item Nuevos hechizos: paralización, teletransporte para huir, colocación
    de trampas, etc.
    \item Mayor relevancia de los puntos de experiencia, podrían permitir
    subir de nivel, mejorar la energía y el maná así como desbloquear
    hechizos.
    \item Enemigos adicionales como arañas gigantes o criaturas voladoras
    con comportamientos diferentes.
    \item Nuevos niveles de mayor tamaño que permitan la aplicación de
    técnicas de búsqueda de caminos jerárquicas.
    \item Escenas narrativas mediante animaciones que fueran contando
    la historia entre niveles.
\end{itemize}




\backmatter % Apéndices, bibliografia ...

\addcontentsline{toc}{chapter}{Software usado}
\chapter*{Software utilizado}
\label{chap:software}
En esta sección hablaremos de las herramientas utilizadas durante el desarrollo
de \wiki\ y \juego. Para cada herramienta ofreceremos una pequeña descripción
de sus funcionalidades y adjuntaremos las razones por las cuales ha sido
elegida para el desarrollo frente a sus competidoras. No haremos mención
a las bibliotecas empleadas ya que han sido profusamente comentadas a lo largo
de la implementación de ambas partes del proyecto.\\


\section*{Subversion}

\textit{Subversion} \cite{website:svn} es un sistema de control de
versiones libre cuya primera versión fue lanzada en el año 2000 por CollabNet.
Es compatible en mayor parte con su predecesor \textit{CVS}. \textit{Subversion}
nos permite contar con una copia de seguridad tanto de los ejemplos de
\wiki\ como del código de \juego\ en todo momento. Gracias a esta herramienta
podemos guardar un historial de todas las versiones de los ficheros
fuente del proyecto así como deshacer cambios en caso de que fuera necesario.
Con \textit{SVN} conseguimos acceso al código del proyecto desde cualquier
equipo. Además, pone a disposición de cualquier interesado el código fuente
de forma sencilla.\\

Existen otros sistemas de control de versiones como \textit{Git} o
\textit{Mercurial}. Se ha elegido \textit{Subversion} por su sencillez
y por mostrarse suficientemente potente para gestionar el código de un proyecto
de un sólo programador. En cualquier caso, su uso era obligatorio según la
normativa del V Concurso Universitario de Software Libre (más sobre
el concurso en el apéndice \nameref{chap:comunidad}).\\

\section*{GNU GCC Compiler}

\textit{GCC} \cite{website:gnu-gcc} es el compilador del proyecto GNU y se
encarga de traducir nuestro código C++ a lenguaje máquina para poder
ejecutarlo posteriormente. El proyecto se inició en 1987 y actualmente
está disponible para multitud de arquitecturas como móviles Symbian
o consolas PlayStation 2.\\

\textit{GCC} fue desarrollado inicialmente para soporta el lenguaje
C aunque actualmente forma el GNU toolchain y es compatible con C++,
Fortran, Pascal, Objective-C, Java y Ada entre otros. Ha sido elegido
por ser el compilador libre compatible con el lenguaje más ampliamente
usado, estable y eficiente.\\

\section*{Make}

\textit{Make} \cite{pdf:make} es una utilidad para automatizar el proceso
de programas y bibliotecas a partir de su código fuente a partir
de ficheros con una sintaxis especial llamados \textit{makefile}. Estos
ficheros le indican a la herramienta cómo ha de ser compilado el software.\\

\textit{Make} acelera el proceso de compilación ya que no vuelve a generar
los ficheros objetos ya creados cuyos fuentes no han sido modificados. Además,
permite limpiar directorios y gestionar distintos modos de compilación
(modo depuración o lanzamiento, por ejemplo). En nuestro proyecto, \textit{Make}
también es utilizado para generar la documentación escrita en \LaTeX.\\

\section*{GNU Debugger}

\textit{GDB} \cite{website:gdb} es como se conoce popularmente al depurador de código del
proyecto GNU. Es un depurador destinado a sistemas Unix compatible con varios
lenguaje entre los que se encuentran, por supuesto, C y C++. Esta herramienta
nos permite marcar puntos de interrupción en el programa y examinar la pila
del programa cuando éste ha terminado su ejecución de manera abrupta. Es
posible acceder al valor de variables en todo momento para tratar de averiguar
las fuentes de los fallos del sistema.\\

El depurador básicamente nos ayuda a saber qué ocurre exactamente durante
la ejecución de nuestra aplicación. En \juego\ ha resultado ser una herramienta
imprescindible a la hora de identificar y solucionar problemas en tiempo
de ejecución. Sin duda, es una herramienta que colabora en el aumento de
la calidad del software si sabe utilizarse correctamente. Cuenta con un
sinfín de opciones y funcionalidades que no han sido explotadas en el
desarrollo de este proyecto.\\

\section*{Valgrind}

\textit{Valgrind} \cite{website:valgrind} es una herramienta para la
depuración del uso de la memoria, detección de pérdidas de memoria y
profiling. Esto nos permite identificar puntos en los que se accede a memoria
aún no inicializada y que probablemente contenga basura. También es útil
para saber si nuestra aplicación no la libera memoria utilizada por los
objetos cuando estos ya han sido destruidos. El profiling consiste en
monitorizar cuánto tiempo permanece el código en ejecución dentro de cada
función, esto nos ayuda a identificar posibles cuellos de botella. Su nombre
viene de la mitología nórdica y se refiere a la entrada del Valhala.\\

Es compatible con sistemas GNU/Linux y Mac OS X. Ha resultado extremadamente
útil en aquellos puntos a los que \textit{GDB} no era capaz de llegar. No
obstante, al ser una herramienta de análisis dinámico, la ejecución
de \juego\ se ralentizaba en exceso aún trabajando con equipos muy potentes.\\

\section*{Vim}

\textit{Vim} \cite{website:vim} es un editor de textos libres creado en
1991 para el computador Amiga. \textit{Vim} significa \textit{Vi IMproved}
ya que es una versión mejorada del anterior editor \textit{Vi}. Es editor
compatible con sistemas GNU/Linux extremadamente ligero que no precisa
de entorno gráfico ya que puede ejecutarse desde la terminal. Está orientado
a la productividad y cuenta con decenas de combinaciones de teclas para
acelerar la escritura de código.\\

Es muy completo, cuenta con corrector ortográfico, resaltado de sintaxis
para decenas de lenguajes e ingentes opciones para personalizar el entorno.
Incluso es posible ampliar sus capacidades mediante extensiones. Ha sido utilizado
para el proyecto por su ayuda a la productividad y ligereza. Puede observarse
\textit{Vim} siendo ejecutado en una terminal en la figura \ref{fig:vim}.\\ 

\figura{vim.jpg}{scale=0.25}{Editor de textox Vim}{fig:vim}{h}


\section*{LaTeX}

\LaTeX \cite{website:latex-wikibooks} es lenguaje de marcado y un sistema
de creación de documentos especialmente orientado al mundo científico y
técnico. El lenguaje \textit{Tex} fue concebido por Donald Knuth durante
los 80 y en 1984 Leslie Lamport creó \LaTeX como un framework para trabajar
con cartas, libros y otro tipo de textos.\\

Los resultados que producen \LaTeX son coherentes, ordenados y muy limpios.
Si bien aprender su uso puede ser complejo, los resultados son de una
enorme calidad si los comparamos con los documentos que producen los editores
convencionales como \textit{Microsoft Word} o \textit{LibreOffice Writer}.
Por esta calidad y posibilidad de automatizar formatos ha sido elegido
\LaTeX para la redacción de la documentación del proyecto.\\

\section*{Doxygen}

\textit{Doxygen} \cite{website:doxygen} es una herramienta de documentación
automática de código. Mediante la inclusión de comentarios especiales
dentro de los ficheros de cabecera de nuestro proyecto, es capaz
de generar documentación con una apariencia atractiva tanto en formato HTML como
en \LaTeX o muchos otros. Es compatible con los lenguajes C, C++, C\#,
Fortran, Java, Objective-C, PHP, Python, IDL y algunos más.\\

Con \textit{Doxygen} se produce una documentación legible por usuarios (con
los conocimientos necesarios) u otros miembros del equipo. Es posible incluir
diagramas de colaboración y herencia gracias a su uso de graphviz. Ha sido
elegida como herramienta de documentación de código por su sencillez de uso,
limpieza y por su amplia aceptación.\\


\section*{Blender}

\textit{Blender} \cite{hess09} es una herramienta libre de modelado y animación
en tres dimensiones multiplataforma. Es compatible con Windows, Mac OS X,
GNU/Linux y otros sistemas operativos. Cuenta con funcionalidades avanzadas
de modelado 3D, mapeado UV para las texturas, renderizado, texturizado, 
animación basada en esqueletos, efectos de partículas y simulaciones físicas.\\

Integra un motor de físicas y colisiones y proporciona una API compatible
con Python para programar complementos e incluso videojuegos completos
gracias al \textit{Blender Game Engine}. Ha sido utilizado en el desarrollo
de cortometrajes libres como Big Buck Bunny aunque también se emplea de forma
profesional en la industria de la publicidad.\\

Ha sido elegido para ser empleado en el proyecto por ser el editor 3D
libre por excelencia y por haber sido probado en entornos de producción
muy importantes. Además de ser utilizado para diseñar los elementos
del escenario, se emplea como editor de niveles completo.\\

\figura{blender.jpg}{scale=0.2}{Herramienta de modelado y animación 3D Blender}{fig:blender}{h}

\section*{GIMP}

\textit{GIMP} \cite{website:gimp} es el editor de imágenes libre del proyecto
GNU, de hecho su nombre es un acrónimo de \textit{GNU Image Manipulation
Program}. Es multiplataforma y está disponible para Windows, GNU/Linux y Mac OS X.
No es comparable a soluciones privativas como \textit{Adobe Photoshop} pero
es capaz de realizar operaciones bastante avanzadas de forma sencilla.
Su interfaz puede verse en la figura \ref{fig:gimp}.\\

Ha sido utilizado en \juego\ para trabajar con las texturas del escenario.
Hemos elegido esta herramienta por contar con una licencia libre, ser
lo suficientemente potente para nuestras necesidades y estar disponible
en varias plataformas (incluida GNU/Linux).\\

\figura{gimp.jpg}{scale=0.22}{Editor gráfico Gimp}{fig:gimp}{h}

\section*{Inkscape}

\textit{Inkscape} \cite{website:inkscape} es una herramienta libre multiplataforma
para trabajar con gráficos vectoriales. Estos diseños no están basados en píxeles
sino en elementos geométricos como vectores o figuras sencillas. Esto permite
un escalado ilimitado sin pérdida de calidad. Inkscape trabaja con ficheros
en formato \textit{SVG}.\\

Ha sido utilizado tanto en \wiki\ como en \juego\ para trabajar con gráficos
bidimensionales tales como elementos de interfaz o diagramas más vistosos.
Es prácticamente la única alternativa libre a la altura del software comercial
existente.\\

\figura{inkscape.jpg}{scale=0.25}{Herramienta de gráficos vectoriales Inkscape}{fig:inkscape}{h}


\section*{MyGUI Layout Editor}

La biblioteca de interfaces \textsc{MyGUI} utiliza unas plantillas en formato
\textit{XML} para definir los elementos de las pantallas como paneles, botones
o imágenes. Es posible escribir estas plantillas manualmente pero es una tarea
ardua. Afortunadamente existe \textit{MyGUI Layout Editor}, una herramienta
para diseñar interfaces y guardarlas en dicho formato de manera que el motor
pueda cargarlas de forma sencilla posteriormente.\\

Es multiplataforma, libre y trabaja de una manera muy limpia ya que no genera
código de ningún tipo, únicamente produce un \textit{XML}. Es muy intuitiva
y gracias a ella se han diseñado todas las pantallas de \juego. La última
versión cuenta con la siguiente interfaz (figura \ref{fig:layouteditor}).\\

\figura{layouteditor.jpg}{scale=0.20}{Editor de plantillas MyGUI Layout Editor}{fig:layouteditor}{h}

\section*{Particle Editor}

Los sistemas de partículas de \textsc{Ogre3D} se definen en scripts con
una sintaxis especial. Al igual que ocurre con la interfaz, escribirlos a mano
es muy pesado y no permite ver los resultados hasta que se inicie el juego.
\textit{Ogre Particle Editor} es un editor de sistemas de partículas para
\textsc{Ogre3D}. Nos permite crear partículas y modificar sus parámetros
viendo los resultados en tiempo real.\\

Una vez hayamos terminado, guarda el resultado en un fichero con la sintaxis
anteriormente mencionada. Esto acelera enormemente el proceso de creación
de efectos de partículas y por ello esta herramienta ha sido utilizada
en \juego.\\

\figura{particleeditor.jpg}{scale=0.4}{Editor de sistemas de partículas Particle Editor}{fig:particleeditor}{h}


\section*{Audacity}

\textit{Audacity} \cite{website:audacity} es un editor de audio libre y
multiplataforma. Nos permite grabar y modificar audio con un gran número
de opciones y variantes. Fue lanzado por primera vez en mayo del 2010
y actualmente cuenta con más de 72 millones de descargas. Si bien es cierto
que carece de herramientas avanzadas de edición de sonido, para nuestras
necesidades era la herramienta ideal.\\

\textit{Audacity} ha sido utilizado en \juego\ para convertir y retocar
los efectos de sonido y las pistas de la banda sonora que envían los
correspondientes artistas. Su interfaz puede observarse en la figura
\ref{fig:audacity}.\\

\figura{audacity.jpg}{scale=0.25}{Editor de audio Audacity}{fig:audacity}{h}

\section*{XVidCap}

\textit{XVidCap} \cite{website:xvidcap} es una utilidad libre para capturar
la pantalla en vídeo y audio. Suele utilizarse para crear tutoriales de procesos
complejos aunque en \juego\ se emplea para tomar los vídeos con el objetivo
de difundir el proyecto. Puede capturar el escritorio completo, ventanas
concretas o un área determinada. Nos permite calibrar el formato de vídeo
y audio, la calidad así como el número de cuadros por segundo.\\

Se trata de la herramienta para capturar vídeo del escritorio más eficiente
y que menos peso ejercía sobre el rendimiento general del sistema. Su interfaz
es muy sencilla y únicamente consta de la barra que puede verse en la figura
\ref{fig:xvidcap}.\\

\figura{xvidcap.jpg}{scale=0.5}{Capturador de pantalla XvidCap}{fig:xvidcap}{h}

\section*{OpenShot Video Editor}

\textit{OpenShot Video Editor} \cite{website:openshot} es un sencillo pero
potente editor de vídeo para sistemas GNU/Linux. Soporta multitud de formatos
de audio y vídeo gracias a su uso de la biblioteca \textsc{ffmpeg}. Creamos
el vídeo añadiendo y organizando recursos sobre las pistas que deseemos.
Es posible incluir transiciones, subtítulos, carteles y otros efectos.
Podemos configurar la exportación de vídeo a través de numerosos parámetros
e incluso se proporcionan perfiles para subir vídeos a conocidos servicios
como \textit{Youtube}.\\

Los vídeos de \juego\ creados con \textit{XVidCap} son procesados con
\textit{OpenShot Video Editor} y procesados para ser subidos a algún
servicio de vídeos por streaming con el objetivo de difundir el proyecto.
Es la herramienta libre más potente y sencilla disponible para GNU/Linux.\\

\figura{openshot.jpg}{scale=0.25}{Editor de vídeo OpenShot}{fig:openshot}{h}

\section*{Planner}

\textit{Planner} \cite{website:planner} es una herramienta libre para entornos
GTK que nos permite planificar y realizar un seguimiento de cualquier proyecto.
Se definen tareas sobre un calendario especificando sus dependencias, duraciones
y fechas límite. Posteriormente, se crean y asignan recursos (materiales
o humanos) a dichas tareas. Finalmente se obtiene un diagrama de Gantt con
la organización temporal del proyecto. Las planificaciones pueden guardarse
en una base de datos \textit{postgresql} o en un sencillo formato \textit{XML}.
Incluso puede exportarse a \textit{HTML} para que pueda ser visualizada
desde cualquier navegador.\\

Toda la planificación del proyecto que puede verse en el capítulo
\ref{chap:calendario} ha sido creada con \textit{Planner}. Es una herramienta
muy intuitiva y mucho más ligera que alternativas como \textit{GanttProject}.\\

\figura{planner.jpg}{scale=0.28}{Planificador de proyectos Planner}{fig:planner}{h}

\section*{BOUML}

\textit{BOUML} \cite{website:bouml} es una herramienta libre para diseñar
diagramas siguiendo la notación \textit{UML}. Cuenta con utilidades de generación
automática de código a partir de los diagramas en lenguajes C++, Java, PHP,
Python e IDL. Es multiplataforma y mucho más ligero que herramientas similares
como \textit{Umbrello}. Permite la reutilización de clases entre diagramas
y cuenta con una amplia variedad de posibilidades: diagramas de clases, 
de interacción, de secuencia, de casos de uso, etc.\\

Tanto en el análisis (capítulo \ref{siontower-analisis}) como el diseño
(capítulo \ref{sec:siontower-diseno}) ha participado la herramienta \textit{BOUML}.
Su interfaz no es elegante pero resulta funcional y sencilla de utilizar,
puede verse en la figura \ref{fig:bouml}.\\

\figura{bouml.jpg}{scale=0.25}{Editor de diagramas con notación UML BOUML}{fig:bouml}{h}

\section*{Dia}

\textit{Dia} \cite{website:dia} es una herramienta libre para crear diagramas
de propósito general. Permite varios tipos como notación \textit{UML},
diagramas de flujo, entidad relación, circuitería y un largo etcétera. Una vez
hayamos terminado de trabajar en nuestro diagrama, podemos exportarlo a
varios formatos como JPG, PNG o SVG.\\

Si bien es mucho menos potente que \textit{BOUML} permite realizar otros
tipos de diagramas. Concretamente, en \juego\ ha sido empleado para esquemas
de flujo como la máquina de estados que puede verse en la figura \ref{fig:dia}.\\

\figura{dia.jpg}{scale=0.28}{Herramienta de creación de diagramas}{fig:dia}{h}


\addcontentsline{toc}{chapter}{Manual de usuario}
\chapter*{Manual de usuario}
\label{chap:manual}
% -*-memoria.tex-*-
% Este fichero es parte de la plantilla LaTeX para
% la realización de Proyectos Final de Carrera, protejido
% bajo los términos de la licencia GFDL.
% Para más información, la licencia completa viene incluida en el
% fichero fdl-1.3.tex

% Copyright (C) 2009 Pablo Recio Quijano 

%-------------------------------------------------------
% ---- Plantilla para libros / memorias PFC -----

% Realizada por Pablo Recio Quijano y Noelia Sales Montes 
% Formato de portada y primera página tomado del PFC de
% Francisco Javier Vázquez Púa, en su proyecto 'libgann'
% -------------------------------------------------------

\documentclass[a4paper,11pt]{article}

\usepackage{./estilos/estiloBase} % Basicamente son todas las
                                  % librerias usadas. En caso de que
                                  % falten librerias se van añadiendo
                                  % al fichero.
\usepackage{./estilos/colores}  % Algunos colores ya generados, para
                                % los algunos estilos más avanzados.
\usepackage{./estilos/comandos} % Algunos comandos personalizados

\graphicspath{{./imagenes/}} % Indicamos la ruta donde se encuentran
                             % las imagenes, para ahorrarnos la ruta
                             % completa, y solo modificar aquí si en
                             % un momento dado lo movemos

\begin{document}


\thispagestyle{empty}
\begin{picture}(0,0)
	\put(65,-90){\includegraphics[scale=0.5]{imagenes/logo-siontower.png}}
\end{picture}\\[5cm]
	
	\begin{center}
		\makeatletter
		{\bf {\Huge Manual de usuario}}
		\\[4cm]
		\@date\\[2cm]
		{\footnotesize Revisión 1}
		\\[5cm]
		\begin{tabular}[t]{c} David Saltares Márquez \end{tabular}\\[3cm]
		\makeatother
		
		\begin{center}
			\includegraphics[scale=0.9]{imagenes/by-nc-sa.png}
		\end{center}
	\end{center}

\cleardoublepage

\tableofcontents

\cleardoublepage

\section{Bienvenido a Sion Tower}

Bienvenido a \juego, un videojuego de estrategia y acción en 3D ambientado
en un mundo fantástico compatible con sistemas GNU/Linux y Windows. A lo
largo de estas páginas podrás conocer los detalles sobre la historia y la
ambientación del juego. Más adelante encontrarás los requisitos mínimos,
de forma que puedas saber si tu equipo podrá ejecutarlo sin problemas.
A continuación se ofrece una completa guía de instalación para los dos
sistemas operativos soportados. Por supuesto, este manual contiene todos
los detalles sobre el control y las posibilidades de \juego. Finalmente
encontrarás información detallada sobre cómo crear tus propios escenarios
y añadirlos al juego.\\

\figura{logo-siontower.png}{scale=0.4}{Logo de Sion Tower}{fig:logo-siontower}{H}

\subsection{Historia}

El gremio de magos está localizado en la Torre Sagrada, \juego. Allí viven,
estudian los libros de hechizos, celebran sus ritos secretos y protegen
innumerables riquezas. Un día, el gremio al completo abandona la Torre para
celebrar un ritual en el bosque cercano. \prota, protagonista
de \juego\ es el único que permanece en el edificio sagrado para protegerlo.
A pesar de ser un joven iniciado e inexperto, no iba a suceder nada porque
el resto se ausentase durante unas horas.\\

La imprudencia del gremio resulta ser completamente desastrosa y una horda
de monstruos no tarda en rodear \juego. La misión del joven aprendiz consiste
en detener la invasión piso por piso de la Torre. ¡Debe evitar que roben
la \textbf{Reliquia Sagrada}!

\subsection{Protagonista}

\prota\ es pertenece a una de las categorías inferiores del gremio, los
iniciados. Se le reconoce fácilmente por su pequeña estatura y sus ropas
sencillas. Es un simple estudiante de magia en la Torre Sagrada y lleva
tiempo tratando de hacerse un hueco entre sus superiores a base de esfuerzo
y entrega. La invasión de la Torre es, en cierta medida, una gran oportunidad
para \prota\ de mostrar su valía. No obstante, está muerto de miedo, nunca
se había visto obligado a emplear sus poderes en una situación tan extrema.

\figura{personaje.jpg}{scale=0.3}{Protagonista de Sion Tower}{fig:protagonista}{H}

Características físicas:

\begin{itemize}
    \itemsep0em
    \item Vida: 100
    \item Maná: 100 (regeneración automática)
    \item Velocidad: $5 m/s$
    \item Hechizos: Bola de fuego, Furia de Gea y Ventisca.
\end{itemize}

\subsection{Enemigos}

Los enemigos de \juego\ son monstruos abominables de procedencia desconocida.
No obstante, su objetivo es claro: eliminar al gremio de magos y robar la
Reliquia Sagrada, una leyenda de grandioso poder. Cada enemigo tiene unas
características visuales y físicas determinadas (mostradas en la tabla inferior)
aunque su comportamiento es similar. Todos tratan de encontrar a los supervivientes
para acabar con ellos.\\

El \textbf{Goblin} es una criatura de mente muy básica, verde, enana y 
muy desagradable. Sólo va equipado con una tosca espada corta y un burdo
taparrabos. No son extremadamente duchos en combate pero sí cuentan con una
gran velocidad. Su gran ventaja es el número, son capaces de utilizar su
superioridad numérica para atosigar al enemigo y acabar con él. 

\figura{goblin.jpg}{scale=0.3}{Goblin}{fig:goblin}{H}

El \textbf{Diablillo} es una criatura venida del propio averno y bastante
superior al Goblin en lo que a propiedades físicas se refiere. Sus garras, alas
y cola de demonio les basta para atacar desgarrando la carne que se encuentra
a su paso, no necesitan ningún tipo de arma adicional.

\figura{demon.jpg}{scale=0.3}{Diablillo}{fig:demon}{H}

El \textbf{Golem de hielo} es una criatura de gran tamaño venida de las
montañas heladas del norte. Su estructura corporal está formada por piedra
maciza y bloques de hielo. Es una de las criaturas más lentas que existen
pero sus golpes son demoledores.

\figura{golem.jpg}{scale=0.3}{Golem de hielo}{fig:golem}{H}

\begin{table}[H]
  \label{caracteristicas-enemigos}
  \begin{center}
  \begin{tabular}{| c ||m{2cm}|m{2cm}|m{2cm}|}
    \hline
    Nombre del enemigo & Vida & Daño & Velocidad \\
    \hline
    Goblin & 5 & 10 & 5 \\
    \hline
    Diablillo & 7 & 15 & 4.5 \\
    \hline 
    Golem de hielo & 10 & 20 & 2 \\
    \hline
  \end{tabular}
\end{center}
\caption{Comparativa de características de enemigos}
\end{table}

\subsection{Hechizos}

Los hechizos son el único arma de \prota\ para detener el avance de los
enemigos. Sus conocimientos mágicos no le permiten hacer un enorme despliegue
de proyectiles devastadores. Debe administrar cuidadosamente su limitada
energía mágica (maná) para convocar débiles cúmulos de energía.\\

Con \textbf{Bola de fuego} puedes invocar un proyectil mágico que abrase
a tus enemigos a su paso. \textbf{Furia de Gea} convoca la propia fuerza
de la naturaleza para volverla en contra de cualquier criatura. Finalmente,
\textbf{Ventisca} lanza varios proyectiles mágicos helados y afilados como
cuchillas.\\

\begin{table}[H]
  \label{caracteristicas-hechizos}
  \begin{center}
  \begin{tabular}{| c ||m{2cm}|m{2cm}|m{2cm}|}
    \hline
    Nombre del hechizo & Maná & Daño & Velocidad \\
    \hline
    Bola de fuego & 2 & 3 & 8 \\
    \hline
    Furia de Gea & 3 & 6 & 8 \\
    \hline 
    Ventisca & 4 & 8 & 8 \\
    \hline
  \end{tabular}
\end{center}
\caption{Comparativa de atributos de hechizos}
\end{table}


\section{Requisitos mínimos}

Para poder disfrutar de \juego\ sin problemas necesitas disponer de un equipo
con unas características iguales o superiores a las que se listan a
continuación:

\begin{itemize}
    \itemsep0em
    \item \textbf{Sistema operativo}: GNU/Linux, Windows XP Service Pack
    2, Windows Vista o Windows 7.
    \item \textbf{Procesador}: Pentium 2GHz o AMD.
    \item \textbf{Memoria}: 100 MB de memoria RAM disponibles.
    \item \textbf{Tarjeta de vídeo}: 128 MB de memoria y aceleración 3D.
    \item \textbf{Espacio en disco}: 100 MB.
    \item \textbf{Control}: ratón y teclado.
\end{itemize}

En caso de utilizar un sistema operativo GNU/Linux y experimentar problemas,
revisa que tengas instalada la última versión de los drivers para tu
tarjeta gráfica y que éstos soportan aceleración por hardware.

\section{Descarga e instalación}
En esta sección se proporcionan las indicaciones pertinentes para que puedas
descargar e instalar \juego\ en tu sistema. En primer lugar, debes acudir
a la sección de ficheros del repositorio del juego en la forja de RedIRIS:\\

\url{http://forja.rediris.es/frs/?group_id=820}\\

Simplemente descarga la versión de \juego\ que coincida con tu sistema. 
No se han publicado binarios para GNU/Linux, por lo que utilizas dicho sistema,
tendrás que descargar el paquete con el código fuente y compilarlo tú mismo.
En las página \pageref{sec:instalacion-linux} encontrarás más información
al respecto.

\subsection{Instalación en GNU/Linux}
\label{sec:instalacion-linux}

Para poder jugar a \juego\ en un sistema operativo GNU/Linux es necesario
descargar e instalar el entorno de desarrollo completo ya que, por el
momento, no se distribuyen binarios del software. El videojuego cuenta
con las siguientes dependencias:

\begin{itemize}
    \itemsep0em
    \item libogre-dev
    \item libsdl1.2-dev
    \item libsdl-mixer1.2-dev
    \item mygui
\end{itemize}

A lo largo del proceso de instalación, supondremos que se posee una distribución
basada en Debian que dispone de la utilidad \texttt{apt-get}. De no ser así,
el proceso será similar utilizando el programa equivalente para instalar
paquetes en la distribución correspondiente. En primer lugar, Debemos
instalar todas las dependencias de \textsc{Ogre3D}, entre las que
se encuentran el compilador \textit{GCC}, la herramienta \textit{make}
o el generador de makefiles \textit{CMake}. Para ello utilizamos
la siguiente orden:

\begin{lstlisting}[style=consola]
sudo apt-get install g++ make libfreetype6-dev libboost-date-time-dev \
  libboost-thread-dev nvidia-cg-toolkit libfreeimage-dev zlib1g-dev \
  libzzip-dev libois-dev libcppunit-dev doxygen libxt-dev libxaw7-dev \
  libxxf86vm-dev libxrandr-dev libglu-dev cmake
\end{lstlisting}

Una vez instaladas las dependencias podemos dirigirnos a la web oficial
de \textsc{Ogre3D} y descargar el código fuente para sistemas GNU/Linux.\\

\url{http://www.ogre3d.org/download/source}\\

El siguiente paso consiste en descomprimir el paquete y acceder al directorio
resultante utilizando la terminal. La herramienta \textit{CMake} nos ayuda
a generar un fichero \texttt{makefile} dependiente de nuestra plataforma
de manera muy sencilla. Para generarlo el makefile, compilar e instalar
\textsc{Ogre3D} introducimos la siguiente secuencia de órdenes:

\begin{lstlisting}[style=consola]
cmake.
make
sudo make install
\end{lstlisting}

Instalar \textsc{libSDL} y \textsc{libSDL mixer} es muy sencillo puesto
que usualmente se encuentran en los repositorios de la distribución. Simplemente
introducimos:

\begin{lstlisting}[style=consola]
sudo apt-get install libsdl1.2-dev libsdl-mixer1.2-dev
\end{lstlisting}

Finalmente procedemos a la instalación de la biblioteca \textsc{MyGUI}.
Hemos de acudir a su sección de ficheros en la forja SourceForge y descargar
la versión \textit{3.2.0 Release Candidate 1}.\\

\url{http://sourceforge.net/projects/my-gui/files/MyGUI/}\\

Cuando finalice la descarga, descomprimimos el paquete en formato zip y
accedemos al directorio resultante utilizando la terminal. En este caso
también generaremos un makefile de forma automática con \textit{CMake}
para después compilar e instalar a través de \textit{make}.

\begin{lstlisting}[style=consola]
cmake .
make
sudo make install
\end{lstlisting}

Llegados a este punto habremos finalizado instalando las dependencias de
\juego. Ahora simplemente es necesario acceder a su directorio mediante
la terminal y comenzar con la compilación. Una vez finalizada con éxito,
podremos ejecutarlo.

\begin{lstlisting}[style=consola]
make
./siontower
\end{lstlisting}



\subsection{Instalación en Windows}

Tras descargar el paquete, utiliza cualquier programa de compresión/descompresión
que soporte el formato zip. Si no tienes ninguno instalado, puedes emplear
\textit{7Zip} sin problemas ya que es Software Libre. Para descargarlo, dirígete
a la siguiente dirección:\\

\url{http://www.7-zip.org}\\

\juego\ es completamente portable y autocontenido, así que si lo prefieres
puedes guardarlo en un lápiz de memoria o en cualquier disco duro externo.
Para iniciar el juego, haz doble click sobre el fichero ejecutable
\texttt{siontower.exe}.

\section{Guía de juego}

En este bloque el jugador novel podrá aprender cómo se juega a \juego.
Recorreremos las pantallas de la aplicación explicando la funcionalidad de
cada una de forma que aprenderemos a crear y gestionar perfiles de jugadores,
seleccionar nivel, controlar al personaje guiándolo hacia su objetivo y
progresar sumando puntos o desbloqueando nuevos escenarios.\\

\subsection{Menú principal}

Justo al iniciar \juego\ se nos presenta el menú principal del juego con
un paisaje montañoso en el que se ve a la Torre Sagrada siendo atacada
por enemigos en medio de una intensa lluvia.

\figura{siontower-menu.png}{scale=0.3}{Menú principal de Sion Tower}{fig:siontower-menu}{H}

Desde esta pantalla podemos llevar a cabo las siguientes acciones:

\begin{itemize}
    \itemsep0em
    \item \textbf{Jugar}: nos lleva a la pantalla de selección de perfil
    y al camino para comenzar una partida.
    \item \textbf{Créditos}: nos muestra una escena con las personas
    que han participado en el desarrollo.
    \item \textbf{Salir}: cierra el juego por completo, ¿seguro que quieres
    dejar de jugar?
\end{itemize}

\subsection{Selección de perfil}

\juego\ cuenta con un sistema de perfiles para que varias personas puedan
jugarlo y controlar su progreso. Al comienzo de la partida sólo está disponible
el primer nivel. Cuando finalices con éxito un escenario, el
siguiente nivel se convertirá en accesible. Además, cada perfil cuenta con
una puntuación que mide nuestras habilidades en el juego. Esta puntuación
puede subir o bajar en función de nuestra actuación en las partidas. Esto
permite que varios usuarios compitan en una máquina de forma asíncrona
para alcanzar la máxima puntuación.\\

\figura{siontower-perfil.png}{scale=0.3}{Pantalla de selección de perfil en Sion Tower}{fig:siontower-perfil}{H}

La pantalla de selección de perfil consta de dos paneles bien diferenciados.
Selección de un perfil ya existente y creación de un nuevo perfil. En el primer
bloque puedes visualizar la experiencia y los niveles desbloqueados de
cada jugador pulsando con el ratón sobre sus nombres. Si tu
perfil se encuentra en la lista y quieres utilizarlo tienes que seleccionar
el nombre y pulsar el botón \textbf{Aceptar}. Es posible eliminar
un perfil de la lista con el botón \textbf{Eliminar}. No obstante, 
hay que ser cautelosos, puesto que un perfil eliminado no es recuperable.\\

Cuando quieras crear un nuevo perfil, escribe su nombre en el panel
inferior y pulsa sobre \textbf{Crear}. Al menos debes introducir algún
carácter y el nombre elegido no debe figurar entre los perfiles existentes,
de lo contrario se mostrará un error.\\

Para volver al menú principal de \juego\ puedes pulsar sobre el botón
\textbf{Atrás}.


\subsection{Selección de nivel}
\label{sec:levelselection}

Tras seleccionar nuestro perfil se nos presentará la pantalla de selección
de nivel. En la lista de niveles aparecen todos los existentes, ya estén
bloqueados o disponibles. Los bloqueados para nuestro perfil lo indican
en rojo y su icono es translúcido.\\

\figura{siontower-nivel.png}{scale=0.3}{Pantalla de selección de nivel en Sion Tower}{fig:siontower-nivel}{H}

Para seleccionar un nivel, pulsa con el ratón sobre su icono. Si deseas
volver a la pantalla de selección de perfil, pulsa sobre \textbf{Atrás}.

\subsection{Partida}

Tras seleccionar el nivel comienza la partida y los enemigos aparecen
en pequeñas oleadas. El objetivo consiste en eliminarlos a todos utilizando
nuestra energía mágica (maná) para invocar hechizos y mantener nuestra
energía vital por encima de cero.\\

\figura{siontower-juego.png}{scale=0.3}{Partida de Sion Tower}{fig:siontower-juego}{H}

Los controles son los siguientes:

\begin{itemize}
    \itemsep0em
    \item \textbf{Movimiento}: para mover al personaje, utiliza las teclas W, A, S y D.
    \item \textbf{Cámara}: la cámara siempre hace un seguimiento al personaje
    para que nunca salga del encuadre. Puedes acercar o alejar la cámara
    utilizando la rueda del ratón. También puedes moverla alrededor del
    personaje moviendo el ratón mientras pulsas el botón derecho.
    \item \textbf{Hechizos}: para lanzar un hechizo apunta con el ratón
    en la dirección deseada y pulsa el botón izquierdo. Si tienes maná suficiente
    lanzarás el hechizo seleccionado. Para cambiar de hechizo, pulsa
    sobre el que quieras seleccionar en el menú de juego. También puedes
    hacer uso de las teclas de acceso rápido: 1, 2 y 3.
    \item \textbf{Pausa}: para activar la pausa puedes pulsar la tecla
    escape o el botón del menú de juego. En el menú de pausa puedes \textbf{Reanudar}
    la partida, volver a la pantalla de \textbf{Selección de nivel} o dirigirte
    al \textbf{Menú principal}.
\end{itemize}

Si pasas el ratón por encima de un hechizo puedes acceder a más información
sobre el mismo: nombre, descripción, maná necesario y daño que causa. Cuando
no tengas maná suficiente para lanzar un hechizo determinado, éste se mostrará
en gris. La energía mágica se recupera progresivamente de forma automática
cuando no se está haciendo uso de ella. No es posible recuperar vida a lo largo
de la partida.

\figura{siontower-pausa.png}{scale=0.3}{Menú de pausa en Sion Tower}{fig:siontower-pausa}{H}

Si nuestra energía se termina por completo habremos perdido la partida y se
nos avisará mediante un cartel. Mientras decides si volver a intentarlo
puedes seguir moviendo la cámara y observando cómo los enemigos han alcanzado
la victoria. Si pulsas la \textbf{barra espaciadora} volverás a la pantalla
de selección de nivel.

\figura{siontower-derrota.png}{scale=0.3}{Mensaje de derrota en Sion Tower}{fig:siontower-derrota}{H}

\subsection{Pantalla de victoria}

Al eliminar a todos los enemigos de un nivel habrás completado con éxito
el escenario y llegarás a la pantalla de victoria en la que el personaje
celebra su logro. Si no estabas jugando al último nivel disponible, 
habrás conseguido desbloquear el siguiente. Tanto la puntuación obtenida
como los niveles desbloqueados se guardan automáticamente en este punto.

\figura{siontower-victoria.png}{scale=0.3}{Pantalla de victoria en Sion Tower}{fig:siontower-victoria}{H}

La puntuación de \juego\ se obtiene mediante la suma de los siguientes
factores:

\begin{itemize}
    \itemsep0em
    \item \textbf{Enemigos}: por cada enemigo eliminado se obtienen puntos.
    Cuanto más poderoso sea el enemigo, mayor será la recompensa en forma de puntos.
    \item \textbf{Vida}: recibimos un complemento por la energía que nos quede
    al finalizar la partida con éxito.
    \item \textbf{Maná}: en el juego se valora mucho la puntería y la administración
    del maná o energía mágica. Cuanto menos maná utilicemos, mayor será la puntuación.
    Es posible que si se malgasta demasiado, la puntuación sea negativa.
    \item \textbf{Tiempo}: recibimos recompensa en forma de punto por
    eliminar a los enemigos lo antes posible, este apartado también puede
    recibir una valoración negativa.
\end{itemize}

Tras terminar con éxito un nivel puedes elegir la opción \textbf{Volver a jugar},
volver a la pantalla de \textbf{Selección de nivel} o regresar al \textbf{Menú principal}.

\subsection{Créditos}

En la pantalla de créditos aparece una escena de la Torre Sagrada con
un panel listando los participantes en el desarrollo de \juego. Cuando
desees volver puedes pulsar el botón \textbf{Atrás}.

\figura{siontower-creditos.png}{scale=0.3}{Autores de Sion Tower}{fig:siontower-creditos}{H}


\section{Creación de niveles}

¿Tienes una gran idea para un nuevo nivel de \juego? ¿Sabes como puedes
mejorar sustancialmente la historia? ¿Te gustaría que todos viesen tus ideas?
El motor del videojuego permite a los usuarios crear y añadir nuevos niveles
de manera muy sencilla utilizando la herramienta de modelado y animación
3D libre por excelencia: \textit{Blender}. El proceso es largo pero sencillo,
sobre todo si sigues detenidamente todos los pasos de este manual.

\subsection{Instalación de Blender y complementos}

\textit{Blender} es la herramienta de modelado y animación 3D de carácter
completamente libre más potente que existe. Utilizando \textit{Blender}
se han producido en su totalidad cortos de animación como
Big Buck Bunny\footnote{Big Buck Bunny: \url{http://bigbuckbunny.org}},
Sintel\footnote{Sintel: \url{http://sintel.org}} o videojuegos como
Yo Frankie!\footnote{Yo Frankie!: \url{http://yofrankie.org}}. \textit{Blender} es
multiplataforma, no importa si posees Windows, Mac o GNU/Linux ya que no
tendrás problema alguno para utilizarlo. Recientemente se ha publicado la
serie 2.5 de la herramienta, que incluye mejoras como un cambio radical de
la interfaz. No obstante, los scripts de exportación necesarios funcionan
en la 2.49, por tanto, esa sera la versión que utilizaremos.\\

\figura{sintel.jpg}{scale=0.3}{Fotograma del cortometraje Sintel}{fig:sintel}{H}

Si utilizas un sistema operativo GNU/Linux basado en Debian, es probable
que \textit{Blender} se encuentre en los repositorios de tu distribución.
En tal caso, bastaría con introducir en una terminal:

\begin{lstlisting}[style=consola]
sudo apt-get install blender
\end{lstlisting}

En caso de contar con un sistema operativo de tipo Windows, debes acudir
a la sección de versiones anteriores de \textit{Blender} y descargar el
instalador:\\

\url{http://download.blender.org/release/Blender2.49a/blender-2.49a-windows.exe}\\

Simplemente abre el instalador haciendo doble click sobre el ejecutable,
sigue los pasos indicados y habrás instalado Blender con éxito.\\

En ambos sistemas operativos es necesario instalar un plugin de exportación
que convierte una escena creada con Blender a un fichero de texto en formato
xml procesable por el motor de \juego. El formato se conoce como \textit{Dotscene}
y es ampliamente utilizado. Así mismo, es necesario otro complemento que
convierte un modelo creado con \textit{Blender} a un xml. Es posible
descargar ambos plugins desde:\\

\url{http://ogreaddons.svn.sourceforge.net/viewvc/ogreaddons/trunk/blender\
sceneexporter/ogredotscene.py}\\

\url{http://ogre.svn.sourceforge.net/viewvc/ogre/branches/v1-6/Tools/BlenderExport}\\

Para instalar los exportadores en sistemas GNU/Linux, hemos de copiar los ficheros
descargados en:\\

\texttt{~/.blender/scripts}\\

En cambio, si utilizamos Windows el directorio será:\\

\texttt{[Instalación de Blender]/.blender/scripts}\\


\subsection{Composición del nivel}

Para seguir esta guía con soltura es recomendable que poseas unos conocimientos
mínimos sobre \textit{Blender}. Deberías saber cómo manejar los elementos
básicos de la interfaz y la forma de manipular los objetos dentro de la
escena: movimiento, rotación, escala, duplicado, etc. Si nunca has utilizado
\textit{Blender} con anterioridad es recomendable que acudas a algún
manual para principiantes como el libro gratuito bajo Creative Commons
\textit{Aprende Blender en 24 horas}\footnote{Aprende Blender en 24 horas: 
\url{http://www.inf-cr.uclm.es/www/cglez/downloads/blender/24hBlenderYafray.pdf}}
del profesor certificado por la Blender Foundation Carlos González Morcillo.

% Nuevo documento
\subsubsection{Crear un nuevo documento}

El primer paso es crear un nuevo documento de \textit{Blender}, para ello
simplemente abrimos la herramienta por primera vez o hacemos click sobre
File $\rightarrow$ New y aceptamos eliminar la escena actual. Para comenzar
no podemos tener objetos presentes por lo que seleccionamos toda la escena
con A y eliminamos con X.

\figura{blender-1.jpg}{scale=0.35}{Creación de un nuevo documento en Blender}{fig:blender-1}{H}

Es recomendable contar con un espacio de trabajo cómodo, la disposición
de las vistas tridimensionales de \textit{Blender} es fundamental. Para
crear un nuevo nivel lo ideal es contar con las vistas frontal, lateral,
cenital y libre. Puedes ver un ejemplo de esta disposición en la figura
\ref{fig:blender-1}.

% Convenciones de nombrado
\subsubsection{Convenciones de nombrado}

En un escenario de \juego\ se dan cita varios tipos de elementos: objetos
del escenario, geometría arbitraría, efectos de partículas, oleadas de enemigos,
etc. \textit{Blender} no distingue entre ellos, para la herramienta de modelado
todos son objetos tridimensionales comunes. Hemos de seguir unas reglas de
nombrado especiales para que el motor sepa distinguir entre una mesa que
forma parte del mobiliario y un icono simbólico que representa la llegada
de un enemigo.\\

Durante todo el proceso de creación del escenario es importante prestar
atención para que los objetos tengan un nombre acorde con las reglas
establecidas. En la pestaña de edición (tecla F9) aparece el panel
\textit{Link and Materials} mostrando el nombre de la malla del objeto
y el del propio objeto. En nuestro caso, objetos semejantes comparten
la misma malla mientras que el nombre de cada objeto debe ser distinto
y ajustarse a las reglas. Puedes verlo en la figura \ref{fig:blender-3}.

\figura{blender-3.jpg}{scale=0.7}{Nombre de una puerta en un escenario}{fig:blender-3}{H}

Las reglas de nombrado son las siguientes:

\begin{itemize}
    \itemsep0em
    \item \textbf{Escenario}: todos los elementos del escenario (paredes,
    suelo, mesas, sillas y otros muebles) siguen la regla \texttt{scene-nombre.numero}.
    El nombre del objeto es lo que lo identifica dentro del catálogo para poder
    recuperar su modelo de colisión y el número ayuda a hacerlo único dentro de la
    escena (dos objetos no pueden tener el mismo nombre). Ejemplo: \texttt{scene-door.005}.
    \item \textbf{Efectos de partículas}: puedes incluir efectos de partículas
    en cualquier punto de la escena. La regla es \texttt{particle-nombre.numero}.
    Ejemplo: \texttt{particle-flame.001}.
    \item \textbf{Luces}: el motor del juego reconoce las luces de manera automática,
    por lo que no debes preocuparte de sus nombres.
    \item \textbf{Malla de navegación}: la malla de navegación indica a los
    enemigos cuales son las zonas transitables del escenario y siempre
    debe llamarse \texttt{navMesh}. Veremos más sobre las mallas de navegación
    en secciones posteriores.
    \item \textbf{Enemigos}: los enemigos siguen la regla \texttt{enemy-tipo-t.numero}.
    La letra t representa el segundo en el que aparecerá el enemigo desde
    que se inicia la partida. Ejemplo: \texttt{enemy-goblin-25.001}.
    \item \textbf{Protagonista}: el elemento cuyo nombre sea \texttt{player}
    definirá la posición inicial del jugador dentro del nivel.
    \item \textbf{Geometría arbitraria}: toda la geometría que no siga
    la convención de nombrado será tratada como complementos del escenario.
    No se calcularán colisiones contra ellos (se podrán atravesar).
\end{itemize}

% Enlazar objetos
\subsubsection{Enlazar objetos}

Los escenarios del juego están repletos de objetos de distintas clases
y puedes reutilizarlos sin ningún tipo de problemas para tus propias creaciones.
Incluir los objetos ya creados para \juego\ en tu nuevo nivel es muy sencillo.
En el menú de \textit{Blender} seleccionamos File $\rightarrow$ Append or
Link (Image Browser). Se abrirá un pequeño navegador de ficheros para
que seleccionemos el \texttt{objeto.blend} a importar. Al abrirlo, tendremos
que seleccionar Object y el nombre del objeto creado en \textit{Blender}.

\figura{blender-2.jpg}{scale=0.35}{Enlazar con un objeto de Blender existente}{fig:blender-2}{H}

Al enlazarlo se añadirá al origen de coordenadas el objeto deseado. No obstante,
los datos de su malla pertenecen a un fichero externo (contorno azul) y lo
deseable es convertirlo en local. De esta manera nuestro proyecto de nivel
no contará con indeseables dependencias externas. Para hacerlo
seleccionamos el objeto con el botón derecho del ratón, pulsamos L y
seleccionamos la opción All. Una vez localizado, podremos mover el objeto,
escalarlo y rotarlo por la escena con total libertad.

\figura{blender-4.jpg}{scale=0.4}{Convertir en local un objeto enlazado}{fig:blender-4}{H}

% Colocando objetos
\subsubsection{Elementos del escenario}

Los elementos del escenario son aquellos objetos que cuentan con un modelo
de colisión. Esto quiere decir que ni el protagonista ni los hechizos podrán
atravesarlos. Los objetos que venían incluidos en \juego\ ya tienen un modelo
de colisión asignado y para utilizarlos sólo hemos de enlazarlos siguiendo
los pasos de la sección anterior. Estos objetos se encuentran en
\texttt{[siontower]/media/blender} y la lista completa junto a sus códigos
es la siguiente:

\begin{itemize}
    \itemsep0em
    \item Cama (bed)
    \item Candelabro (candle)
    \item Columna (column)
    \item Muro con arco (door)
    \item Muro con puerta (door2)
    \item Muro (wall)
    \item Suelo $8x8 m$ (floor8x8)
    \item Suelo $4x4 m$ (floor4x4)
    \item Reliquia (reliq)
    \item Silla (chair)
    \item Taburete (roundchair)
    \item Estanterías (shelves)
    \item Escaleras (staircase)
    \item Mesa (table)
    \item Caja de madera (woodenbox)
\end{itemize}

Cuando enlacemos un objeto y lo convirtamos en local podemos moverlo utilizando
la tecla G, rotarlo con R o escalarlo mediante S. Las tres operaciones anteriores
pueden restringirse a uno de los tres ejes. Por ejemplo, para desplazar
una mesa a lo largo del eje X será necesario seleccionar la mesa con el botón
derecho del ratón, pulsar G, pulsar X y desplazarla. Es necesario que
no haya desniveles y que el nivel del suelo esté situado en el plano $Z = 0$
de \textit{Blender}\\

Basta con enlazar al principio cada objeto y duplicarlo cada vez que deseemos
uno semejante. En \textit{Blender} es posible duplicar objetos seleccionando
el elemento a duplicar y pulsando SHIFT $+$ D. No obstante, dicha operación
conlleva tener redundancias en los datos de la malla y lo que deseamos es
contar con dos instancias de la misma malla. Para duplicar elementos
utilizaremos la combinación ALT $+$ D. Con estas sencillas instrucciones
ya posees los conocimientos suficientes para componer escenas como la de
la figura \ref{fig:blender-5}.

\figura{blender-5.jpg}{scale=0.35}{Elementos del escenario ya colocados}{fig:blender-5}{H}

% Nuevos objetos
\subsubsection{Nuevos objetos}

Puedes añadir nuevos objetos que no estén en el catálogo modelados por tí
mismo o que hayas encontrado en algún banco de recursos libres. Una vez
modelado y texturizado, será necesario que lo exportes en un formato compatible
con el motor gráfico del juego \textsc{Ogre3D}, el formato binario \texttt{.mesh}.
Puedes obtener más información sobre la exportación en \wiki:\\

\url{http://wikis.uca.es/iberogre/index.php/Exportar_modelos_desde_Blender}\\

Los modelos exportados se sitúan en el directorio \texttt{[siontower]/media/meshes},
los materiales en \texttt{[siontower]/media/materials} y las texturas
en \texttt{[siontower]/media/textures}. El último paso consiste en definir
el modelo colisionable del nuevo elemento del escenario. Debes acudir al
fichero xml \texttt{[siontower]/media/levels/bodycatalog.xml} y editarlo
de forma manual.\\

Cada objeto cuenta con un nombre único y  tiene asociado un cuerpo (body)
compuesto por una o más formas (shapes). Los objetos tienen un tipo, pueden
ser 1 (suelo) o 2 (mobiliario). Cada forma es una figura geométrica
sencilla:

\begin{itemize}
    \itemsep0em
    \item \textbf{Esfera}: \textit{sphere}, definida por un centro y un
    radio.
    \item \textbf{Plano}: \textit{plane}, definido por un punto y un vector
    perpendicular.
    \item \textbf{Caja alineada}: \textit{aabb}, un hexaedro
    alineado con los ejes, definido por un centro y unas distancias hasta
    sus lados. 
    \item \textbf{Caja orientada}: \textit{obb}, hexaedro al que
    se le ha aplicado una rotación, definido por un centro, distancias
    hasta los lados y los ejes sobre los que se apoya.
\end{itemize}

Ha de añadirse una nueva entrada por cada objeto nuevo que se añada si se
desea que cuente con un modelo colisionable. La sintaxis del fichero
xml es bastante sencilla.

\begin{lstlisting}[style=xml]
<?xml version="1.0" encoding="UTF-8" ?>
<bodies>
    
    <body name="floor8x8" type="1">
        <shape type="plane">
            <position x="0" y="0" z="0" />
            <normal x="0" y="1" z="0" />
        </shape>
    </body>
    
    <body name="chair" type="2">
        <shape type="obb">
            <center x="0" y="-0.036" z="0.147"/>
            <extent x="0.2275" y="0.4675" z="0.21"/>
            <axes a00="1" a01="0" a02="0" a10="0" a11="1"
                  a12="0" a20="0" a21="0" a22="1" />
        </shape>
    </body>
    
</bodies>
\end{lstlisting}


% Efectos de partículas
\subsubsection{Efectos de partículas}

Como ya se ha mencionado anteriormente, los efectos de partículas siguen
la nomenclatura \texttt{particle -nombre.numero}. Lo ideal es representarlos
mediante pequeñas esferas de un color relacionado con el efecto que generan.
En \juego\ se incluyen tres tipos de sistemas de partículas por defecto:

\begin{itemize}
    \itemsep0em
    \item \textbf{flame}: llama del tamaño adecuado para ser utilizada en las antorchas.
    \item \textbf{flame2}: llama del tamaño adecuado para colocarse sobre los candelabros.
    \item \textbf{rain}: lluvia intensa utilizada en el menú principal del juego.
\end{itemize}

\figura{blender-6.jpg}{scale=0.35}{Esfera simbolizando un sistema de partículas}{fig:blender-6}{H}

Los sistemas de partículas se definen en scripts de extensión \texttt{.particle}
almacenados en el directorio \texttt{[siontower]/media/particles}. Si conoces
la sintaxis, puedes añadir los que desees para utilizarlos en tus propios niveles.
Hay herramientas como \textit{Ogre Particle Editor}\footnote{Ogre Particle Editor: \url{http://www.ogre3d.org/tikiwiki/OGRE+Particle+Editor&structure=Tools}}
que permite generar dichos scripts utilizando una interfaz gráfica sencilla.

% Luz
\subsubsection{Iluminación}

Añadir luces a una escena es imprescindible, de lo contrario esta aparecería
completamente oscura dentro del juego. Para hacerlo pulsa la barra
espaciadora dentro de la escena 3D y selecciona Add $\rightarrow$ Lamp
y uno de los tipos de luces aceptados por \juego: Lamp, Sun y Spot.
No tienes porqué preocuparte por ninguna convención de nombrado en este caso.
Una vez hayas añadido la luz, puedes moverla por la escena como desees.\\

Es posible configurar varios parámetros de cada punto de luz como su intensidad
y color accediendo al panel de sombreado (F5) y al subpanel de controles
de luz. Lo recomendable es que ninguna luz ascienda de $1.0$ en intensidad
y que los colores sean suaves.

\figura{blender-7.jpg}{scale=0.4}{Panel de configuración de luces}{fig:blender-7}{H}


% Malla de navegación
\subsubsection{Malla de navegación}

La malla de navegación es un conjunto de triángulos conectados que definen
un área transitable para los enemigos. En \juego\ los monstruos no podrían
encontrar el camino hacia el personaje si no fuera por la malla de navegación.
Es similar a contar con un tablero repleto de casillas conectadas. Una vez
tengamos claro dónde se colocarán los obstáculos podemos ir creando la malla.
Recuerda que el objeto debe llamarse \texttt{navMesh} (se distingue entre
mayúsculas y minúsculas). El nombre de la propia malla debe ser único
entre las mallas del juego, lo normal es llamarla \texttt{navMesh-número}
donde número es la posición del nivel en la historia.

\figura{blender-8.jpg}{scale=0.35}{Creación de la malla de navegación}{fig:blender-8}{H}

Para comenzar a crear la malla puedes añadir un plano pequeño con la barra
espaciadora, Add $\rightarrow$ Mesh $\rightarrow$ Plane. Entra en modo edición
utilizando el tabulador y selecciona uno de los vértices con el botón derecho
para eliminarlo con X. Asegúrate de que la malla siempre permanece en el suelo,
es importante para que los enemigos no se eleven misteriosamente. Debe estar
formada exclusivamente por triángulos, no pueden existir cuadriláteros.\\

Extender la malla es sencillo, selecciona un vértice y pulsa E para extruirlo
creando uno nuevo y una arista de unión. Ve repitiendo la operación para
cubrir las zonas transitables del escenario y recuerda dejar un margen
prudencial con los obstáculos. Debes crear caras entre los grupos de tres
vértices que formen los triángulos, para hacerlo selecciona los tres vértices
y pulsa F. Al finalizar es imprescindible que todas las caras de la malla
miren hacia el mismo lado (arriba o abajo), para conseguirlo, selecciona
todos los vértices con A y pulsa Ctrl $+$ N.\\

Finalmente, debes exportar la malla para que el motor de \juego\ pueda
procesarla. Selecciona la malla fuera del modo edición (pulsa el tabulador)
y accede a File $\rightarrow$ Export $\rightarrow$ OGRE Meshes.

\figura{blender-9.jpg}{scale=0.25}{Exportación de la malla de navegación}{fig:blender-9}{H}

El nombre del material es irrelevante porque no nos hará falta, asegúrate
de marcar la opción \textit{Fix Up Axis to Y} y elegir el directorio
\texttt{[siontower]/media} como destino de la exportación.

% Enemigos
\subsection{Enemigos y jugador}

Los enemigos suelen representarse mediante una pirámide de base cuadrada
en posición horizontal. Como ya hemos mencionado, la nomenclatura para los
enemigos es \texttt{enemy-tipo-t.numero} siendo los tipos posibles: \textit{goblin},
\textit{demon} y \textit{golem}. Para que el diseño de niveles sea más cómodo, los goblins
llevan asignado el color verde, los demonios el rojo y los gólems el azul
aunque no es obligatorio de cara a la exportación.\\

Tras añadir un enemigo, puedes duplicarlo utilizando la combinación SHIFT
$+$ D, moverlo con G y rotarlo con R (aunque lo normal es hacerlo sobre
el eje vertical únicamente). La pirámide de los enemigos debe colocarse
atravesando el suelo, de forma que el vértice superior esté en el punto 0
del eje Z.

\figura{blender-10.jpg}{scale=0.35}{Disposición de enemigos en el nivel}{fig:blender-10}{H}

Debes añadir otro elemento simbólico como una esfera o cono para indicar
la posición inicial del jugador. Simplemente has de nombrarlo \texttt{player}.


\subsection{Integración del nivel}

Es conveniente que poco a poco vayas guardando tus progresos diseñando el
nivel ya que no es extraño equivocarse o dejar el trabajo a media. Para ello
selecciona File $\rightarrow$ Save, elige una ruta y presiona el botón de
guardado. A continuación se adjuntan las indicaciones para integrar el nivel
que acabas de crear en \juego.

% Exportación del nivel
\subsubsection{Script de exportación}

Tras haber guardado el nivel como fichero \texttt{.blend} es el momento
de proceder a la exportación a formato \textit{DotScene}. Para ello selecciona
File $\rightarrow$ Export $\rightarrow$ Ogre Scene. Se desplegará un panel
con todos los elementos de la escena, si quieres que todos aparezcan
en el nivel, pulsa sobre All. Recuerda activar la opción \textit{Fix Up Axis
To Y} y selecciona el directorio de niveles como destino de la exportación:
\texttt{[siontower]/media/levels}.

\figura{blender-11.jpg}{scale=0.25}{Exportación de la escena}{fig:blender-11}{H}

Por cada nivel existen dos ficheros xml, el primero es el de la escena
y tiene como nombre \texttt{codigo\_scene.xml}. Por su parte, el segundo
contiene información complementaria como el nombre del nivel, descripción
y música que sonará, su nomenclatura es \texttt{codigo\_info.xml}. Por ejemplo,
si el código del nivel que deseamos exportar es \texttt{level05}, el nombre
del fichero escena debería ser \texttt{level05\_scene.xml}.

% Añadir el nivel al juego
\subsubsection{Añadir el nivel al juego}

% level-info
Para integrar el nuevo nivel dentro de \juego\ debes crear un fichero
\texttt{codigo\_info.xml} en el directorio \texttt{[siontower]/media/levels}.
La sintaxis es muy sencilla, a continuación se muestra el ejemplo para
el primer nivel del juego. El fichero de música debe estar en formato \textsc{Ogg}
y estar situado en \texttt{[siontower]/media/music}.

\begin{lstlisting}[style=xml]
<?xml version="1.0" encoding="UTF-8" ?>
<basicInfo>
    <name>The Hall</name>
    <description>Some enemies have assaulted the main hall of\nthe Sacred Tower. Stop the invasion!</description>
    <song name="nivel1.ogg" group="" />
</basicInfo>
\end{lstlisting}

% Icono
Como ya has visto en la página \pageref{sec:levelselection}, en la pantalla
de selección de nivel, cada escenario viene acompañado de un icono. Si no
haces nada, el hueco quedará vacío para tu nuevo nivel. No obstante, si lo deseas,
puedes añadir un fichero con nombre \texttt{codigo.png} al directorio
\texttt{[siontower]/media/textures}.

% Fichero levels
Finalmente, debes añadir tu nuevo nivel al fichero que indexa los niveles,
\texttt{[siontower]/media/levels/levels.xml}. Por ejemplo, si añadiésemos
el quinto nivel de nuestro ejemplo anterior, el fichero quedaría de la
siguiente manera.

\begin{lstlisting}[style=xml]
<?xml version="1.0" encoding="UTF-8" ?>
<levels>
    <level id="level01"></level>
    <level id="level02"></level>
    <level id="level03"></level>
    <level id="level04"></level>
    <level id="level05"></level>
</levels>
\end{lstlisting}

Con esto habrás conseguido añadir un nivel completamente nuevo y personalizado
a \juego. Para comprobar que todo está correcto, inicia el juego y trata
de seleccionar el escenario. Antes de compartirlo con tus amigos, es recomendable
que compruebes que ni es demasiado complicado ni demasiado sencillo, ¡de lo
contrario no sería divertido!

\end{document}


\addcontentsline{toc}{chapter}{Comunidad y difusión}
\chapter*{Comunidad y difusión}
\label{chap:comunidad}
\section*{Concurso Universitario de Software Libre}

El proyecto \wiki\ y \juego\ ha participado en el V Concurso Universitario
de Software Libre en las categorías de \textit{Comunidad} y \textit{Educación y ocio}.
Se trata de un concurso de software, hardware y documentación libre a nivel
nacional en el que pueden participar grupos de hasta tres estudiantes universitarios,
de bachiller o de ciclos superiores. Se valora el desarrollo del proyecto
desde el comienzo del curso hasta la fase final que tuvo lugar el día
12 de mayo de 2011.\\

En esta edición se presentaron un total de 115 proyectos de diversa índole
y los resultados que obtuvo \wiki\ y \juego\ no pudieron ser más satisfactorios.
En la fase local del concurso fue galardonado con el premio al mejor
proyecto de \textit{Ocio} mientras que en la fase nacional, que se celebró en Granada,
se recibió el premio al mejor proyecto de \textit{Comunidad}.\\

El concurso se ha mostrado como un aliciente de lo más positivo para desarrollar
el proyecto con más entusiasmo y apertura hacia la comunidad. Ha sido muy
beneficioso en términos de audiencia y difusión gracias a los medios
que se han hecho eco de la convocatoria. Sin duda, ha sido uno de los
factores que más me ha motivado a seguir hacia delante tomándomelo como un
reto personal y buscando una experiencia enriquecedora junto al resto
de participantes.\\

\figura{vcusl.jpg}{scale=0.35}{Finalistas del V Concurso Universitario de Software Libre}{fig:vcusl}{h}

\section*{Creación de comunidad}

\wiki\ y \juego\ cuenta con un elevado aspecto de comunidad ya que juntos
forman una plataforma de aprendizaje de desarrollo de videojuegos en tres
dimensiones con \textsc{Ogre3D}. Era imprescindible llevar a cabo acciones
para difundir el proyecto y buscar la creación de dicha comunidad. La comunicación
con los lectores debía ser fluida, directa y cercana tratando de apelar
a su curiosidad e interés por la materia. A continuación, se listan los medios
empleados para contribuir a la difusión del proyecto.\\

\begin{itemize}
    \item \textbf{Blog de desarrollo}
    
    En mi blog personal se ha creado
    una sección especial para informar de los avances del proyecto y documentar
    los subsistemas y algoritmos más relevantes. En total se han redactado
    más de 70 artículos y se han recibido más de 85.000 visitas. Puede accederse
    desde la siguiente dirección.\\
    
    \url{http://siondream.com/blog/category/proyectos/pfc}\\
    
    \item \textbf{Forja}
    
    El proyecto se ha alojado en la forja de RedIRIS la cual no ha sido utilizada
    únicamente por su repositorio \textit{Subversion}. Se ha hecho uso
    de la sección de noticias para informar de los avances importantes,
    de la lista de tareas para gestionar el trabajo pendiente y de la lista
    de ficheros para publicar los resultados (juego y documentación adicional).
    El sitio del proyecto en la forja de RedIRIS puede ser accedido desde
    la siguiente dirección web.\\
    
    \url{https://forja.rediris.es/projects/cusl5-iberogre}\\
    
    \item \textbf{Twitter}
    
    Twitter es una red social de microblogging en la que los usuarios publican
    mensajes cortos. Es ampliamente utilizada para seguir noticias y estar
    informado de la actualidad en diversos sectores muy específicos. Existía
    una comunidad de desarrolladores hispanohablantes muy activa dentro de Twitter
    por lo que se decidió que el proyecto tuviera presencia en dicha red social.
    La comunicación en Twitter ha sido muy útil para acercarnos a los lectores
    y recibir sus sugerencias. Actualmente la cuenta $@$IberOgre posee 98
    seguidores y puede accederse desde:\\
    
    \url{http://twitter.com/#!/IberOgre}\\
    
    \item \textbf{Web en la forja}
    
    La forja de RedIRIS proporciona a los proyectos un pequeño espacio web.
    Si bien no otorga libertad absoluta para publicar contenido (solo
    se permiten webs estáticas en HTML) cuenta con un posicionamiento
    extremadamente favorable en buscadores. Se ha aprovechado dicho espacio
    con una web a modo de índice indicando brevemente en qué consiste el
    proyecto y enlazando a los medios oficiales. Puede verse en la figura
    \ref{fig:webproyecto} y accederse desde la siguiente dirección.\\
    
    \url{http://cusl5-iberogre.forja.rediris.es}\\
    
    \item \textbf{Canal de Youtube}
    
    Durante todo el desarrollo se han ido subiendo los progresos de \juego\
    al servicio de vídeo vía streaming por excelencia. Esto ha permitido
    que el interés por el proyecto crezca progresivamente. En total se han
    publicado diez vídeos los cuales se han reproducido en más de 2.800 ocasiones.
    Puede accederse al canal correspondiente en la siguiente dirección.\\
    
    \url{http://www.youtube.com/user/davidsaltares}\\
    
\end{itemize}

\figura{webproyecto.jpg}{scale=0.35}{Web del proyecto en la forja RedIRIS}{fig:webproyecto}{h}

\section*{Difusión}

Desde el momento en el que \wiki\ comenzó a contar con varios artículos
publicados, el público comenzó a tomar interés en el proyecto. Por supuesto,
el mayor empuje se produjo tras las fases local y final del V Concurso
Universitario de Software Libre. A continuación, haremos un repaso por los
medios que se han hecho eco del proyecto.\\

\begin{itemize}
    \item \textbf{Comunidades de desarrollo}
    
    Diversas comunidades de desarrollo de videojuegos en español mostraron
    rápidamente su interés principalmente en \wiki\ por proporcionar
    documentación en su idioma nativo. A continuación, enlazamos a algunos
    de los medios que han publicado artículos sobre el proyecto.\\
    
    \url{http://www.creagames.es/iberogre-un-proyecto-espanol-de-ogre-engine}\\
    \url{http://razonartificial.com/2011/03/iberogre-documentacion-de-ogre-en-espanol}\\
    \url{http://programandoideas.com/2011/01/iberogre-tutoriales-de-ogre3d-en-espanol}\\
    
    \item \textbf{Web oficial de Ogre3D}
    
    \wiki\ aparece mencionada en la cuarta edición de las noticias relacionadas
    con la comunidad de \textsc{Ogre3D}. La plataforma educativa fue visible
    en la portada de la web oficial del motor.\\
    
    \url{http://www.ogre3d.org/2011/03/01/ogre-news-4}\\
    
    \item \textbf{Twitter}
    
    El propio creador de \textsc{Ogre3D}, Steve Streeting, encontró la cuenta
    oficial de Twitter de \wiki\ y la recomendó públicamente. Desde entonces
    el número de seguidores y lectores aumentó considerablemente.\\
    
    \item \textbf{Prensa}
    
    Tras el V Concurso Universitario de Software Libre varios medios
    de prensa escrita publicaron una noticia al respecto. El proyecto
    apareció en Viva Cádiz, Diario de Cádiz, La Voz, la web oficial
    de la Universidad de Cádiz y varios medios más.\\
\end{itemize}


\clearpage
\addcontentsline{toc}{chapter}{Bibliografia y referencias}
\bibliographystyle{plain}
\bibliography{bibliografia}




%\addcontentsline{toc}{chapter}{Instalación de \LaTeX}
%\chapter*{Instalación de \LaTeX}
%\input{instalacion.tex}

\input{fdl-1.3.tex}

\end{document}
