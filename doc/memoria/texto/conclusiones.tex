Tras haber explicado todo el proceso de desarrollo de \wiki\ y \juego\
a lo largo del presente documento, en esta sección haremos una valoración
acerca del proyecto. En primer lugar hablaré de como me ha afectado en el plano
personal y académico para después tratar los aspectos técnicos. Finalmente
finalizaremos con las posibles ampliaciones que podría sufrir en el futuro.\\

\section{Conclusiones personales}

Con \wiki\ y \juego\ he adquirido una gran cantidad de conceptos y me he
enriquecido muchísimo como persona y desarrollador. Si bien ya había
trabajado de forma independiente en algún videojuego sencillo o proyecto
de distinta índole, este sin duda era el reto de mayor envergadura al que
me había enfrentado jamás. De ahí que la fase de aprendizaje fuese tan
larga al comienzo del proyecto aunque después se extendiera durante
todo este tiempo.\\

En los siguientes puntos repasaré de forma superficial lo que he aprendido:

\begin{enumerate}
    \itemsep0em
    \item \textbf{Juegos 3D}: hasta el momento había desarrollado juegos
    sencillos en dos dimensiones pero no conocía el mundo de las tres dimensiones.
    Consideré que el Proyecto Fin de Carrera me brindaba una oportunidad
    excelente para aprender y me decidí a intentarlo. Es una aproximación
    muy diferente, atractiva y llena de retos. Es necesario conocer todo
    un conjunto de técnicas y conceptos totalmente nuevos y me ha resultado
    de lo más interesante.\\
    
    \item \textbf{Nuevas bibliotecas}: he hecho uso de muchas bibliotecas
    que no conocía y he necesitado aprender a utilizarlas correctamente.
    Entre estas bibliotecas se encuentran la propia \textsc{Ogre3D},
    \textsc{MyGUI}, \textsc{Boost} o \textsc{pugixml}. Anteriormente
    había utilizado \textsc{libSDL mixer} pero nunca me había visto obligado
    a integrarla dentro de un sistema más grande como es la gestión
    de recursos de \textsc{Ogre3D}. \textsc{Boost} es una biblioteca de
    lo más versátil y potente que complementa en muchos aspectos a la
    estándar de C++. De hecho, muchas de sus características aparecerán
    en el nuevo C++ 0x.\\
    
    \item \textbf{Matemáticas para videojuegos}: no sólo he tenido que repasar
    conceptos matemáticos de geometría del espacio y álgebra sino que he
    tenido que incorporar otros nuevos. Tanto de cara a \wiki\ como para
    \juego\ me he visto obligado a aplicar estos conceptos con el objetivo
    de buscar soluciones a problemas de programación.\\
    
    \item \textbf{Lenguaje C++}: desde el segundo curso de Ingeniería
    Técnica en Informática de Gestión he estado utilizando C++ no sólo
    dentro del ámbito estrictamente académico. No obstante, desarrollar
    \juego\ me ha servido para profundizar en los detalles del lenguaje
    y aprender a utilizarlos a mi favor.\\
    
    \item \textbf{Python}: este lenguaje de scripting orientado a objetos
    se ha mostrado extremadamente útil a la hora de desarrollar
    pequeños programas auxiliares. Por ejemplo, lo he utilizado en el script
    que soluciona el problema de la escala en la exportación de modelos
    tridimensionales y en el que extrae las cadenas traducibles de plantillas
    de interfaz. Es un lenguaje extremadamente sencillo y fácil de aprender
    pero muy potente y que agiliza el desarrollo de herramientas simples.\\
     
    \item \textbf{Optimización}: los videojuegos son sistemas complejos
    que, de no prestar especial atención, podrían hacer un consumo irracional
    de recursos. Es necesario aplicar técnicas de optimización tanto en
    términos de uso de procesador como de consumo de memoria. Este proyecto
    me ha servido para conocer varias de estas técnicas y algoritmos concretos
    cuyo rendimiento es superior al de otros en momentos concretos. Por ejemplo,
    en primer lugar empleé el algoritmo A* para la búsqueda de caminos
    aunque la precomputación con Floyd probó ser más eficiente.\\
    
    \item \textbf{Técnicas de IA}: hasta el momento conocía algunas técnicas
    de inteligencia artificial sobre teoría de juegos, búsqueda de caminos
    o autómatas para modelar comportamientos. No obstante, no había aplicado
    dichas técnicas al mundo de las tres dimensiones. En \juego\ se realiza
    una búsqueda de caminos a partir de una malla definida en tiempo de diseño.
    Además, los enemigos hacen uso de algoritmos de movimiento (
    \textit{Steering Behaviors}) que desconocía hasta este momento.\\
     
    \item \textbf{Diseño de un videojuego}: los videojuegos que había
    desarrollado con anterioridad con contaban con un documento de diseño
    en el que se detallara la forma de jugar, personajes, historia, etc.
    En este caso ha sido muy necesario ya que ha ayudado al proceso de análisis
    y diseño. Además, los colaboradores han podido conocer las necesidades
    del proyecto en cuanto a recursos artísticos se refiere. Los integrantes
    del equipo conocíamos en todo momento el estilo de juego y la apariencia
    que debía tener \juego.\\
    
    \item \textbf{Trabajo en equipo}: en \wiki\ se ha trabajado junto
    a la comunidad en todo momento. Los lectores enviaban sus opiniones y
    éstas debían ser tenidas en cuenta. En \juego\ el trabajo en equipo
    se hizo mucho más evidente ya que colaboramos en todo momento seis
    personas especializadas en disciplinas muy distintas. Hubo que realizar
    labores de coordinación, comunicación y resolución de conflictos
    (sobre todo relacionados con los formatos empleados). Una experiencia
    muy enriquecedora y que, sin duda, me ayudará en mi futuro profesional.\\ 
    
    \item \textbf{Redacción en MediaWiki}: hasta el momento mi relación
    con el motor \textit{MediaWiki} se había limitado a ediciones muy
    esporádicas en Wikipedia que no requerían conocimientos de la sintaxis
    que se utilizaba. Me he visto obligado a conocer la sintaxis de redacción
    en \textit{Wikimedia} para redactar correctamente los artículos.\\
    
    \item \textbf{Trabajo en wikis}: no sólo basta con conocer la sintaxis
    de \textit{Wikimedia} para poder participar en la confección de una wiki.
    Existen toda una serie de convenciones de escritura, comportamiento,
    nomenclatura y estructuras que deben ser conocidas.\\
    
    \item \textbf{Aplicación de conocimientos}: el desarrollo de este
    proyecto me ha sido especialmente útil para poner en práctica aquellos
    conocimientos adquiridos durante la Ingeniería Técnica. Sobre todo
    me refiero a aquellos relacionados con la Ingeniería del Software.\\
\end{enumerate}

\section{Conclusiones técnicas}

En \wiki\ y \juego\ se han cumplido con los objetivos propuestos en el
capítulo introductorio de esta memoria de Proyecto Fin de Carrera. En la
plataforma de aprendizaje se ha conseguido:

% Objetivos cumplidos con IberOgre
\begin{itemize}
    \itemsep0em
    \item Se ha creado contenido organizado en bloques temáticos tal y como
    se propuso: introducción, matemáticas para videojuegos, \textsc{Ogre3D},
    otras tecnologías y videojuegos desarrollados con \textsc{Ogre3D}.
    \item La navegación es intuitiva y los artículos están ordenados
    de forma aproximada en dificultad ascendente.
    \item Es posible adquirir los conocimientos de geometría del espacio
    necesarios para desarrollar juegos en 3D.
    \item Se han cubierto los aspectos básicos del uso del motor de renderizado
    \textsc{Ogre3D}.
    \item Otras tecnologías enfocadas al desarrollo de videojuegos han sido
    explicadas a través de varios artículos.
    \item Los artículos están dotados de un enfoque práctico gracias a los
    ejemplos finales y a los pequeños fragmentos de código intermedios.
    \item Varios usuarios han mostrado interés, han colaborado con el proyecto
    ya sea mediante correcciones, artículos, sugerencias o ayudando a difundir
    la plataforma.
\end{itemize}

% Objetivos cumplidos con Sion Tower
En el videojuego \juego\ hemos conseguido:

\begin{itemize}
    \itemsep0em
    \item Construir un videojuego completo empleando los conocimientos de
    \wiki\ y otros conceptos profusamente documentados.
    \item Creación de un motor orientado a la creación de contenido. Es posible
    crear nuevos niveles complejos sin necesidad de tocar una sola línea
    de código.
    \item Aplicación multilenguaje gracias a \textsc{gettext}.
    \item Creación de un videojuego que entretiene gracias a las impresiones
    que han prestado los colaboradores a lo largo de todas las fases del
    desarrollo.
    \item Implementación de un motor modular fácilmente ampliable. Varios
    subsistemas han sido liberados de forma independiente y están siendo
    utilizados por otros usuarios.
\end{itemize}

Se han generado estadísticas sobre el uso del repositorio \textit{Subversion}
empleando la herramienta libre \textit{StatsSVN} \cite{website:statssvn}.
En total el proyecto está compuesto por 619.000 líneas de código aunque
entre ellas se encuentran las distintas ramas (con código duplicado)
y líneas de documentación (esta memoria, por ejemplo). Cabe destacar
que se han realizado más de 500 commits, lo que permite volver hacia atrás
de forma sencilla. Es posible acceder al informe de estadísticas desde
la siguiente dirección.\\

\url{http://siondream.com/iberogre-siontower-statsvn}\\

\section{Trabajos futuros}

Es cierto que los objetivos que nos marcamos al comienzo del desarrollo
han sido cumplidos satisfactoriamente, no obstante se han detectado puntos
en los que el proyecto podría mejorar. A continuación hacemos una lista
de las posibles mejoras de \wiki.

% Mejoras en IberOgre
%   - Artículos en Ogre
%   - Artículos en otras tecnologías
%   - Artículos en videojuegos
\begin{itemize}
    \item Consolidación de una comunidad de redactores y lectores para
    la plataforma de pruebas.
    \item Nuevos artículos sobre videojuegos desarrollados con \textsc{Ogre3D}
    en los que el propio desarrollador comente la experiencia del desarrollo
    y proporcione o enlace documentación de interés.
    \item Artículos en la sección \textsc{Ogre3D} que documenten los sistemas
    de \textit{shading} con los que cuenta el motor.
    \item Documentar alguna biblioteca o sistema para incluir juego en red en
    la sección de otras tecnologías.\\
\end{itemize}

En \juego\ podrían incluirse las siguientes mejoras:

% Mejoras en Sion Tower
%   - Más hechizos
%   - Experiencia, subida de nivel, aumento tde características
%   - Idiomas adicionales
%   - IA distinta para cada enemigo

\begin{itemize}
    \item Nuevos hechizos: paralización, teletransporte para huir, colocación
    de trampas, etc.
    \item Mayor relevancia de los puntos de experiencia, podrían permitir
    subir de nivel, mejorar la energía y el maná así como desbloquear
    hechizos.
    \item Enemigos adicionales como arañas gigantes o criaturas voladoras
    con comportamientos diferentes.
    \item Nuevos niveles de mayor tamaño que permitan la aplicación de
    técnicas de búsqueda de caminos jerárquicas.
    \item Escenas narrativas mediante animaciones que fueran contando
    la historia entre niveles.
\end{itemize}
