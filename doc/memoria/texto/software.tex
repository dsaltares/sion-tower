En esta sección hablaremos de las herramientas utilizadas durante el desarrollo
de \wiki\ y \juego. Para cada herramienta ofreceremos una pequeña descripción
de sus funcionalidades y adjuntaremos las razones por las cuales ha sido
elegida para el desarrollo frente a sus competidoras. No haremos mención
a las bibliotecas empleadas ya que han sido profusamente comentadas a lo largo
de la implementación de ambas partes del proyecto.\\


\section*{Subversion}

\textit{Subversion} \cite{website:svn} es un sistema de control de
versiones libre cuya primera versión fue lanzada en el año 2000 por CollabNet.
Es compatible en mayor parte con su predecesor \textit{CVS}. \textit{Subversion}
nos permite contar con una copia de seguridad tanto de los ejemplos de
\wiki\ como del código de \juego\ en todo momento. Gracias a esta herramienta
podemos guardar un historial de todas las versiones de los ficheros
fuente del proyecto así como deshacer cambios en caso de que fuera necesario.
Con \textit{SVN} conseguimos acceso al código del proyecto desde cualquier
equipo. Además, pone a disposición de cualquier interesado el código fuente
de forma sencilla.\\

Existen otros sistemas de control de versiones como \textit{Git} o
\textit{Mercurial}. Se ha elegido \textit{Subversion} por su sencillez
y por mostrarse suficientemente potente para gestionar el código de un proyecto
de un sólo programador. En cualquier caso, su uso era obligatorio según la
normativa del V Concurso Universitario de Software Libre (más sobre
el concurso en el apéndice \nameref{chap:comunidad}).\\

\section*{GNU GCC Compiler}

\textit{GCC} \cite{website:gnu-gcc} es el compilador del proyecto GNU y se
encarga de traducir nuestro código C++ a lenguaje máquina para poder
ejecutarlo posteriormente. El proyecto se inició en 1987 y actualmente
está disponible para multitud de arquitecturas como móviles Symbian
o consolas PlayStation 2.\\

\textit{GCC} fue desarrollado inicialmente para soporta el lenguaje
C aunque actualmente forma el GNU toolchain y es compatible con C++,
Fortran, Pascal, Objective-C, Java y Ada entre otros. Ha sido elegido
por ser el compilador libre compatible con el lenguaje más ampliamente
usado, estable y eficiente.\\

\section*{Make}

\textit{Make} \cite{pdf:make} es una utilidad para automatizar el proceso
de programas y bibliotecas a partir de su código fuente a partir
de ficheros con una sintaxis especial llamados \textit{makefile}. Estos
ficheros le indican a la herramienta cómo ha de ser compilado el software.\\

\textit{Make} acelera el proceso de compilación ya que no vuelve a generar
los ficheros objetos ya creados cuyos fuentes no han sido modificados. Además,
permite limpiar directorios y gestionar distintos modos de compilación
(modo depuración o lanzamiento, por ejemplo). En nuestro proyecto, \textit{Make}
también es utilizado para generar la documentación escrita en \LaTeX.\\

\section*{GNU Debugger}

\textit{GDB} \cite{website:gdb} es como se conoce popularmente al depurador de código del
proyecto GNU. Es un depurador destinado a sistemas Unix compatible con varios
lenguaje entre los que se encuentran, por supuesto, C y C++. Esta herramienta
nos permite marcar puntos de interrupción en el programa y examinar la pila
del programa cuando éste ha terminado su ejecución de manera abrupta. Es
posible acceder al valor de variables en todo momento para tratar de averiguar
las fuentes de los fallos del sistema.\\

El depurador básicamente nos ayuda a saber qué ocurre exactamente durante
la ejecución de nuestra aplicación. En \juego\ ha resultado ser una herramienta
imprescindible a la hora de identificar y solucionar problemas en tiempo
de ejecución. Sin duda, es una herramienta que colabora en el aumento de
la calidad del software si sabe utilizarse correctamente. Cuenta con un
sinfín de opciones y funcionalidades que no han sido explotadas en el
desarrollo de este proyecto.\\

\section*{Valgrind}

\textit{Valgrind} \cite{website:valgrind} es una herramienta para la
depuración del uso de la memoria, detección de pérdidas de memoria y
profiling. Esto nos permite identificar puntos en los que se accede a memoria
aún no inicializada y que probablemente contenga basura. También es útil
para saber si nuestra aplicación no la libera memoria utilizada por los
objetos cuando estos ya han sido destruidos. El profiling consiste en
monitorizar cuánto tiempo permanece el código en ejecución dentro de cada
función, esto nos ayuda a identificar posibles cuellos de botella. Su nombre
viene de la mitología nórdica y se refiere a la entrada del Valhala.\\

Es compatible con sistemas GNU/Linux y Mac OS X. Ha resultado extremadamente
útil en aquellos puntos a los que \textit{GDB} no era capaz de llegar. No
obstante, al ser una herramienta de análisis dinámico, la ejecución
de \juego\ se ralentizaba en exceso aún trabajando con equipos muy potentes.\\

\section*{Vim}

\textit{Vim} \cite{website:vim} es un editor de textos libres creado en
1991 para el computador Amiga. \textit{Vim} significa \textit{Vi IMproved}
ya que es una versión mejorada del anterior editor \textit{Vi}. Es editor
compatible con sistemas GNU/Linux extremadamente ligero que no precisa
de entorno gráfico ya que puede ejecutarse desde la terminal. Está orientado
a la productividad y cuenta con decenas de combinaciones de teclas para
acelerar la escritura de código.\\

Es muy completo, cuenta con corrector ortográfico, resaltado de sintaxis
para decenas de lenguajes e ingentes opciones para personalizar el entorno.
Incluso es posible ampliar sus capacidades mediante extensiones. Ha sido utilizado
para el proyecto por su ayuda a la productividad y ligereza. Puede observarse
\textit{Vim} siendo ejecutado en una terminal en la figura \ref{fig:vim}.\\ 

\figura{vim.jpg}{scale=0.25}{Editor de textox Vim}{fig:vim}{h}


\section*{LaTeX}

\LaTeX \cite{website:latex-wikibooks} es lenguaje de marcado y un sistema
de creación de documentos especialmente orientado al mundo científico y
técnico. El lenguaje \textit{Tex} fue concebido por Donald Knuth durante
los 80 y en 1984 Leslie Lamport creó \LaTeX como un framework para trabajar
con cartas, libros y otro tipo de textos.\\

Los resultados que producen \LaTeX son coherentes, ordenados y muy limpios.
Si bien aprender su uso puede ser complejo, los resultados son de una
enorme calidad si los comparamos con los documentos que producen los editores
convencionales como \textit{Microsoft Word} o \textit{LibreOffice Writer}.
Por esta calidad y posibilidad de automatizar formatos ha sido elegido
\LaTeX para la redacción de la documentación del proyecto.\\

\section*{Doxygen}

\textit{Doxygen} \cite{website:doxygen} es una herramienta de documentación
automática de código. Mediante la inclusión de comentarios especiales
dentro de los ficheros de cabecera de nuestro proyecto, es capaz
de generar documentación con una apariencia atractiva tanto en formato HTML como
en \LaTeX o muchos otros. Es compatible con los lenguajes C, C++, C\#,
Fortran, Java, Objective-C, PHP, Python, IDL y algunos más.\\

Con \textit{Doxygen} se produce una documentación legible por usuarios (con
los conocimientos necesarios) u otros miembros del equipo. Es posible incluir
diagramas de colaboración y herencia gracias a su uso de graphviz. Ha sido
elegida como herramienta de documentación de código por su sencillez de uso,
limpieza y por su amplia aceptación.\\


\section*{Blender}

\textit{Blender} \cite{hess09} es una herramienta libre de modelado y animación
en tres dimensiones multiplataforma. Es compatible con Windows, Mac OS X,
GNU/Linux y otros sistemas operativos. Cuenta con funcionalidades avanzadas
de modelado 3D, mapeado UV para las texturas, renderizado, texturizado, 
animación basada en esqueletos, efectos de partículas y simulaciones físicas.\\

Integra un motor de físicas y colisiones y proporciona una API compatible
con Python para programar complementos e incluso videojuegos completos
gracias al \textit{Blender Game Engine}. Ha sido utilizado en el desarrollo
de cortometrajes libres como Big Buck Bunny aunque también se emplea de forma
profesional en la industria de la publicidad.\\

Ha sido elegido para ser empleado en el proyecto por ser el editor 3D
libre por excelencia y por haber sido probado en entornos de producción
muy importantes. Además de ser utilizado para diseñar los elementos
del escenario, se emplea como editor de niveles completo.\\

\figura{blender.jpg}{scale=0.2}{Herramienta de modelado y animación 3D Blender}{fig:blender}{h}

\section*{GIMP}

\textit{GIMP} \cite{website:gimp} es el editor de imágenes libre del proyecto
GNU, de hecho su nombre es un acrónimo de \textit{GNU Image Manipulation
Program}. Es multiplataforma y está disponible para Windows, GNU/Linux y Mac OS X.
No es comparable a soluciones privativas como \textit{Adobe Photoshop} pero
es capaz de realizar operaciones bastante avanzadas de forma sencilla.
Su interfaz puede verse en la figura \ref{fig:gimp}.\\

Ha sido utilizado en \juego\ para trabajar con las texturas del escenario.
Hemos elegido esta herramienta por contar con una licencia libre, ser
lo suficientemente potente para nuestras necesidades y estar disponible
en varias plataformas (incluida GNU/Linux).\\

\figura{gimp.jpg}{scale=0.22}{Editor gráfico Gimp}{fig:gimp}{h}

\section*{Inkscape}

\textit{Inkscape} \cite{website:inkscape} es una herramienta libre multiplataforma
para trabajar con gráficos vectoriales. Estos diseños no están basados en píxeles
sino en elementos geométricos como vectores o figuras sencillas. Esto permite
un escalado ilimitado sin pérdida de calidad. Inkscape trabaja con ficheros
en formato \textit{SVG}.\\

Ha sido utilizado tanto en \wiki\ como en \juego\ para trabajar con gráficos
bidimensionales tales como elementos de interfaz o diagramas más vistosos.
Es prácticamente la única alternativa libre a la altura del software comercial
existente.\\

\figura{inkscape.jpg}{scale=0.25}{Herramienta de gráficos vectoriales Inkscape}{fig:inkscape}{h}


\section*{MyGUI Layout Editor}

La biblioteca de interfaces \textsc{MyGUI} utiliza unas plantillas en formato
\textit{XML} para definir los elementos de las pantallas como paneles, botones
o imágenes. Es posible escribir estas plantillas manualmente pero es una tarea
ardua. Afortunadamente existe \textit{MyGUI Layout Editor}, una herramienta
para diseñar interfaces y guardarlas en dicho formato de manera que el motor
pueda cargarlas de forma sencilla posteriormente.\\

Es multiplataforma, libre y trabaja de una manera muy limpia ya que no genera
código de ningún tipo, únicamente produce un \textit{XML}. Es muy intuitiva
y gracias a ella se han diseñado todas las pantallas de \juego. La última
versión cuenta con la siguiente interfaz (figura \ref{fig:layouteditor}).\\

\figura{layouteditor.jpg}{scale=0.20}{Editor de plantillas MyGUI Layout Editor}{fig:layouteditor}{h}

\section*{Particle Editor}

Los sistemas de partículas de \textsc{Ogre3D} se definen en scripts con
una sintaxis especial. Al igual que ocurre con la interfaz, escribirlos a mano
es muy pesado y no permite ver los resultados hasta que se inicie el juego.
\textit{Ogre Particle Editor} es un editor de sistemas de partículas para
\textsc{Ogre3D}. Nos permite crear partículas y modificar sus parámetros
viendo los resultados en tiempo real.\\

Una vez hayamos terminado, guarda el resultado en un fichero con la sintaxis
anteriormente mencionada. Esto acelera enormemente el proceso de creación
de efectos de partículas y por ello esta herramienta ha sido utilizada
en \juego.\\

\figura{particleeditor.jpg}{scale=0.4}{Editor de sistemas de partículas Particle Editor}{fig:particleeditor}{h}


\section*{Audacity}

\textit{Audacity} \cite{website:audacity} es un editor de audio libre y
multiplataforma. Nos permite grabar y modificar audio con un gran número
de opciones y variantes. Fue lanzado por primera vez en mayo del 2010
y actualmente cuenta con más de 72 millones de descargas. Si bien es cierto
que carece de herramientas avanzadas de edición de sonido, para nuestras
necesidades era la herramienta ideal.\\

\textit{Audacity} ha sido utilizado en \juego\ para convertir y retocar
los efectos de sonido y las pistas de la banda sonora que envían los
correspondientes artistas. Su interfaz puede observarse en la figura
\ref{fig:audacity}.\\

\figura{audacity.jpg}{scale=0.25}{Editor de audio Audacity}{fig:audacity}{h}

\section*{XVidCap}

\textit{XVidCap} \cite{website:xvidcap} es una utilidad libre para capturar
la pantalla en vídeo y audio. Suele utilizarse para crear tutoriales de procesos
complejos aunque en \juego\ se emplea para tomar los vídeos con el objetivo
de difundir el proyecto. Puede capturar el escritorio completo, ventanas
concretas o un área determinada. Nos permite calibrar el formato de vídeo
y audio, la calidad así como el número de cuadros por segundo.\\

Se trata de la herramienta para capturar vídeo del escritorio más eficiente
y que menos peso ejercía sobre el rendimiento general del sistema. Su interfaz
es muy sencilla y únicamente consta de la barra que puede verse en la figura
\ref{fig:xvidcap}.\\

\figura{xvidcap.jpg}{scale=0.5}{Capturador de pantalla XvidCap}{fig:xvidcap}{h}

\section*{OpenShot Video Editor}

\textit{OpenShot Video Editor} \cite{website:openshot} es un sencillo pero
potente editor de vídeo para sistemas GNU/Linux. Soporta multitud de formatos
de audio y vídeo gracias a su uso de la biblioteca \textsc{ffmpeg}. Creamos
el vídeo añadiendo y organizando recursos sobre las pistas que deseemos.
Es posible incluir transiciones, subtítulos, carteles y otros efectos.
Podemos configurar la exportación de vídeo a través de numerosos parámetros
e incluso se proporcionan perfiles para subir vídeos a conocidos servicios
como \textit{Youtube}.\\

Los vídeos de \juego\ creados con \textit{XVidCap} son procesados con
\textit{OpenShot Video Editor} y procesados para ser subidos a algún
servicio de vídeos por streaming con el objetivo de difundir el proyecto.
Es la herramienta libre más potente y sencilla disponible para GNU/Linux.\\

\figura{openshot.jpg}{scale=0.25}{Editor de vídeo OpenShot}{fig:openshot}{h}

\section*{Planner}

\textit{Planner} \cite{website:planner} es una herramienta libre para entornos
GTK que nos permite planificar y realizar un seguimiento de cualquier proyecto.
Se definen tareas sobre un calendario especificando sus dependencias, duraciones
y fechas límite. Posteriormente, se crean y asignan recursos (materiales
o humanos) a dichas tareas. Finalmente se obtiene un diagrama de Gantt con
la organización temporal del proyecto. Las planificaciones pueden guardarse
en una base de datos \textit{postgresql} o en un sencillo formato \textit{XML}.
Incluso puede exportarse a \textit{HTML} para que pueda ser visualizada
desde cualquier navegador.\\

Toda la planificación del proyecto que puede verse en el capítulo
\ref{chap:calendario} ha sido creada con \textit{Planner}. Es una herramienta
muy intuitiva y mucho más ligera que alternativas como \textit{GanttProject}.\\

\figura{planner.jpg}{scale=0.28}{Planificador de proyectos Planner}{fig:planner}{h}

\section*{BOUML}

\textit{BOUML} \cite{website:bouml} es una herramienta libre para diseñar
diagramas siguiendo la notación \textit{UML}. Cuenta con utilidades de generación
automática de código a partir de los diagramas en lenguajes C++, Java, PHP,
Python e IDL. Es multiplataforma y mucho más ligero que herramientas similares
como \textit{Umbrello}. Permite la reutilización de clases entre diagramas
y cuenta con una amplia variedad de posibilidades: diagramas de clases, 
de interacción, de secuencia, de casos de uso, etc.\\

Tanto en el análisis (capítulo \ref{siontower-analisis}) como el diseño
(capítulo \ref{sec:siontower-diseno}) ha participado la herramienta \textit{BOUML}.
Su interfaz no es elegante pero resulta funcional y sencilla de utilizar,
puede verse en la figura \ref{fig:bouml}.\\

\figura{bouml.jpg}{scale=0.25}{Editor de diagramas con notación UML BOUML}{fig:bouml}{h}

\section*{Dia}

\textit{Dia} \cite{website:dia} es una herramienta libre para crear diagramas
de propósito general. Permite varios tipos como notación \textit{UML},
diagramas de flujo, entidad relación, circuitería y un largo etcétera. Una vez
hayamos terminado de trabajar en nuestro diagrama, podemos exportarlo a
varios formatos como JPG, PNG o SVG.\\

Si bien es mucho menos potente que \textit{BOUML} permite realizar otros
tipos de diagramas. Concretamente, en \juego\ ha sido empleado para esquemas
de flujo como la máquina de estados que puede verse en la figura \ref{fig:dia}.\\

\figura{dia.jpg}{scale=0.28}{Herramienta de creación de diagramas}{fig:dia}{h}
