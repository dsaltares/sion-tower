\documentclass[green]{beamer}

\usepackage[utf8]{inputenc}
\usepackage[spanish,activeacute]{babel}
\usepackage{graphicx} % Imágenes
\usepackage{fancyhdr} % Cabeceras
\usepackage{listings}

\usetheme{Berlin}

\title{IberOgre y Sion Tower \\ en el \\ Hackathon OSLUCA 2010}
\author{David Saltares Márquez}

%\AtBeginSection[]
%{
%  \begin{frame}<beamer>
%    \frametitle{Índice}
%    \tableofcontents[currentsection,currentsubsection]
%  \end{frame}
%}

\begin{document}

\begin{frame}[fragile]
	\titlepage 
	
	{\scriptsize
	\begin{center}
	    \begin{verbatim}
	             https://forja.rediris.es/projects/cusl5-iberogre/
	    \end{verbatim}
	\end{center}
	}
\end{frame}

\section{IberOgre}

\begin{frame}
	\frametitle{¿Qué es IberOgre?}
    
    Wiki en español sobre desarrollo de videojuegos en 3D usando Ogre.
    
    \begin{columns}[c]
		\column{200pt}
        
		\begin{block}{Características}
            \begin{itemize}
                \item Matemáticas y uso de Ogre.
                \item Artículos, fragmentos de código y ejemplos grandes.
                \item Cubre el vacío de documentación.
                \item Apoya el uso de software libre.
                \item Ha recibido algunas colaboraciones y el beneplácito de Steve Streeting.
            \end{itemize}            
        \end{block}

		\column{100pt}
        
		\begin{center}
			\includegraphics[scale=0.08]{img/logo-iberogre.png}
		\end{center}
	\end{columns} 
\end{frame}

\section{Sion Tower}

\begin{frame}
	\frametitle{¿Qué es Sion Tower?}
    
    Videojuego del género Tower Defense
    
    \begin{columns}[c]
		\column{225pt}
        
		\begin{block}{Características}
            \begin{itemize}
                \item Ejemplifica el contenido de IberOgre.
                \item Un pequeño mago debe defender la torre sagrada ante una invasión.
                \item Varios niveles, enemigos, habilidades y trampas.
                \item Desarrollado en C++ con Ogre y OIS.
                \item Consultar su Game Design Document para más info.
            \end{itemize}            
        \end{block}

		\column{100pt}
        
		\begin{center}
			\includegraphics[scale=0.14]{img/logo-siontower.png}
		\end{center}
	\end{columns} 
\end{frame}

\section{¿Qué necesito?}

\begin{frame}
	\frametitle{¿Qué necesito?}
    
    \begin{center}
        ¡Toneladas de ayuda!
    \end{center}
    
    {\footnotesize
    \begin{columns}[t]
		\column{150pt}
    
		\begin{block}{IberOgre}
            \begin{itemize}
                \item Nuevos artículos:
                \begin{itemize}
                    \item Vectores, matrices, cuaternios...
                    \item Partículas, materiales, sombras...
                \end{itemize}
                \item Revisar artículos existentes y probar los ejemplos.
                \item ¡Cuenta tu experiencia desarrollando con Ogre!
            \end{itemize}           
        \end{block}

		\column{150pt}
        
		\begin{block}{Sion Tower}
            \begin{itemize}
                \item ¡Un logo, por el amor de Tux!
                \item Arte (hay una lista completa en el GDD).
                \item Sistema de audio
            \end{itemize}            
        \end{block}
	\end{columns} 
	}
\end{frame}

\begin{frame}
	\frametitle{¡Gracias!}
    
    \begin{center}
        Muchas gracias y... \emph{Happy coding!}
    \end{center}
\end{frame}

\end{document}
