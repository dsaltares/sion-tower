% Tipo de documento
\documentclass[16pt,spanish]{article}

% Ruta a la plantilla
\def \plpath{.}

% Paquetes
% Importaciones y paquetes %%%%%%%%%%%%%%%%%%%%%%%%%%%%%%%%%%%%%%

% Codificación
\usepackage[utf8]{inputenc}
\usepackage[spanish,activeacute]{babel}

% Paquetes extras
\usepackage{listings} % Trozos de código
\usepackage{graphicx} % Imágenes
\usepackage{fancyhdr} % Cabeceras
\usepackage{lscape}   % Apaisado
\usepackage{hyperref} % Enlaces
\usepackage{float} % para que H funcione en figure
\usepackage[left=2.5cm,top=2.5cm,right=2cm,bottom=2.5cm]{geometry}

% Configuración  %%%%%%%%%%%%%%%%%%%%%%%%%%%%%%%%%%%%%%%%%%%%%%%%%

\lstset{language=C++,
	showstringspaces=true}

% Figura centrada con \figura{proporcion}{ruta}{caption}{label}
% Ejemplo de uso: \figura{0.5}{img.png}{Boceto}{figura1}
% El argumento proporción es opcional y por defecto es el ancho del texto
%
%                 nargs   defaults
\newcommand{\figura}[4][1]{
\begin{figure}[H!] % Aparecerá justo donde está el código
	\centering
	     \includegraphics[width=#1\textwidth]{#2}
	\caption{#3}
	\label{#4}
\end{figure}
}

% Figura centrada y rotada 90º a la derecha
% con \figura{proporcion}{ruta}{caption}{label}
% Ejemplo de uso: \figura{0.5}{img.png}{Boceto}{figura1}
% El argumento proporción es opcional y por defecto es el ancho del texto
%
%                 nargs   defaults
\newcommand{\figuraApaisada}[4][1]{
\begin{figure}[H!] % Aparecerá justo donde está el código
	\centering
	     \includegraphics[width=#1\textwidth,angle=90]{#2}
	\caption{#3}
	\label{#4}
\end{figure}
}




%  Datos del documento a Editar %%%%%%%%%%%%%%%%%%%%%%%%%%%%%%%%%%%

% Título del documento
\def \titlename{}

% Autores separados por \and
\def \authorname{}

% Versión de la revisión (En blanco para documentos nuevos)
\def \revname{\ } % Ponemos un espacio para evitar errores en la cabecera.

% Fecha (En blanco para la fecha de hoy)
\def \datename{}

% Variables a usar para mantener un sistema consistente
\def \juego{ \emph {Nombre del Juego} }
\def \jugador{ \emph {Usuario del Juego} }

% Configuración  %%%%%%%%%%%%%%%%%%%%%%%%%%%%%%%%%%%%%%%%%%%%%%%%%

\title{\titlename}
\author{\authorname}
%\date{\datename}

% Cabecera y pie del documento
\pagestyle{fancy}
\renewcommand{\headrulewidth}{0.2pt}
\fancyhead[HC]{ {\footnotesize \titlename} }
\fancyhead[FR]{ {\footnotesize \thepage} }


%%%%%%%%%%%%%%%%%%%%%%%%%%%%%%%%%%%%%%%%%%%%%%%%%%%%%%%%%%%%%%%%%

\begin{document}

% Portada %%%%%%%%%%%%%%%%%%%%%%%%%%%%%%%%%%%%%%%%%%%%%%%%%

% Fichero con la portada %%%%%%

\thispagestyle{empty}
%\begin{picture}(0,0)
%	\put(-73,-35){\includegraphics[scale=0.42]{\plpath/img/cabecera.png}}
%\end{picture}\\[4cm]
	
	\begin{center}
		\makeatletter
		{\bf {\Huge \@title}}
		\\[2cm]
		{\bf {\huge \subtitlename}}\\[4cm]
		\@date\\[2cm]
		{\footnotesize Revisión \revname}
		\\[7cm]
		\begin{tabular}[t]{c} \@author \end{tabular}\\[3cm]
		\makeatother
		
		\begin{center}
			\includegraphics[scale=0.9]{\plpath/img/by-nc-sa.png}
		\end{center}
	\end{center}

\cleardoublepage




% Índice %%%%%%%%%%%%%%%%%%%%%%%%%%%%%%%%%%%%%%%%%%%%%%%%%%

\tableofcontents
\cleardoublepage

% Cambios en esta Revisión %%%%%%%%%%%%%%%%%%%%%%%%%%%%%%%%%%%%%%
%\section{Cambios}
%\label{sec:cambios} 
% El label sirve par hacer referencia a esta sección. Se usará:
%    En la Sección~\ref{sec:cambios} en la página~\pageref{sec:cambios}
%    y el resultado será:
%    En la Sección 1 en la página 3.

\paragraph{}
Cambios con respecto a versiones anteriores del documento.

\begin{itemize}
	\item {\bf Revision 1}
		\begin{itemize}
			\item Cambio1
			\item Cambio2
		\end{itemize}
\end{itemize}

%%%%%%%%%%%%%%%%%%%%%%%%%%%%%%%%%%%%%%%%%%%%%%%%%%%%%%%%%%%%%%%%%

\section{Descripción}
\label{sec:descripcion}

% No usar \\ para separar párrafos. Usar esto en su lugar.
\paragraph{}
Lorem ipsum\ldots

% Listas
\begin{itemize}
	\item itemize1
	\item itemize2
	\item itemize3
	\item itemize4
\end{itemize}

% Entorno verbatim para ejemplos
{\footnotesize ej:}
\begin{verbatim}
identificador              | categoría | descripción                                 | valor 2010  |

rebiun.metro.alumno          rebiun      Metros cuadrados por alumno                   0.8345
bau.quejas.infraestructuras  bau         Quejas recibidas sobre las infraestructuras   15321
\end{verbatim}

% Ejemplo de figura apaisada (ocupan mucho mejor el espacio)
% Se puede hacer \ref a ellas empleando el label fig:juego
\figuraApaisada{\plpath/img/cabecera.png}{Boceto de \juego}{fig:juego}

\begin{description}
	\item[Juego] Duke Nukem
	\item[Santo] Stallman
\end{description}

\end{document}
